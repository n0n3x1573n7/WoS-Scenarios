\documentclass[11pt]{report}
%Use \documentclass[12pt]{book} to printout and make it into a book
\renewcommand{\numberline}[1]{#1~}

\newif\ifDLC
\DLCfalse %\DLCtrue or \DLCfalse

\newif\iffullchangelog
\fullchangelogfalse %\fullchangelogtrue or \fullchangelogfalse

%\usepackage{kotex}
\usepackage{CJKutf8}

\usepackage[utf8]{inputenc}
\usepackage[OT1, T2A]{fontenc}

\usepackage{imakeidx}
\indexsetup{othercode=\footnotesize}
\makeindex[columns=3, title={이야기 찾아보기}]
\let\oldindex\index
\renewcommand*{\index}[1]{\oldindex{#1}\ignorespaces}
\makeatletter
\def\indexspace{}
\makeatother

\usepackage{fontspec}
\setmainfont[ItalicFont={*},ItalicFeatures={FakeSlant=.167}]{NanumBarunGothic}
\newfontfamily{\storyfont}[Scale=1.1]{SDMiSaeng.ttf}

\usepackage[hidelinks,unicode,bookmarks=true]{hyperref}
\usepackage[usenames,dvipsnames]{color}
\usepackage[round]{natbib}
\hypersetup{colorlinks,
	citecolor=Red,
	linkcolor=Green,
	urlcolor=Blue}

\usepackage[nobottomtitles]{titlesec}
\titleformat*{\section}{\LARGE\bfseries}
\titleformat*{\subsection}{\Large\bfseries}
\titleformat*{\subsubsection}{\Large\bfseries}

\usepackage{fancyvrb}
\usepackage{graphicx}
\usepackage{subfig}
\usepackage{amsmath}
\usepackage{amsthm}
\usepackage{amssymb}
%\usepackage{relsize}
\usepackage{centernot}
\usepackage[top=2cm, left=2cm, right=2cm, bottom=2cm]{geometry}
\usepackage{titling}
%\usepackage{lipsum}
\usepackage{standalone}
\usepackage{enumitem}
\setlist{nosep}

\usepackage{etoolbox}
\AfterEndEnvironment{enumerate}{\vskip-\lastskip}
\AfterEndEnvironment{itemize}{\vskip-\lastskip}

\usepackage{multirow}
\usepackage{ulem}
\usepackage{bm}
\usepackage{tikz}
\usepackage{tabularx}
\usepackage{mdframed}
\usepackage{etoolbox}

\usepackage[
type={CC},
modifier={by-nc-sa},
version={4.0},
]{doclicense}

\usepackage[yyyymmdd]{datetime}

\usepackage{makecell}

\usepackage{minitoc}
\noptcrule
\doparttoc

\usepackage{chngcntr}
\counterwithin*{chapter}{part}

\usepackage{setspace}
\renewcommand{\baselinestretch}{1.2}

\setlength{\droptitle}{-3em}

\renewcommand\mtcgapbeforeheads{0pt}

\usepackage{arydshln}

\usepackage{xifthen}
\usepackage{xparse}

\makeatletter
\newcommand\footnoteref[1]{\protected@xdef\@thefnmark{\ref{#1}}\@footnotemark}
\makeatother

\setlength\dashlinedash{0.5pt}
\setlength\dashlinegap{2.0pt}
\setlength\arrayrulewidth{0.3pt}
\newcommand{\basesepline}{\hdashline}

\newcommand{\widthratio}{0.975}

\newenvironment{tightcenter}{%
	\setlength\topsep{0pt}%
	\setlength\parskip{0pt}%
	\par\centering}{\par\noindent\ignorespacesafterend}

%environment for WoT(lite) stories, appears on index
\newenvironment{lite}[2][]
{ \ignorespaces \smallbreak \par
	\noindent
	\begin{minipage}{\linewidth}
		\begin{tightcenter}
			{\large \ifthenelse{\isempty{#1}}{\textbf{[#2]}}{\hypertarget{#1}{\textbf{[#2]}}}}\\[1ex]
			\index{#2}
			\tabular{|p{\widthratio\linewidth}|}
			\hline
		}
		{
			\endtabular
		\end{tightcenter}
	\end{minipage}
	\par
	\smallbreak
}

%environment for normal stories, appears on index
\newenvironment{story}[3][]
{ \ignorespaces \smallbreak \par
	\noindent
	\begin{minipage}{\linewidth}
		\begin{tightcenter}
			{\large \ifthenelse{\isempty{#1}}{\textbf{[#2]}}{\hypertarget{#1}{\textbf{[#2]}}}}\\[1ex]
			\index{#2}
			\tabular{|p{\widthratio\linewidth}|}
			\hline
			\textbf{속성}: #3
			\\\hline
			\ignorespaces\unskip
		}
		{
			\endtabular
		\end{tightcenter}
	\end{minipage}
	\par
	\smallbreak
}

%environment for spoiling stories(i.e. scenarios), does not appear on index
\newenvironment{spoiler}[3][]
{ \ignorespaces \smallbreak \par
	\noindent
	\begin{minipage}{\linewidth}
		\begin{tightcenter}
			{\large \ifthenelse{\isempty{#1}}{\textbf{[#2]}}{\hypertarget{#1}{\textbf{[#2]}}}}\\[1ex]
			\tabular{|p{\widthratio\linewidth}|}
			\hline
			\textbf{속성}: #3
			\\\hline
			\ignorespaces\unskip
		}
		{
			\endtabular
		\end{tightcenter}
	\end{minipage}
	\par
	\smallbreak
}

\newcommand{\storyref}[2]{\hyperlink{#1}{\textnormal{[#2]}}}

\newcounter{qnactr}

\newenvironment{faq}[1]
{
		\refstepcounter{qnactr}
		\par\smallbreak
		\textbf{\large Q\theqnactr. #1}
		\newline\rmfamily
		A\theqnactr.
}
{\par\smallbreak}

\newcommand{\statchange}[2]
{
	\textbf{스탯 #1}: #2
}

\newcommand{\entry}[2][\basesepline]
{
	\ignorespaces#2\ignorespaces\unskip\\#1
}

\newcommand{\pre}[2][\basesepline]
{
	\entry[#1]{\textbf{필요 조건}: \textit{#2}}
}

\newcommand{\cost}[2][\hline]
{
	\entry[#1]{\textbf{개연성 코스트}: #2}
}

\newcommand{\limitedtrauma}[3][\basesepline]
{
	\entry[#1]{\textbf{제약}\ifthenelse{\isempty{#2}}{}{\textbf{(#2)}}: #3}
}

\newcommand{\triggertrauma}[4][\basesepline]
{
	\entry[#1]{\textbf{트리거}\ifthenelse{\isempty{#2}}{}{\textbf{(#2)}}: #3\\\textbf{효과}: #4}
}

\newcommand{\flavour}[2][\basesepline]
{
	\entry[#1]{{\storyfont\large#2}}
}

\newcommand{\world}[1]{{\storyfont \large #1 \par}}

\setlength\parindent{0pt}

\renewenvironment{center}
{\smallskip\parskip=0pt\par\nopagebreak\centering}
{\par\noindent\ignorespacesafterend\smallskip}

\usepackage{fancyhdr}
\ifDLC
	\pagestyle{fancy}
	\fancyhf{}
	\cfoot{\thepage}
	\rhead{\textcolor{red}{DLC ENABLED}}
	\lhead{\textcolor{red}{DLC ENABLED}}
	\rfoot{\textcolor{red}{DLC ENABLED}}
	\lfoot{\textcolor{red}{DLC ENABLED}}
\fi

\title{
	World of the Storytellers Scenarios\\
	이야기꾼의 세계 시나리오집\\
	\large Last Update: \today
	\ifDLC{\\ \textcolor{red}{DLC ENABLED}}\fi
}

\author{}

\date{}

\begin{document}
	%original: https://docs.google.com/document/d/1CewodUdP8zI3f81ktz6iiZYG4b1cFzARG2g1HFeswzg/edit
	\maketitle
	
	\setcounter{tocdepth}{-1}
	
\iffalse
	\chapter*{서론}
		\input{./Chapters/서론.tex}
\fi
	
	\tableofcontents
	\part{시간의 박물관(Museum of Time)}
		\documentclass{report}

\begin{document}
	\chapter*{본 시나리오의 내용을 보고 싶지 않으신 분들께서는 \pageref{endof_MoT}쪽으로 넘어가 주시기 바랍니다.}
	
	\chapter{시놉시스와 시나리오 기본 정보}
		\documentclass{report}

\begin{document}
	\textbf{시나리오 이름}: 시간의 박물관(Museum of Time)
	
	\textbf{시나리오 작가}: None(\href{https://www.twitter.com/n0n3x1573n7_WS}{@n0n3x1573n7\_WS})
	
	\textbf{사용 룰}: 이야기꾼의 세계(World of the Storytellers)
	
	\textbf{권장 인원}: 1\textasciitilde2인
	
	\textbf{트리거 워닝}: 이 시나리오는 밀폐 공간과 어둠에 대한 묘사가 등장합니다. 이에 대해 플레이어에게 충분히 숙지시키지 않고 플레이하거나, 이 또는 이와 유사한 상황이 유발되었을 시 해당 사항을 무시하고 플레이하는 행위를 절대 금합니다.
	
	\subsubsection*{시놉시스}
	
	수많은 서사의 아티팩트들을 보관하고 있는 Chronos 박물관은 시간과 공간의 끝의 한 지점에 숨겨져 있는 박물관입니다. 이야기꾼들이 [깨달은 자]가 되었을 때에 시스템이 이야기꾼들에게 여러 세계의 기술과 마법, 종교 등을 간접 체험할 수 있도록 돕기 위해서 이 박물관을 방문하도록 권장합니다. Chronos 박물관에 방문한 이야기꾼들에게 무슨 사건이 벌어질까요?
\end{document}
	
	\chapter{시간의 박물관의 이야기(이야기꾼 공개용)}
		\documentclass{report}

\begin{document}
	\begin{story}{시간의 끝}{[시간]}
		\entry{이 서사는 이미 시간의 끝을 향해 달려가고 있기 때문에 서사의 개연성을 해칠 염려가 매우 적다. 따라서, [침범] 판정에 실패할 때, 다음 씬(비전투) 또는 두 턴 후(전투)까지 해당 판정에 실패한 이야기가 전면 봉쇄되어, 기술 뿐 아니라 해당 이야기로부터의 도움도 받을 수 없다(단, 스탯은 유지된다). 개연성 판정 난이도의 초기화는 이야기의 봉쇄가 일어나면 즉시 일어나나, 이야기가 불안정해짐에 따라 판정이 일어날 때 마다 난이도가 1 상승한다.}
	\end{story}
	
	\begin{story}{비정형의 공간}{[공간]}
		\entry{우주 상에 떠다니는 불규칙한 공간이기 때문에 명중 또는 회피 판정을 할 때에는 별도의 해당 페널티를 상쇄할 만한 이야기가 없다면 명중/회피 페널티로서 4df를 굴려 해당 수치를 판정치에 반드시 더해야 한다.}
	\end{story}
	
	\begin{story}{시간의 신 Chronos의 저주}{[저주: 신]}
		\entry{시공간 이동에 관련된 모든 기술, 마법, 흑마법, 초능력 등이 이곳에서는 봉인된다.}
		
		\entry{[일부 비공개]}
	\end{story}
\end{document}
	
	\chapter*{스포일러 방지 및 메모용 빈 페이지 입니다. 본 시나리오를 플레이하실 분께서는 \pageref{endof_MoT}쪽으로 넘어가 주시기 바랍니다.}
	
	\parttoc
	
	\chapter{시간의 박물관 시나리오의 흐름}
		\documentclass{report}

\begin{document}
	아래의 모든 내용은 이 서사에 들어오게 될 이야기꾼에 맞추어 개변하실 수 있습니다. 다만, 시나리오의 큰 흐름을 변경하거나, 개변한 시나리오를 재배포하지는 말아주세요.
	
	\section*{박물관 배경 지식}
	\world{
		크로노스 박물관.
		
		시간의 끝으로부터 멀지 않은 시간의 찰나에 존재하는, 모든 우주와 시간의 아티팩트를 모아둔, 시간 상에서 정지되어 있는 박물관이다. 모든 시간 여행자들의 일기장이자, 꿈이고, 고향과도 같은 곳이며, 수많은 도둑들이 호시탐탐 노리고 있는 곳이다.
		
		이 곳이 많은 침입을 받았으나 대부분의 경우 실패한 이유는 박물관 내부에서의 시공간에 관련된 모든 능력을 이 박물관의 가장 중요한 규칙, [시간의 신 Chronos의 저주]가 막아버리기 때문이다. 시공간 이동에 관련된 모든 기술, 마법, 흑마법, 초능력 등이 이곳에서는 봉인되며, 박물관의 시설물을 파괴할 수 있는 기술과 무기들은 즉시 압수되어 해당 인물들이 박물관 안에 있는 동안 전시 품목으로 추가된다.
		
		이 박물관은 여러 프로토콜들로 자동화되어 운영되고 있다. 그 중 대표적으로는 시간상에서 아티팩트를 수집해오는 Kairos 프로토콜, 그 아티팩트를 보존하는 Aion 프로토콜이 있다.
		
		아티팩트들은 위험도에 따라 Atropos, Clotho, Lachesis의 세 단계로 나뉘는 Moirai 척도로 분류된다. Moirai 척도의 기준은 해당 아티팩트가 도둑맞아 잘못된 시대로 갔을 때의 위험도를 판분류된다. 여기서의 위험도는 개연성을 해침으로서 시간패러독스가 발생할 위험도를 의미한다.
		
		Atropos는 전혀 위협적이지 않은 경우. 대부분의 보석류와 제한적인 고전 무기 등이 이에 속한다.
		
		Clotho는 해당 시대 이전에만 위협적인 경우. 대부분의 무기와 기술이 이에 속한다.
		
		Lachesis는 언제든지 위협을 가할 여지가 있는 경우.
		
		Lachesis급의 아티팩트가 도둑맞으면 우주의 안정성 자체가 위협을 받을 수 있으므로 가져가게 둘 바에는 박물관을 폐쇄 후 폭파하도록 되어 있으며, 이를 Moros 프로토콜이라 칭한다.
		
		아티팩트는 모두 홀로그램과도 같은 phasing 상태로 보관되어 있다. 이는 시간상의 한 조각을 떼어다가 저장한 것으로, 실물 크기의 실제 물건이지만 사용하거나 건드리지는 못하도록 되어 있다. 이는 Aion 프로토콜이 담당하고 있다.
	}
	
	\bigskip
	
	프로토콜명과 그 이름을 따온 그리스 신의 역할, 그리고 이 서사에서 해당 프로토콜의 역할을 정리하면 다음과 같습니다:
	
	\begin{itemize}
		\item Chronos: 시간 그 자체 - 도서관명, 총괄 인공지능
		\item Kairos: 기회의 신 - 수집 프로토콜
		\item Aion: 영원의 신 - 보존 프로토콜
		\item Moirai: 운명의 세 여신 - 위협 단계
		\begin{itemize}
			\item Atropos: 물레 - 평상 단계
			\item Clotho: 자 - 위협 단계
			\item Lachesis: 가위 - 위험 단계
		\end{itemize}
		\item Moros: 죽음의 신 - 자폭 프로토콜
	\end{itemize}
	
	\section*{[태초의 이야기](선택)}
	시스템과 이야기꾼이 처음으로 만나게 됩니다. 시스템은 [태초의 이야기]에 대해 설명을 해주고는, 이야기꾼들이 시간의 박물관에 가도록 유도합니다.
	
	\section*{도입부}
	박물관에 들어온 이야기꾼들은 처음으로 크로노스를 만나게 됩니다. 크로노스는 실체가 있어도 좋고, 텍스트와 음성 인터페이스로만 이루어져 있어도 좋습니다. 크로노스는 이 박물관의 관장이자 총괄 인공지능이고, 또 다른 [깨달은 자]입니다.
	
	이야기꾼들은 로비로 들어와 크로노스와 대화를 나눕니다. 적당한 시점에 끊고, 박물관을 관람하도록 합시다.
	
	\section*{박물관 관람}
	박물관의 지도를 공개할때에, 서버실은 아직 어떤 방인지 공개되지 않아야 합니다. 서버실 앞에는 [관리자 외 출입 금지] 표지판이 붙어 있다는 사실만을 알 수 있습니다. 서버실이 어떤 방인지를 알기 위해서는 크로노스에게 직접 물어보거나, 서버실에 직접 들어가야 합니다. 이 시점에서 서버실은 잠겨있습니다.
	
	이야기꾼들은 박물관의 네 가지 전시실을 모두 돌아볼 수 있습니다. 이 때에는 각 아티팩트의 개략적인 설명만을 해주도록 합시다. 이 아티팩트들의 실제 효과를 이 때에 알 수 있는 방법은 감정사 계열의 이야기를 가지고 있거나, 해당 아티팩트의 능력을 분석할 수 있는 적당한 지식이 있어야 합니다.
	
	두 번째 전시실에서 나올때에, 복도에서 조금 위화감이 든다는 언급을 해주세요. 만약 뭐가 위화감이 드는지 알아보고 싶다면, 이야기꾼들은 성공치가 3인 인식 판정을 해야합니다. 실패한다면 아무것도 알 수 없으나, 성공한다면 서버실의 문이 조금 열려있다는 사실을 알 수 있습니다. 지금 서버실에 대해 질의를 한다면, 크로노스는 "왜 열려있는지 모르겠다"는 반응입니다. 왜 그런지 확인해볼테니, 남은 전시실을 관람하라고 할 것입니다.
	
	이야기꾼들이 이 때 바로 서버실을 확인하든 전시실을 모두 관람한 후 확인하든 상관 없이 서버실을 확인할 수 있습니다. 서버실에서 성공치가 5인 인식 판정을 성공한다면 메모리카드가 본체 하나에서 빼꼼 삐져나와 있는 것을 확인할 수 있습니다. 만약 물어본다면 크로노스는 이 메모리카드에 대해 아는 바가 없으며, 이 박물관 안에는 이야기꾼들밖에 인식할 수 없다고 대답합니다. 이 메모리카드에는 Janus 바이러스가 들어있습니다.
	
	이 메모리카드는 제거되면 수장고에서 나와 있으면서 아티팩트를 소지하고 있지 않은 모든 도적단원들이 박물관 밖으로 퇴출되어 행동불능에 빠지는 매우 중요한 메모리카드입니다.
	
	\section*{헤르메스 도적단과의 조우}
	전시실을 모두 관람한 후 긴 시간동안 서버실을 확인하지 않거나, 서버실에 입장하고 잠시 후, 로비쪽에서 쿵 하는 커다란 소리가 들립니다. 로비로 달려나가면 지하의 빈 공간(지하 수장고입니다.)으로 이어져 있는 사다리를 볼 수 있으나, 지하는 너무 캄캄해서 잘 보이지 않습니다. 지하는 광원이 없다면 주변 한 구역까지만을 볼 수 있는 어둠에 잠겨 있습니다.
	
	크로노스는 그 후 Lachesis급의 도구의 무더기를 누군가 헤집는 것을 감지합니다. 이상하게도, 누가 헤집는지는 감지되지 않았지만요. 크로노스는 이야기꾼들에게 다음 이야기를 지급합니다:
	\begin{story}{도슨트}{[박물관]}
		\entry{현재 특별 전시중인 아티팩트들의 효과를 정확하게 알 수 있다.}
		
		\entry{현재 특별 전시중인 아티팩트들을 사용할 수 있게 된다.}
		
		\entry[\hline]{지하 수장고에 있는 무기를 되찾는다면 사용할 수 있다.}
	\end{story}
	여기에서 전투에 들어가기 전, 이야기꾼들은 아티팩트를 집기 위해 전시실로 되돌아갈 수 있습니다. 하지만 이야기꾼들이 집은 아티팩트 하나당 도적단 두목이 한 턴을 먼저 진행할 수 있다는 점을 반드시 상기시켜 주세요. 또한 도적단이 행동을 할 때, 도적단 두목이 [명령] 이야기를 사용할 수 있다는 점을 기억하세요.
	
	\section*{헤르메스 도적단의 목표와 행동}
	헤르메스 도적단의 목적은 이 박물관의 수장고 안에 들어있는 Lachesis급의 도구인 [헤르메스의 지팡이]를 이 박물관 밖으로 반출하는 것입니다. 이들이 행해야 하는 행동은 다음과 같습니다:
	\begin{enumerate}
		\item 이야기꾼들이 박물관을 관람하는 동안:
		\begin{itemize}
			\item 도적 2가 박물관 서버실로 침투하여 Janus 바이러스를 심습니다.
			\begin{itemize}
				\item 이야기꾼들이 위화감을 느낄 때에는 도적 2가 바이러스를 심은 후 나올때입니다. 이때 서버실에 들어가도 아무도 없습니다.
			\end{itemize}
			\item 이 바이러스는 Chronos로부터 도적단을 숨겨줍니다.
		\end{itemize}
		\item 지하로 내려가는 타일을 깨트려 내려갑니다.
		\item 두목은 전투가 시작한 후 한 턴간 아티팩트를 찾아 획득합니다.
		\item 그 이후 서버실로 가서 봉쇄 상태에 빠진 주 출입구를 열어야 합니다.
		\begin{itemize}
			\item 최초에는 도적 2가 시도합니다. 도적 2가 행동불능에 빠진 후에는 도적 1이, 도적 1이 행동불능에 빠진 후에는 도적단 두목이 시도합니다.
			\begin{itemize}
				\item 시도하는 도적은 받는 모든 공격을 무시하고 서버실로 돌진합니다.
			\end{itemize}
			\item 서버실에 도착하면 한 턴을 소모해 도적은 주 출입구를 열기로 시도할 수 있습니다. 이 경우, 지식으로 판정합니다. 성공치는 3입니다.
			\begin{itemize}
				\item 대실패한 경우, 자폭 타이머가 시작하여 3라운드 후 박물관이 폭발합니다.
				\begin{itemize}
					\item 이 상태에서 다시 시도하여 성공한 경우, 자폭 타이머를 임시로 멈출 수 있습니다.
					\item 대성공한 경우, 자폭 타이머가 초기화됩니다.
				\end{itemize}
				\item 실패한 경우, 아무 일도 일어나지 않습니다.
				\item 통과 또는 성공한 경우, 주 출입구가 열리나 모든 전시실의 문이 닫힙니다.
				\item 대성공한 경우, 주 출입구가 열립니다.
			\end{itemize}
		\end{itemize}
		\item 주 출입구가 열린 후, 누군가가 아티팩트를 들고 빠져나가야 합니다.
		\begin{itemize}
			\item 아티팩트를 던져서 출입구 밖으로 반출하면, 아티팩트는 크로노스에 의해 다시 반입됩니다.
			\item 또는, 난이도를 높이기 위해서 네 번째 도적이 밖에 존재하여 아티팩트를 던져서 아티팩트가 박물관에서 나가는 순간 아티팩트를 훔치는 데에 성공한 것으로 할 수 있습니다.
		\end{itemize}
	\end{enumerate}
	
	\section*{결말}
	세 가지 결론이 나올 수 있습니다.
	\begin{enumerate}
		\item 모든 도적단이 행동불능에 빠진다.
		\begin{itemize}
			\item 도적단은 모두 체포되어 처벌을 받습니다.
			\item 박물관은 이야기꾼들에게 감사하며 이야기꾼(들)에게 원하는 아티팩트 한가지를 지급합니다. 이 아티팩트는 코스트는 없으나, 레플리카 버전이기 때문에 한 씬 동안만 사용할 수 있습니다.
			\item 이야기꾼들은 다음 이야기를 얻습니다:
			\begin{story}{헤르메스 도적단 체포}{[설화]}
				\cost{0}
			\end{story}
		\end{itemize}
		
		\item 도적단이 [헤르메스의 지팡이]를 들고 도망치는데에 성공한다.
		\begin{itemize}
			\item 박물관은 보안 점검을 위해서 폐쇄에 들어갑니다.
			\item 이야기꾼들은 실망한채로 [태초의 이야기]로 되돌아갑니다.
		\end{itemize}
		
		\item 박물관의 자폭 시퀀스가 발동되어, 모두와 함께 폭발한다.
		\begin{itemize}
			\item 박물관은 시공간 상에서 사라집니다.
			\item 이야기꾼들은 다음 기피증을 얻고 [태초의 이야기]로 되돌아갑니다:
			\begin{story}{폭발}{[기피]}
				\triggertrauma{기피}{자신이 폭발물의 효과 반경 안에 존재한다는 사실을 알고 있다.}{해당 폭발물의 효과 반경에서 벗어나기 전까지지 모든 판정에 -1을 얻는다.}
				
				\entry{이 이야기는 일주일동안 유지된다.}
				
				\cost{0}
			\end{story}
		\end{itemize}
	\end{enumerate}
\end{document}
	
	\chapter{시간의 박물관의 이야기}
		\documentclass{report}

\begin{document}
	이야기꾼들에게 비공개된 정보의 텍스트는 빨간색으로 표시되어 있습니다.
	
	\begin{story}{시간의 끝}{[시간]}
		\entry{이 서사는 이미 시간의 끝을 향해 달려가고 있기 때문에 서사의 개연성을 해칠 염려가 매우 적다. 따라서, [침범] 판정에 실패할 때, 다음 씬(비전투) 또는 두 턴 후(전투)까지 해당 판정에 실패한 이야기가 전면 봉쇄되어, 기술 뿐 아니라 해당 이야기로부터의 도움도 받을 수 없다(단, 스탯은 유지된다). 개연성 판정 난이도의 초기화는 이야기의 봉쇄가 일어나면 즉시 일어나나, 이야기가 불안정해짐에 따라 판정이 일어날 때 마다 난이도가 1 상승한다.}
	\end{story}
	
	\begin{story}{비정형의 공간}{[공간]}
		\entry{우주 상에 떠다니는 불규칙한 공간이기 때문에 명중 또는 회피 판정을 할 때에는 별도의 해당 페널티를 상쇄할 만한 이야기가 없다면 명중/회피 페널티로서 4df를 굴려 해당 수치를 판정치에 반드시 더해야 한다.}
	\end{story}
	
	\begin{story}{시간의 신 Chronos의 저주}{[저주: 신]}
		\entry{시공간 이동에 관련된 모든 기술, 마법, 흑마법, 초능력 등이 이곳에서는 봉인된다.}
		
		\entry{\textcolor{Red}{박물관의 시설물을 파괴할 수 있는 기술과 무기들은 즉시 압수되어 해당 인물들이 박물관 안에 있는 동안 전시 품목으로 추가되며, 허가받지 않고 이들을 훔치고자 한 이들은 모두 즉시 시간상에서 사라진다.}}
	\end{story}
\end{document}
	
	\chapter{박물관 - 지상 전시실 지도와 전시물품}
		\documentclass{report}

\begin{document}
	
	\section*{지상 전시실 지도}
	\begin{tabular}{!{\color{black}\vrule}p{3cm}!{\color{black}\vrule}p{3cm}!{\color{black}\vrule}p{3cm}!{\color{black}\vrule}p{3cm}!{\color{black}\vrule}p{3cm}!{\color{black}\vrule}p{3cm}!{\color{black}\vrule}}
		\hline
		\multirow{5}{*}{로비} & \multicolumn{2}{p{3cm}!{\color{black}\vrule}}{\multirow{2}{*}{보석 전시실}} & \multicolumn{2}{p{3cm}!{\color{black}\vrule}}{\multirow{2}{*}{마법 전시실}} & \multirow{5}{*}{보안실} \\
		& \multicolumn{2}{p{3cm}!{\color{black}\vrule}}{}                        & \multicolumn{2}{p{3cm}!{\color{black}\vrule}}{}                        &                      \\ \cline{2-5}
		& \multicolumn{4}{p{6cm}!{\color{black}\vrule}}{복도}                                                                     &                      \\ \cline{2-5}
		& \multicolumn{2}{p{3cm}!{\color{black}\vrule}}{\multirow{2}{*}{기술 전시실}} & \multicolumn{2}{p{3cm}!{\color{black}\vrule}}{\multirow{2}{*}{종교 전시실}} &                      \\
		& \multicolumn{2}{p{3cm}!{\color{black}\vrule}}{}                        & \multicolumn{2}{c!{\color{black}\vrule}}{}                        &                      \\ \hline
	\end{tabular}
	
	\bigskip
	
	박물관의 네 가지 전시실에는 각각 네 가지씩의 아티팩트들이 특별 전시품으로서 전시되어 있습니다. 이 아티팩트들은 이야기꾼에 따라 변경하는 것을 권장하며, 이야기꾼들이 \emph{사용하고 싶도록} 만들어야 합니다. 예를 들어, 이야기꾼들의 능력의 페널티를 상쇄시킨다거나 하는 식으로요. 아래의 표에는 존재할만한 아티팩트들을 나열해두었습니다.
	
	\section*{보석 전시실}
	\begin{tabularx}{\textwidth}{l!{\color{black}\vrule}l!{\color{black}\vrule}X!{\color{black}\vrule}l!{\color{black}\vrule}l!{\color{black}\vrule}l}
		\textbf{속성} & \textbf{명칭} & \textbf{능력} & \textbf{스탯 +} & \textbf{스탯 -} & \textbf{코스트}\\ \hline \hline
		[저주][보석]& 루비   & 소유자는 한 턴에 한 번, 개연성을 1 소모하고 한 구역 내의 모든 대상에게 회피 불가능의 물리 또는 정신 피해를 1 줄 수 있다.   & 자본     & & 0    \\ \hline
		[저주][보석]& 사파이어   & 소유자는 한 턴에 한 번, 개연성을 1 소모하고 한 구역 내의 모든 대상에게 물리적 상태 [감전됨 □] 또는 정신적 상태 [멍해짐 □]을 줄 수 있다.   & 자본     & & 0    \\ \hline
		[저주][보석]& 오팔   &  소유자는 한 턴에 한 번, 개연성을 1 소모하고 한 구역 내의 자신을 제외한 모든 대상의 체력 또는 정신력을 1 회복시킬 수 있다. 개연성은 회복시킬 수 없다.  & 자본     & & 0    \\ \hline
		[저주][보석]& 에메랄드   & 소유자는 한 턴에 한 번, 개연성을 1 소모하고 이동을 1회 추가로 할 수 있다.   & 자본     & & 0    \\ 
	\end{tabularx}
	
	\section*{마법 전시실}
	\begin{tabularx}{\textwidth}{l!{\color{black}\vrule}l!{\color{black}\vrule}X!{\color{black}\vrule}l!{\color{black}\vrule}l!{\color{black}\vrule}l}
		\textbf{속성} & \textbf{명칭} & \textbf{능력} & \textbf{스탯 +} & \textbf{스탯 -} & \textbf{코스트} \\ \hline \hline
		[마법][마나]& 불안정한 수정 & 마나를 사용한다면, 최대 마나가 10\% 증가한다. 이 아티팩트를 공중으로 던지면 폭발하여 자신 외의 같은 구역 안에 있는 모든 이에게 [실명됨: 1턴]을 준다. &  & & 0 \\ \hline
		[마법][목걸이]& 예지의 목걸이 & 착용자는 회피와 조준 판정에 +1을 받는다. 이 목걸이를 파괴함으로서 자동 성공을 결과로 가질 수 있다. &  & & 0 \\ \hline
		[흑마법][혈액] & 응고된 혈액 & 매 턴 정신력 1을 소모한다. 정신력이 0이 되면 이 아티팩트는 영구히 소실된다. &  & & -10 \\ \hline
		[마법][시계] & 시간의 회중시계 & 자신의 턴에 주사위에 의한 판정([행운] 등)을 할 때, 두 번 굴려 그 중 하나를 선택할 수 있다.& 속도 & & 0 \\ \hline
		[마법][목걸이]& 민첩의 목걸이 & 착용자는 회피 판정이나 조준 판정을 함에 있어 +1을 받는다. &  & & 0 \\
	\end{tabularx}
	
	\section*{기술 전시실}
	\begin{tabularx}{\textwidth}{l!{\color{black}\vrule}l!{\color{black}\vrule}X!{\color{black}\vrule}l!{\color{black}\vrule}l!{\color{black}\vrule}l}
		\textbf{속성} & \textbf{명칭} & \textbf{능력} & \textbf{스탯 +} & \textbf{스탯 -} & \textbf{코스트}\\ \hline \hline
		[기술][생물]& 기계 공생체 & 한 턴을 소모해 혈액에 심을 수 있다.\newline 심기면 훔칠 수 없어지며, 이동을 포기하면 보호막 3을 얻을 수 있고, 다음 스탯에 변화를 준다: \newline \textbf{스탯+}: 기민, 근력 \newline \textbf{스탯-}: 의지, 공감, 인식  &   &     & 0 \\ \hline
		[기술][환상]& 테서렉트 & 누군가 개연성 판정에 실패할 때, 테서렉트의 코스트가 2 증가한다. \newline 테서렉트의 코스트가 0이 되면 테서렉트가 폭발하며 소유자를 제외한 이들의 시간이 잠시 멈춘다. 즉시 한 턴을 진행한다. &  &      & -10 \\ \hline
		[기술][안정]& 댐퍼 & 자신의 턴이 종료될 때, 4df를 굴려 해당 값의 절대값에 1을 뺀 만큼의 개연성을 회복할 수 있다.  &  &      & 0 \\ \hline
		[기술][무기]& 죽음의 키스 & 단 한 번 발사할 수 있는 저격총. 사격 또는 사격:총기의 두 배 중 높은 쪽으로 판정하고, 인식과 기민 중 낮은 쪽으로 회피한다. 적중한다면, 해당 적의 체력을 1 남기고 모두 잃게 한다.  &  &      & 0 \\ \hline
		[기술][무기]& 레이저 건 & 턴당 한 번, 시야가 확보된 대상에게 회피 불가능한 피해 1을 주는 레이저를 발사한다. &  & & 0\\
	\end{tabularx}
	
	\section*{종교 전시실}
	\begin{tabularx}{\textwidth}{l!{\color{black}\vrule}l!{\color{black}\vrule}X!{\color{black}\vrule}l!{\color{black}\vrule}l!{\color{black}\vrule}l}
		\textbf{속성} & \textbf{명칭} & \textbf{능력} & \textbf{스탯 +} & \textbf{스탯 -} & \textbf{코스트}\\ \hline \hline
		[신성][십자가]& 순교자의 십자가 & 십자가를 소유한 상태로 이야기가 봉쇄되면, 방어막 3을 얻는다. &  & & 0 \\ \hline
		[신성][묵주]& 대주교의 묵주 & 묵주를 소유한 상태로 이야기가 봉쇄되면, 자신을 포함한 한 대상의 체력 2를 회복시킨다. &  & & 0 \\ \hline
		[신성][기도]& 성기사의 방패 & 구역 내에서 방패를 들고 무릎을 꿇은 채로 정신을 집중하고 있는 동안, 해당 구역에서 나갈수도 들어올 수도 없는 방벽이 생성된다. 이 방벽은 정신집중을 해제하거나, 안팎을 통틀어 10의 피해를 받으면 사라진다.  &  & & 0 \\ \hline
		[신성][토템]& 대정령의 토템 & 한 턴을 소모해 토템을 설치하거나 철거할 수 있다. 설치된 상태에서 같은 구역에 있는 모든 이들은 체력 1을 정신력 1, 또는 정신력 1을 체력 1으로 바꿀 수 있다. &  & & 0 \\
	\end{tabularx}
\end{document}
	
	\chapter{박물관 - 지하 수장고 지도와 이야기}
		\documentclass{report}

\begin{document}
	\begin{tabular}{!{\color{black}\vrule}p{2cm}!{\color{black}\vrule}p{2cm}!{\color{black}\vrule}p{2cm}!{\color{black}\vrule}p{2cm}!{\color{black}\vrule}p{2cm}!{\color{black}\vrule}p{2cm}!{\color{black}\vrule}}
		\hline
		&     &  &     &  &    \\ \hline
		&     &  &     &  &    \\ \hline
		\makecell{\centering 사다리} & \makecell{\centering 도적2} &  & \makecell{\centering 도적1} &  & \makecell{\centering 두목} \\ \hline
		&     &  &     &  &    \\ \hline
		&     &  &     &  &    \\ \hline
	\end{tabular}
	
	\bigskip
	
	모든 칸에 해당하는 이야기입니다:
	\begin{story}{어지럽게 얽히고설킨 창고}{[장소]}
		\entry{한 턴에 두 번까지 이동할 수 있습니다.}
		
		\entry[\hline]{바닥을 통해 이동할 때, 4df를 굴립니다. -2 이하의 결과가 나오면 다음 자신의 턴까지 유지되는 물리적인 부정적 상태 [균형을 잃음]을 얻고, 이번 턴에는 더 이상 이동할 수 없습니다.}
	\end{story}
	
	\bigskip
	
	사다리 칸에 해당하는 이야기입니다:
	\begin{story}{사다리}{[사물]}
		\entry[\hline]{고정된 철제 사다리가 놓여있는 칸입니다. 한 턴을 소모해 로비로 이동할 수 있습니다. 단, 같은 구역에 있는 이들의 근력에 의해 끌어내려질 수 있으나, 이를 근력 또는 민첩으로 대항하여 성공한다면 무사히 로비로 이동할 수 있습니다.}
	\end{story}
	
	만약 진입한 이야기꾼 중 무기를 [시간의 신 Chronos의 저주]에 의해 빼앗긴 이가 있다면, 지하 수장고 내의 무작위 칸에 무기가 숨겨져 있습니다. Chronos는 이 무기의 위치를 요청한다면 알려줄 것이지만 이 무기를 회수하는 것은 이야기꾼의 몫입니다.
\end{document}
	
	\chapter{헤르메스 도적단}
		\documentclass{report}

\begin{document}
	\section*{도적단 두목}
	\textbf{체력}: 15(2인일 시 20), \textbf{정신력}: 10(2인일 시 15)
	
	\textbf{스탯}: \textbf{기민} 3, \textbf{도발} 1, \textbf{의지} 1, \textbf{전투} 1, 나머지 0
	
	\begin{story}[thief-leader:steal]{훔치기}{[도적단]}
		\entry[\hline]{같은 구역 안에 있는 한 대상을 지정한다. 그 대상이 가지고 있는 무작위 물체를 훔칠 수 있다. 단, 대상은 인식 판정을 하여 훔친 이의 기민 이상이 나온다면 이를 저지하여 훔친 이에게 피해 1을 입히고, 훔치기를 실패로 할 수 있다.}
	\end{story}
	
	\begin{story}{표적}{[도적단]}
		\entry[\hline]{방 안에 있는 한 가지 물체를 표적으로 지정할 수 있다. 해당 물체를 \storyref{thief-leader:steal}{훔치기} 할 때에 피해를 입었더라도 \storyref{thief-leader:steal}{훔치기}에 성공한다. 들키지 않았다면 다른 물체를 즉시 다시 표적으로 지정할 수 있으며, 들켰다면 다음 씬에 지정할 수 있다.}
	\end{story}
	
	\begin{story}{명령}{[도적단]}
		\entry[\hline]{자신의 턴 대신 같은 공간(지하층, 각 전시실, 복도, 서버실, 로비 각각을 한 공간으로 친다.)에 있는 모든 도적의 턴을 진행할 수 있다.}
	\end{story}
	
	\begin{story}{헤르메스의 지팡이}{[아티팩트]}
		\entry{공중에 떠서 이동할 수 있다. 아무도 없는 칸을 통해서라면 2회 이동할 수 있다.}
		
		\entry[\hline]{\statchange{+}{기민\footnote{도적단 두목의 [기민] 수치는 [헤르메스의 지팡이]가 이미 적용된 수치입니다.}}}
	\end{story}
	
	\begin{story}{도적 두목의 단도}{[아이템]}
		\entry[\hline]{한 턴에 한 번, 칼을 휘둘러 같은 구역에 있는 대상에게 회피 불가능한 피해 4를 준다.}
	\end{story}
	
	\begin{story}{활}{[아이템]}
		\entry[\hline]{한 턴 장전 후 발사한다. 사격으로 판정하고 기민으로 회피할 수 있다. 적중시, 개연성에 피해를 2 준다.}
	\end{story}
	
	\section*{도적 1}
	\textbf{체력}: 15, \textbf{정신력}: 10
	
	\textbf{스탯}: \textbf{기민} 1, 나머지 0
	
	\begin{story}[thief-1:steal]{훔치기}{[도적단]}
		\entry[\hline]{같은 구역 안에 있는 한 대상을 지정한다. 그 대상이 가지고 있는 무작위 물체를 훔칠 수 있다. 단, 대상은 인식 판정을 하여 훔친 이의 기민 이상이 나온다면 이를 저지하여 훔친 이에게 피해 1을 입히고, 훔치기를 실패로 할 수 있다.}
	\end{story}
	
	\begin{story}{빠른 손발}{[도적단]}
		\entry[\hline]{\storyref{thief-1:steal}{훔치기}를 할 때에 한해서 자신의 기민에 +1. 또한, 다른 행동을 하지 않는다면 한 턴에 아무도 없는 칸을 통해서(시작칸 기준) 2회 이동할 수 있다.\footnote{이 이야기로 인해 \storyref{thief-1:steal}{훔치기} 한정 \textbf{기민}이 2가 됩니다.}}
	\end{story}
	
	\begin{story}[thief-1:dagger]{도적의 단도}{[도적단]}
		\entry[\hline]{한 턴에 한 번, 칼을 휘둘러 같은 구역에 있는 대상에게 회피 불가능한 피해 2를 준다.}
	\end{story}
	
	\begin{story}[thief-1:taunt]{도발}{[버서커]}
		\entry{같은 구역에 있는 한 대상을 선택하여 대상에게 의지 판정을 하게 한다. 만약 자신의 도발이 더 높다면, 다음 턴에는 해당 대상은 반드시 자신을 공격해야 한다.}
		
		\entry[\hline]{\storyref{thief-1:berserker-change}{상태 변화: 버서커} 발동시 "\textbf{스탯+}: 도발"을 추가로 가진다.}
	\end{story}
	
	\begin{story}[thief-1:berserker-change]{상태 변화: 버서커}{[생애]}
		\entry{체력이 5 이하로 떨어지면, 이 능력과 \storyref{thief-1:taunt}{도발}의 스탯 변화가 해제되고 한번에 피해 1 이상을 받을 수 없게 되나, \storyref{thief-1:steal}{훔치기}와 \storyref{thief-1:dagger}{도적의 단도}가 봉인된다. 한 턴에 한 번, 같은 구역 안에 있는 대상에게 피해 (6-현재 체력)을 줄 수 있다. 피해량 이상의 기민으로 회피할 수 있다.}
		
		\entry[\hline]{발동시 다음을 추가로 가진다:
		
		\statchange{+}{근력, 도발}
		
		\statchange{-}{기만, 은신, 공감, 의지\footnote{해당 능력이 발동되면, 스탯이 다음과 같이 변한다:  \textbf{기만} -1, \textbf{은신} -1, \textbf{공감} -1, \textbf{의지} -1, \textbf{근력} 1, \textbf{기민} 1, \textbf{도발} 2, 나머지 0}}}
	\end{story}
	
	\section*{도적 2}
	\textbf{체력}: 10, \textbf{정신력}: 10
	
	\textbf{스탯}: \textbf{기민} 1, 나머지 0
	
	\textbf{상태}: \textbf{은신} □□
	
	\begin{story}[thief-2:steal]{훔치기}{[도적단]}
		\entry[\hline]{같은 구역 안에 있는 한 대상을 지정한다. 그 대상이 가지고 있는 무작위 물체를 훔칠 수 있다. 단, 대상은 인식 판정을 하여 훔친 이의 기민 이상이 나온다면 이를 저지하여 훔친 이에게 피해 1을 입히고, 훔치기를 실패로 할 수 있다.}
	\end{story}
	
	\begin{story}{완벽한 은신}{[도적단]}
		\entry[\hline]{한 턴을 소모해 [은신 □□]를 얻는다. 이 상태가 있는 동안 다른 이들에게 보이지 않는다. 피해를 받거나 이동하면 [은신] 상태 한 칸이 소모된다. 자신의 턴이 시작할 때 같은 칸에 있는 적 한 명당 [은신] 한 칸이 소모된다. \storyref{thief-2:steal}{훔치기}의 판정을 잔여 은신 상태+은신 스탯 또는 기민 스탯 중 높은 쪽으로 판정하나, 실패할 시 모든 [은신] 상태를 잃는다.}
	\end{story}
	
	\begin{story}{초심자의 행운}{[도적단]}
		\entry[\hline]{매 턴 한 번, 주사위를 굴리는 판정에서 재굴림을 시도할 수 있다.}
	\end{story}
	
	\begin{story}{도적의 표창}{[아이템]}
		\entry[\hline]{한 턴에 두 번, 표창을 던져 한 대상에게 피해 1을 줄 수 있으나, 기민 또는 인식 중 높은쪽이 자신의 기민보다 낮다면 회피한다. 떨어진 구역당 회피에 +1을 받는다.}
	\end{story}
\end{document}
		
	\chapter{패치 노트}
		\documentclass{report}

\begin{document}
	\section*{2019. 07. 12. 시나리오 아이디어 생성 및 작성 시작}
	
	\section*{2019. 07. 23. 1차 베타테스트}
	베타테스트에 참여해주신 철화구야선생님께 감사를 드립니다.
	
	\section*{2019. 08. 16. 2차 베타테스트}
	베타테스트에 참여해주신 소낙님게 감사를 드립니다.
	
	\section*{2019. 09. 04. 배포본 완성}
	
	\section*{2019. 10. 24. 트리거 워닝 추가 및 룰북에 합병}
	박물관 수장고에 대한 밀폐 공간과 어둠에 대한 트리거 워닝을 추가했습니다.
	
	\section*{2019. 11. 13. 이야기 링크}
	본 룰북의 이야기 링크 기능의 추가를 이용해 이야기 링크를 추가했습니다.

	\section*{2019. 11. 13. 전시품 목록 순서 재정렬}
	전시품 목록의 순서를 재정렬하고, 오타를 수정했습니다.
\end{document}
	
	\chapter*{스포일러 방지 및 메모용 빈 페이지 입니다.}
\end{document}
	
	\part{나갈 수 없는 탑}
		\label{endof_MoT}
		\documentclass{report}

\begin{document}
	\chapter*{본 시나리오의 내용을 보고 싶지 않으신 분들께서는 \pageref{endof_Tower}쪽으로 넘어가 주시기 바랍니다.}
	
	\chapter{시놉시스와 시나리오 기본 정보}
		\documentclass{report}

\begin{document}
	\textbf{시나리오 이름}: 나갈 수 없는 탑
	
	\textbf{시나리오 작가}: 소낙(\href{https://twitter.com/knock_tr}{@knock\_tr})
	
	\textbf{사용 룰}: 이야기꾼의 세계(World of the Storytellers)
	
	\textbf{권장 인원}: 1인
	
	\textbf{트리거 워닝}: 이 시나리오는 플레이어에게 감정적 고통을 유발시킬 수 있는 내용을 포함하고 있습니다. 이에 대해 플레이어에게 충분히 숙지시키지 않고 플레이하는 행위를 금합니다.
	
	\subsubsection*{시놉시스}
	
	이야기꾼은 어떤 마법사의 제자 역할이 되어 탑에 들어가게 됩니다. 마법사의 일을 돕는 척 하며 마법사의 탑에 숨겨진 마석을 찾아야 합니다.
\end{document}
%
	
	\chapter{등장인물의 이야기(이야기꾼 공개용)}
		\documentclass{report}

\begin{document}
	\section*{마법사의 제자}
	
	\begin{story}{마법사의 제자}{[역할]}
		\flavour{나는 제자인가, 잡일꾼인가……. 바쁘다는 핑계로 아무것도 가르쳐 주지 않은 마법사는 이번 논문의 집필이 끝나면 꼭 수업을 시작해 주겠다고 약속했다. "이번에는 진짜죠?"}
		\entry[\hline]{\statchange{+}{의지}}
	\end{story}
\end{document}
%
	
	\chapter*{스포일러 방지 및 메모용 빈 페이지 입니다. 본 시나리오를 플레이하실 분께서는 \pageref{endof_Unliving}쪽으로 넘어가 주시기 바랍니다.}
	
	\parttoc
	
	\chapter{진상}
		\documentclass{report}

\begin{document}



\end{document}

	
	\chapter{진행}
		\documentclass{report}

\begin{document}
	\section{Intro phase : 시스템 파트}
	
	시스템은 이야기꾼에게 제목이 알려지지 않은 한 서사를 소개합니다. 서사 속에서 중요한 역할을 하는 마석이 행방불명되었는데, 타락한 자들이 서사를 오염시키기 위해 그 마석을 서사 속 변방의 마법사의 거처에 숨겨두었다고 합니다.
	
	시스템은 서사를 정상화시키기 위해 마석을 찾아 달라고 이야기꾼에게 부탁하며, 이를 위해 마법사의 탑에 출입하기에 적절한 역할 이야기를 부여할 것이라고 설명합니다. 이야기꾼이 마석을 찾아낸다면 시스템은 마석을 본디 있어야 할 곳으로 되돌릴 것입니다. 이야기꾼은 \storyref{quest:magicstone}{암룡의 마석 회수}를 받고 서사 속으로 진입합니다.
	
	
	
	
	\section{Main phase : 서사 파트}
	
	서사에 입장한 이야기꾼이 유기물로 된 지성체라면 외형은 거의 변하지 않습니다. 아니라면 이야기꾼은 유기물로 된 지성체의 형태 중 원래 이야기꾼의 모습에 가까운 것으로 변합니다. 마법사의 탑의 서재에서 시작합니다.
	
	세션은 마법사의 탑 내에서 진행됩니다. 자세한 사항은 \hyperlink{tower:information}{정보} 챕터를 참고하세요.
	
	
	
	\subsection{전투}
	
	이야기꾼이 \storyref{black-scale}{검은 비늘 조각}을 발견하고 탑의 모든 장소의 조사를 끝냈다면 \storyref{role:disciple}{마법사의 제자}가 \storyref{role:big-wing}{큰 날개의 흑룡}이야기로 바뀝니다. 이야기꾼의 신체는 거대한 용으로 변하며, 적용 범위에 용이 나타났기 때문에 \storyref{magic:tower}{탑의 봉인}이 활성화됩니다. 마법사는 이야기꾼을 제압하기 위해 전투를 시작합니다.
	
	전투를 포기한다면 \storyref{role:big-wing}{큰 날개의 흑룡}은 효력을 잃고 다시 \storyref{role:disciple}{마법사의 제자}로 되돌아감을 알려주세요. 또한 \storyref{magic:tower}{탑의 봉인}또한 공격할 수 있는 대상임을 알려주세요. 마법사는 도발을 사용하여 \storyref{magic:tower}{탑의 봉인}을 공격하는 것을 방해할 수 있습니다.
	
	\storyref{magic:tower}{탑의 봉인}을 파괴한 경우, 마법사는 날아가려는 이야기꾼에게 대화를 시도합니다. 마법사는 가까이 와 달라고 부탁하고, 이를 거절한다면 이야기꾼은 용이 되어 날아가, 시스템의 부탁을 수행하지 못한 채 서사의 밖으로 되돌아갑니다. 하지만 마법사의 부탁을 수락한다면 아래 마법사를 전투에서 제압하지 않았을 때와 같이 취급합니다.
	
	마법사를 전투에서 제압하지 못했거나 제압하지 않은 경우, 마법사는 소지하고 있던 \storyref{teardrop}{미타아트의 눈물}을 사용해 이야기꾼의 기억을 다시 봉인합니다. \storyref{role:big-wing}{큰 날개의 흑룡} 이야기는 그 효력을 잃고 다시 \storyref{role:disciple}{마법사의 제자}로 돌아가고, 이야기꾼은 서사에서 퇴장하게 됩니다.
	
	마법사를 전투에서 제압했다면 마법사의 몸 속에서 \storyref{teardrop}{미타아트의 눈물}이 출현합니다. \storyref{teardrop}{미타아트의 눈물}을 습득한 이야기꾼은 서사에서 퇴장할 수 있습니다.
	
	
	
	\section{Outro Phase : 시스템 파트}
	
	\subsection{엔딩}
	
	퀘스트를 달성한다면 서사 밖으로 나와 시스템을 만날 수 있습니다. 퀘스트 달성 정도에 따라 다음과 같은 결과가 발생합니다.
	
	
	\subsubsection{메인 퀘스트의 일차 목표 달성}
	
	미타아트의 눈물을 획득했다면 시스템에게 건네고 보상을 받을 수 있습니다. 미타아트의 눈물을 시스템에게 주지 않는다면 다음과 같은 이야기를 획득합니다.
	
	\begin{story}[teardrop]{미타아트의 눈물}{[도구]}
		\flavour{강력한 힘을 가진 마석. 용을 사랑했던 마법사가 지니고 있었다. 어떤 힘은 상실에서 말미암기도 한다.}
		
		\entry{이 이야기를 사용한 판정을 시스템의 동의 하에 대성공으로 취급한다. 단, 이미 이루어진 판정을 번복하기 위해 사용할 수 없으며, 한 세션에서 한번만 사용할 수 있다.}
	\end{story}
	
	\subsubsection{메인 퀘스트의 이차 목표를 달성하였으나 일차 목표를 달성하지 못함\\또는, 미타아트의 눈물을 시스템에게 건넴}
	
	시스템은 이야기꾼의 공로를 치하하며 이야기꾼의 도움으로 마석을 원래 자리에 가져다 놓을 수 있게 되었다고 말합니다. 이야기꾼에게 다음 이야기 중 하나의 이야기를 보상으로 지급합니다.
	
	\begin{story}{마법사의 제자}{[생애]}
		\flavour{나는 제자인가, 잡일꾼인가……. 바쁘다는 핑계로 아무것도 가르쳐 주지 않은 마법사는 이번 논문의 집필이 끝나면 꼭 수업을 시작해 주겠다고 약속했다. ``이번에는 진짜죠?"}
		
		\entry{\statchange{+}{의지}}
	\end{story}
	
	\begin{story}{큰 날개의 흑룡}{[마법]}
		\entry{사격 판정에 성공 시 목표에게 3의 피해를 준다. 이 이야기를 사용하면 원래 모습으로 돌아갈 때 까지 장면에 등장하는 아군이 아닌 모든 존재로부터 우선적으로 공격받는다.}
	\end{story}
	
	\begin{story}{마법사의 만년필}{[도구]}
		\entry{마도구 전문 브랜드 `타미노펜'에서 출시된 한정 에디션 `깜부기불'. 스승이 사용하던 것. 몹시 오래된 것임에도 불구하고 관리가 잘 되어 있다. 스승은 아마 그의 스승에게서 받았으리라. 다음과 같은 문장이 각인되어 있다.}
		
		\flavour{빛은 길을 비춰줄 뿐, 내딛는 것은 당신의 몫이다.}
	\end{story}
\end{document}

	
	\chapter{정보}
	\hypertarget{tower:information}{}
		\documentclass{report}

\begin{document}

	\hypertarget{scenario-rule}{}
	\section{시나리오 전용 규칙}
		\documentclass{report}

\begin{document}

	\begin{story}{이름이 알려지지 않은 서사}{[세계]}
		\entry[\hline]{마법과 이종족이 존재하는 세계의 서사. 관련 이야기를 사용할 수 있습니다. 전자기기가 존재하지 않으며 관련 이야기를 사용할 수 없습니다.}
	\end{story}
		
	\begin{story}{숨바꼭질}{[마법사]}
		\entry[\hline]{탑 내 장소이동시 4df를 굴립니다. 0과 그 이하의 값이 나오면 입장한 장소에 마법사가 등장합니다. 한 장소에서 마법사가 등장했다면 바로 다음 장소 이동시에는 마법사 등장 여부를 판정하지 않습니다.}
	\end{story}

\end{document}
%
	
	\hypertarget{quest}{}
	\section{퀘스트}
		\documentclass{report}

\begin{document}
	
	\begin{spoiler}[quest:magicstone]{암룡의 마석 회수}{[퀘스트]}
		\entry{\textbf{일차 목표} : `암룡의 마석' 습득}
		
		\entry{\textbf{이차 목표} : `암룡의 마석'의 소재 확인}
		
		\entry{\textbf{지급 이야기} : \storyref{role:disciple}{마법사의 제자}}
		
		\entry{\textbf{성공 보상} : 서사의 기여도와 역할에 관련된 이야기 한 개}
		
		\entry[\hline]{\textbf{실패 보상} : -}
	\end{spoiler}
	
	\begin{spoiler}[quset:familiar]{놀아주세요!}{[서브퀘스트]}
		\flavour{패밀리어와 즐거운 시간을 보내면 달성되는 퀘스트입니다. 별다른 판정 없이 RP만으로 수행 가능하며, 어떻게 노는지에 대해 특별한 아이디어가 없다면 패밀리어가 공 등의 장난감을 물고 와서 놀아달라고 조르는 것으로 시작하는 것이 무난합니다.}
		
		\entry{\textbf{목표} : 패밀리어와 놀아주기}
		
		\entry{\textbf{성공 보상} : 패밀리어가 기뻐합니다.}
		
		\entry[\hline]{\textbf{실패 보상} : 패밀리어가 속상해합니다.}
	\end{spoiler}
	
	\begin{spoiler}[quest:book]{책 정리}{[서브퀘스트]}
		\entry{4df를 세 번 굴립니다. 서가 하나마다 정리 성공 여부를 판정하는 것으로 간주합니다. 2번 이상 성공하면 퀘스트에 성공합니다.}
		
		\entry{\textbf{목표} : 서재의 책 정리}
		
		\entry{\textbf{성공 보상} : 
			\begin{spoiler}[state:proudness]{뿌듯함}{[상태]}
				\flavour{만사를 긍정적으로 볼 수 있을 것 같다. 오래 갈 기분은 아니지만, 기분이 좋다는 건 소중한 것이다.}
								
				\entry[\hline]{판정 실패시 재굴림. 이 이야기는 한번 사용하면 소멸한다.}
			\end{spoiler}
		}
		
		\entry[\hline]{\textbf{실패 보상} : 
			\begin{spoiler}[state:stressed]{스트레스}{[상태]}
				\flavour{마치 고통 받기 위해 만들어진 인생 같아.}
				\entry[\hline]{2장면동안 지속된다.}
			\end{spoiler}
		}
	\end{spoiler}
	
\end{document}
%
	
	\hypertarget{npc}{}
	\section{NPC}
		\documentclass{report}

\begin{document}
	\subsection*{마법사}
		체력 10/10
		
		사격, 기만, 의지 1\footnote{이야기들의 스탯 증가가 반영된 수치입니다.}
		
		숲 속의 외딴 탑에 기거하는 마법사.

		이 탑에 봉인된 드래곤 [큰 날개의 흑룡]이 세상에 나가지 못하도록 감시하고 있습니다. 이것은 또한 나날이 강해지고 있는 인류로부터 [큰 날개의 흑룡]을 보호하는 것이기도 합니다. 마법사는 동시에 그를 자유롭게 해 주지 못하는 것에 대한 죄책감을 가지고 있습니다. 과거에 [큰 날개의 흑룡]의 제자 또는 피보호자였습니다. 세션에서는 이러한 관계가 뒤바뀐 셈입니다.
		
		전투가 막바지에 이르면 메테오를 사용합니다.
	
		\begin{spoiler}[npc-wizard:wizard]{마법사}{[권능][직업]}
			\flavour{현실과 환상의 경계를 허무는 이능을 부리는 자. PC의 스승이자 이 탑의 주인.}
			
			\entry{마나 50을 가진다.}
			
			\entry[\hline]{\statchange{+}{사격}}
		\end{spoiler}
		
		\begin{spoiler}[npc-wizard:warder]{마탑의 간수}{[생애][권능]}
			\flavour{가장 강력하고 위험한 환상이 두 발과 날개가 묶인 채 여기에 있다. 내가 여기에 있는 한 당신은 저 밖의 누구도 해칠 수 없으며, 저 밖의 누구도 당신을 해칠 수 없다. }
			
			\entry[\hline]{\statchange{+}{기만}}
		\end{spoiler}
		
		\begin{spoiler}[npc-wizard:taunt]{도발}{[요령]}
			\entry[\hline]{대상을 정해 의지 판정을 하게 한다. 만약 자신의 도발이 더 높다면, 다음 턴에는 해당 대상은 반드시 자신을 공격해야 한다.}
		\end{spoiler}
		
		\begin{spoiler}[npc-wizard:armor]{어둠의 가호}{[방어구]}
			\flavour{어둠을 묵도할지언정, 그 그늘 아래 보호받았던 나는 두려워하지 않는다.}
			
			\entry{모든 종류의 피해를 1 경감한다.}
			
			\entry[\hline]{\statchange{+}{의지}}
		\end{spoiler}
		
		\begin{spoiler}[npc-wizard:magic-bullet]{마력탄}{[마법]}
			\entry[\hline]{한 차례 영창하여 마나1을 소모하고 3의 피해를 준다.}
		\end{spoiler}
		
		\begin{spoiler}[npc-wizard:consecutive-magic-bullet]{마력탄 연사}{[마법]}
			\entry[\hline]{마나 3을 소모하여 여러 개의 마력탄을 동시에 발사한다. 사격으로 판정하여 0 이상의 수치가 나오면 나온 수치+2 만큼의 피해를 준다.}
		\end{spoiler}
		
		\begin{spoiler}[npc-wizard:meteo]{메테오 스트라이크}{[마법]}
			\entry[\hline]{응축된 거대한 마력 덩어리를 떨어트린다. 두 턴 영창하여 마나 5를 소모하고 10의 피해를 준다. 한 전투에서 한 번만 사용 가능하다.}
		\end{spoiler}
	
	\subsection*{패밀리어}
		체력 10/10
		마법사는 제자를 보호하기 위해 패밀리어를 이야기꾼의 곁에 붙여놓았습니다.
		
		패밀리어는 세션이 진행되는 동안 이야기꾼을 따라다니며 애교를 부리거나, 놀아 주거나, 때로는 위협으로부터 이야기꾼을 대신해 싸우기도 합니다. GM은 패밀리어를 통해 PC에게 힌트를 건네거나 장난을 치는 등의 RP를 할 수 있습니다. 패밀리어의 모습은 어떤 동물도 좋지만 고양이, 햄스터, 도마뱀, 참새 등 작은 동물을 권합니다. 플레이어와 상의하여 선호하는 동물로 정해도 좋습니다.
		
		이야기꾼과 마법사의 전투가 발생하면 마법사에게 돌아가 \storyref{npc-wizard:armor}{어둠의 가호}로 변합니다.
		
		\begin{spoiler}[npc-familiar:familiar]{패밀리어}{[마법]}
			\flavour[\hline]{동물의 모습을 한 작은 그림자. 마법사가 부리는 마법생물이다.}
		\end{spoiler}
		
		\begin{spoiler}[npc-familiar:truth]{진실의 수호자}{[임무]}
			\flavour[\hline]{당신의 술자는 사랑하는 이를 지킬 것을 명령했다.}
		\end{spoiler}
		
		\begin{spoiler}[npc-familiar:darkness]{어둠}{[성질]}
			\entry[\hline]{모든 종류의 피해를 1 경감한다.}
		\end{spoiler}
	
	\subsection*{슬라임}
		체력 5/5
		
		창고의 잡동사니 틈에서 튀어나온 작은 몬스터. 몸이 산성으로 되어 있어 접촉 시 가벼운 화상을 입힙니다. 슬라임에게 상처를 입은 채로 마법사와 마주친다면 마법사가 상처를 치료해줍니다. 박쥐나 뱀 등 작고 위협도가 크지 않은 다른 동물이나 몬스터로 개변해도 좋습니다.
		
		\begin{spoiler}[npc-silme:jump]{튀어오르기}{[행위]}
			\flavour[\hline]{아무 일도 일어나지 않았다.}
		\end{spoiler}
		
		\begin{spoiler}[npc-silme:acid]{산성}{[성질]}
			\entry[\hline]{접촉시 가벼운 화상을 입는다. 위협적이진 않으나 따갑다.}
		\end{spoiler}
	
\end{document}
%
	
	\hypertarget{tower-story}{}
	\section{이야기}
		\documentclass{report}

\begin{document}
	
	\begin{story}[role:disciple]{마법사의 제자}{[역할][직업]}
		\flavour{나는 제자인가, 잡일꾼인가……. 바쁘다는 핑계로 아무것도 가르쳐 주지 않은 마법사는 이번 논문의 집필이 끝나면 꼭 수업을 시작해 주겠다고 약속했다. “이번에는 진짜죠?”}
		\entry[\hline]{\statchange{+}{의지}}
	\end{story}
	
	서사 속에 진입했을 때 시스템으로부터 주어지는 이야기입니다. 이야기꾼은 본래부터 존재하던 ‘마법사의 제자’자리에 대신 들어갔거나 그에게 빙의했다고 보아도 무방합니다. 아직 마법을 제대로 배우지 못한 수련생이므로 고등한 마법과 관련된 이야기는 사용하기 어렵습니다.

	플레이어가 만약 대학원생이거나 악덕 교수에게 고통 받은 경험이 있다면 이런 설정에 반감을 가지고 마법사를 지나치게 적대시할 수 있습니다. 이 경우 마법사를 불쌍하게 연출하여 동정심을 불러일으키거나 아예 다른 관계성을 제시하는 것을 추천합니다.
	
	\begin{story}[item:fountain-pen]{마법사의 만년필}{[도구]}
		\entry{마도구 전문 브랜드 `타미노펜'에서 출시된 한정 에디션 `깜뿌기불'. 몹시 오래 된 것임에도 불구하고 관리가 잘 되어 있다. 다음과 같은 문장이 각인되어 있다.}
		
		\flavour[\hline]{빛은 길을 비춰줄 뿐, 내딛는 것은 당신의 몫이다.}
	\end{story}
	
	만년필을 자세히 조사하면 \storyref{item:memo-in-pen}{만년필 속 쪽지}를 발견할 수 있습니다.
	
	\begin{story}[item:memo-in-pen]{만년필 속 쪽지}{[서류]}
		\flavour[\hline]{`당신이 잊었어도 괜찮아요. 이제 당신의 곁을 떠나지 않을게요.'}
	\end{story}
	
	\begin{story}[item:teardrop]{미타이트의 눈물}{[도구]}
		\flavour{메인 퀘스트의 목적인 마석입니다.}
		
		\entry[\hline]{강력한 힘을 가진 마석. 인류가 기억하는 가장 오래된 용 미타아트의 이름과 함께 전해져 내려온다. 짙은 그림자와 어둠으로 이루어진 암룡(暗龍)은 자손들과 자녀들을 거느리고 인류와 전쟁을 벌였다고 한다.}
	\end{story}
	
	\begin{story}[magic:tower]{탑의 봉인}{[마법]}
		\flavour{이야기꾼이 공격하여 파괴할 수 있는 대상입니다. 이 사실을 전투가 시작했을 때 알려주세요. 다섯 번 공격당하면 파괴됩니다.}
		
		\entry[\hline]{드래곤을 억압하는 마법. 영역 안에 있는 용은 비행이 불가능하며, 모든 판정에 -1을 받고, 매 차례마다 체력을 1 잃는다.}
	\end{story}
	
	\begin{story}[role:big-wing]{큰 날개의 흑룡}{[역할][종족][권능]}
		\flavour{절멸한 용족의 마지막 후예 중 하나. 예로부터 드래곤은 가장 강력하고 위험한 환상 중 하나였고 재앙으로 불렸다. 인류는 당신을 위협으로 여겨 제압하고 이 탑에 봉인했다. 당신은 거짓된 기억을 주입당해 자신이 사람이라 믿으며 이 곳에 갇혀있었다.}
		
		\entry[\hline]{사격 판정에 성공 시 목표에게 3의 피해를 준다.}
	\end{story}
	
	특정 조건을 만족할 경우 \storyref{role:disciple}{마법사의 제자}는 이 이야기로 변합니다. 이 이야기를 얻었을 때 의지판정(+2 혹은 그 이상일 경우 성공)을 하여 실패하면 3차례 동안 지속되는 상태인 \storyref{state:angry}{격노}를 가집니다. 의지 판정에 대성공했을 경우 \storyref{memoey:even-different}{서로 다르더라도}를 얻습니다.
	
	\begin{story}[state:angry]{격노}{[상태]}
		\entry[\hline]{외부의 자극으로 이성을 잃은 상태. 공격할 수 있는 대상이 있다면 반드시 공격해야 한다.}
	\end{story}
	
	\begin{story}[memoey:even-different]{서로 다르더라도}{[기억]}
		\entry[\hline]{검은 용과 작은 인간은 친구였습니다. 둘은 서로 달랐지만 사랑했습니다. 크고 강한 용은 작은 인간을 지켜주겠다고 약속했습니다.}
	\end{story}

	
\end{document}
%
	
	\hypertarget{tower-of-wizard}{}
	\section{마법사의 탑}
		\documentclass{report}

\begin{document}

	마법사의 연구소이자 거주지. 왕국에서 먼 곳에 떨어진 숲 속에 위치하고 있습니다.
	
	\begin{story}{숨바꼭질}{[이동]}
		\entry{탑 내에서 장소를 이동할 때, 4df를 굴린다. 0 이하의 값이 나오면 입장한 장소에 마법사가 등장한다.}
		
		\entry[\hline]{한 장소에서 마법사가 등장했다면 바로 다음 장소 이동시에는 마법사 등장 여부를 판정하지 않는다.}
	\end{story}
	
	구조는 다음과 같습니다. 높은 곳부터 서술합니다.

	\subsection*{탑의 꼭대기}
		탑의 옥상. 탑 주변의 경치가 한눈에 들어옵니다. 자세히 조사한다면 탑 주변에 둘러쳐진 결계를 발견합니다.
	
	\subsection*{서재}
		마법사의 서재이자 연구실. 벽 가득히 책이 꽂혀 있습니다. 책이 엉망진창으로 섞여 있다는 것을 발견할 수 있습니다. 정리하는 것은 분명 이야기꾼의 몫이겠죠. 자세히 조사하면 어린아이 글씨로 낙서가 된 책과, 페이지 일부가 통째로 찢겨나간 책을 발견할 수 있습니다.
		
		세션을 시작하는 장소로, 마법사로부터 \storyref{item:fountain-pen}{마법사의 만년필}을 받습니다.
		
		서재 외의 공간에서 마법사를 만난다면 `왜 여기에 있느냐. 일은 다 했느냐. 그럴 리가 없지.' 라는 등의 말을 하며 이야기꾼이 농땡이를 치고 있다는 사실을 지적합니다. 그러나 크게 탓하거나 돌아가 일하라고 지시하지는 않습니다.
	
	\subsection*{창고}
		다양한 물건들이 분류되지 않은 채로 여기저기 쌓여 있는 공간. 물건들 틈에서 슬라임을 발견할 수 있습니다.
	
	\subsection*{마법사의 방}
		마법사의 침실. 자세히 조사한다면 마법사로 보이는 인물과 누군가가 함께 찍힌 사진을 발견할 수 있습니다. 이 사진은 어린 시절의 마법사와 그 스승-검은 날개의 흑룡-이 함께 찍은 사진으로, 이야기꾼은 자세히 조사하더라도 이 이야기를 알 수 없습니다. 다만 이유를 알 수 없는 친숙함이나 묘한 기시감 등을 느낄 수 있을 뿐입니다.
		
		마법사를 동반하지 않고 방문했을 경우, 문이 마법으로 잠겨 있어 열 수 없습니다. 문을 억지로 열기 위해서는 5 이상의 충격을 주어야 합니다. 마법사는 문이 열린 것을 눈치 채고 1턴 후에 달려오므로, 빠르게 조사를 마치거나 자리를 이탈해야 합니다. 만약 행운 판정에 성공한다면 마법사는 3턴 후에 도착합니다.
		
		마법사에게 현장을 들켰을 경우, 문이 부서진 이유에 대해 마법사에게 납득시키지 못한다면 서사에서 추방당합니다.
	
	\subsection*{정원}
		탑을 중심으로 원형으로 자리한 공터. 걸터앉을 수 있는 잡동사니가 있습니다.
		
		정원을 통해 이야기꾼이 밖으로 나가려고 시도할 경우, 먼저 `탑 밖으로 나가면 마법사가 걱정할 것이다', `마석은 탑 밖이 아니라 안에 숨겨져 있다' 등으로 이야기꾼을 설득하는 쪽을 권합니다. 그럼에도 불구하고 이야기꾼이 나가고자 한다면 뭔가에 가로막힌 듯 숲으로 들어갈 수 없습니다.
	
	\subsection*{정원}
		쇠사슬과 커다란 수갑 등 뭔가를 가두어 놓았던 흔적이 있습니다. 자세히 조사할 경우 \hypertarget{black-scale}{검은 비늘 조각}을 발견할 수 있습니다.
	
\end{document}
%
	
\end{document}
%
		
	\chapter*{스포일러 방지 및 메모용 빈 페이지 입니다.}
\end{document}

	
	\part{살아 있지 못한 마을(The Unliving Village)}
		\label{endof_Tower}
		\documentclass{report}

\begin{document}
	\chapter*{본 시나리오의 내용을 보고 싶지 않으신 분들께서는 \pageref{endof_Unliving}쪽으로 넘어가 주시기 바랍니다.}
	
	\chapter{시놉시스와 시나리오 기본 정보}
		\documentclass{report}

\begin{document}
	\textbf{시나리오 이름}: 살아 있지 못한 마을(The Unliving)
	
	\textbf{시나리오 작가}: None(\href{https://www.twitter.com/n0n3x1573n7_WS}{@n0n3x1573n7\_WS})
	
	\textbf{권장 인원}: 1인, 2\textasciitilde4인 대립
	
	\textbf{트리거 워닝}: (추가 예정)
	
	\subsubsection*{시놉시스}
	
	중세 시대의 작은 마을. 어느 주말, 예배를 보려고 모여있던 사람들이 어째서인지 좀비로 변하고 있었습니다.
	
	마을에 있는 사람 중 좀비가 되지 않은 사람은 단 네 명.
	
	예배당에서 노래를 부르고 있던 성가단원.
	
	예배당에서 예배를 집전하고 있던 사제.
	
	은퇴해서 마을에서 쉬고 있다가 예배를 빼먹은 성기사.
	
	그리고 이 소식을 듣고 급하게 이 마을로 파견된 이단심판관.
	
	네 명의 생존자는 이 일을 조사해 해결하고자 합니다. 이 마을에 내린 저주를 이들은 해결할 수 있을까요?
\end{document}%
	
	\chapter{등장인물들의 이야기(이야기꾼 공개용)}
		\documentclass{report}

\begin{document}
	1인인 경우에는 성가단원과 성기사 중 한 명을 선택합니다.
	
	2\textasciitilde3인인 경우에는 사제를 제외한 등장인물들 중 한 명씩을 선택합니다.
	
	\section*{성가단원}
	
	\begin{spoiler}{성가대}{[역할]}
		\entry[\hline]{노래를 부르거나 악기를 연주하여 음악을 통해 마법을 사용할 수 있다. 다음 노래를 사용할 수 있다:
		\begin{spoiler}{성가}{[노래]}
			\entry[\hline]{한 턴에 이 노래를 부르기로 선택한다면 이동을 제외한 다른 행동을 할 수 없다. 턴이 종료될 때, 같은 구역에 있는 모든 생명체의 체력을 1 회복하고, 언데드에게 [신성] 피해를 1 준다.}
		\end{spoiler}
		}
	\end{spoiler}
	
	\section*{성기사}
	
	\begin{spoiler}{신성한 일격}{[역할]}
		\entry[\hline]{사거리에 상관 없이 대상을 정한다. 대상에게 [신성] 피해를 2 주고, [기절] 상태에 빠트린다. [기절] 상태의 상대는 다음 턴 모든 행동이 불가능하다.}
	\end{spoiler}
	
	\section*{사제}
	
	\begin{spoiler}{기도}{[역할]}
		\entry[\hline]{사거리에 상관 없이 대상을 정한다. 대상의 다음 턴이 시작될 때, 대상의 체력을 2 회복시키거나, [신성] 피해를 2 준다.}
	\end{spoiler}
	
	\section*{이단심판관}
	
	\begin{spoiler}{심판}{[역할]}
		\entry[\hline]{씬이 시작할 때, [심판]의 대상을 하나 정한다. 해당 대상에게 주는 모든 피해가 1 증가한다.}
	\end{spoiler}
\end{document}%
	
	\chapter{서사의 이야기}
		\documentclass{report}

\begin{document}
	\begin{story}[dark-age]{어둠의 시대}{[중세]}
		\flavour{마녀 사냥이 일어나는 어둠의 시대.}
		
		\entry{현대적인 기술력을 사용하거나, 신성력 외의 마법을 사용하기 위해서는 반드시 [추방] 판정을 거쳐야 한다.}
		
		\entry{[추방] 판정에 실패하면, 개연성에 피해를 받는 대신 해당 이야기를 이 서사의 결말을 맞기 전까지 봉인하는 것으로 대체할 수 있다.}
	\end{story}
	
	\begin{story}{몰입}{[이야기의 의지]}
		\entry{등장인물들이 알고 있는 것들 중 \textbf{공개 조건}이 존재하는 지식 또는 이야기들은 등장인물의 자존심이나 비밀과 매우 크게 연관이 있기 때문에 서사 상으로 스스로 밝힐 이유가 없는 정보들이기 때문에, 스스로 밝힐 수 없다. 이를 스스로 밝힌다면, [침범] 판정의 난이도가 영구적으로 1 증가하며, 즉시 [침범] 판정을 해, 실패한다면 해당 씬 동안 서사 속에서 얻은 것이 아닌 무작위 이야기가 하나 봉인된다.}
	\end{story}
	
	\pagebreak
	\section*{선택 이야기}
	
	이 시나리오에는 조사해야할 것 같은 대상은 많아보이지만 실제 진상에 접근하는 데에 사용할 수 있는 정보는 적습니다. 그렇기 때문에, 아래의 \storyref{system:tutorial}{튜토리얼} 이야기의 사용을 권장합니다.
	
	\begin{story}[system:tutorial]{튜토리얼}{[도움: 시스템]\ifDLC[inSANe]\fi}
		\entry{서사 속에서 만나는 중요해보이는 인물, 사물, 상황들을 묘사하는 이야기를 알 수 있다. 이런 이야기들을 직접 조사하여 해당 이야기에 가려서 숨겨져있던 사실을 밝혀내고, 상황을 진행시킬 수 있다. 이런 조사를 하기 위해서는 상황에 맞는 어떠한 방법을 사용해도 무방하다. 이 세계의 서사를 풀어나가기 위해서는, 이 이야기들만을 조사해도 충분하다.}
	\end{story}
	
	보다 더 어려운 진행을 위해서, 다음 규칙을 적용시킬 수 있습니다.
	
	\begin{story}{박진감}{[이야기의 의지]}
		\entry{이 이야기는 빠르게 진행된다. 씬이 끝날때, 잃은 정신력과 체력의 반(소숫점 아래 버림)만을 회복한다\footnote{또는, 아예 회복하지 않도록 할 수도 있습니다.}.}
	\end{story}
\end{document}%
	
	\chapter*{스포일러 방지 및 메모용 빈 페이지 입니다. 본 시나리오를 플레이하실 분께서는 \pageref{endof_Unliving}쪽으로 넘어가 주시기 바랍니다.}
	
	\parttoc
	
	\chapter{시나리오의 흐름}
		\documentclass{report}

\begin{document}
	\section{태초의 이야기(선택)}
		태초의 이야기에서 시스템은 네 이야기꾼을 불러, 서사의 기본 정보와 공개 이야기에 대해 이야기해줍니다. 여기에서 네 가지 역할, 즉 \hyperlink{cursed-bard}{성가단원}, \hyperlink{cowardly-priest}{사제}, \hyperlink{corrupt-paladin}{타락한}, \hyperlink{hurt-rogue}{이단심판관}을 부여합니다. 이 때, 네 가지 역할을 네 명이 시스템과 따로 만나서 나눠줘도 되고, 네명이 같이 시스템과 모여서 각자 고르도록 해도 상관 없습니다.
		
		이 이야기에 왜 들어가야 하는지를 납득시키기 위해서라면, [타락한 자]가 들어가서 이야기를 휘젓고 있으며 이 작용을 막기 위함이라고 하시면 됩니다\footnote{실제로 사제가 소환하고자 한 악마가 바로 이 [타락한 자]이기 때문이죠.}.
		
		개인적으로는 시스템이 네 개의 역할을 지정하는 것을 추천드립니다. \hyperlink{cowardly-priest}{사제}를 받은 이에게 흑막이 본인이라는 사실을 밝혀야 하기 때문이기도 하고, 동의를 구해야 하기 때문입니다. 반드시 동의하지 않아도 된다는 점을 강조해주세요.
		
		동의하지 않은 경우에는 "역할은 역할일 뿐"이라는 점을 명시하고, \hyperlink{alternative:no-criminal}{흑막 거부시의 대체 세계선}을 따라서 진행하시면 됩니다.
	
	\section{마을의 구조}
		마을은 마을 외곽, 마을 내부, 교회의 총 세 겹으로 구성되어 있습니다. 이야기꾼들은 마을로 들어가는 길에 서 있습니다. 사제는 마을에서 나와 이단심판관을 맞이하러 나와있었고, 그 과정에서 좀비로 변하지 않은 성가단원 꼬마와 전직 성기사를 데리고 나왔습니다.
		
		현재 시각은 해가 뉘엿뉘엿 지고 있는 저녁 6시입니다. 이야기꾼들에게 시간이 흐르고 있다는 사실을 반드시 상기시켜주세요.
		
			\subsection{마을 외곽}
				\hypertarget{search:newspaper-stand}{}
				\subsubsection*{신문 가판대}
					\begin{spoiler}[search:news]{신문}{[튜토리얼:조사대상]}
						\entry{신문을 읽는 데에 30분을 사용할 수 있습니다.}
						
						\entry[\hline\hline]{인근 마을들에 도둑이 들었다는 소식이 신문 1면에 대서특필 되어 있습니다.}
						
						\entry[\hline]{\textbf{누군가가 이단심판관(외부인)을 의심할 때 공개}: \hyperlink{search:rogue-bag}{이단심판관의 가방}을 이제 조사할 수 있습니다.}
					\end{spoiler}
				
				\hypertarget{search:rogue-bag}{}
				\subsubsection*{이단심판관의 가방}
					이 가방은 이단심판관이 항상 들고 다니는 가방입니다. 누군가가 이단심판관을 의심하기 시작할 때에야 조사할 수 있도록 해 주세요.
					
					\begin{spoiler}[search:villager-identifications]{신분 증명서 더미}{[튜토리얼:조사대상]}
						\entry[\hline]{이단 심판관 증명서와 함께, 사제와 성기사의 신분 증명서 요약본이 들어 있습니다.}
					\end{spoiler}
					
					\begin{spoiler}[search:files]{서류철}{[튜토리얼:조사대상]}
						\entry{서류더미를 읽는 데에 30분을 사용할 수 있습니다.}
						
						\entry{[정화의 노래]에 관련된 정보가 정리된 서류철입니다. 이 문장이 특히 눈에 띄는군요.}
						
						\flavour[\hline]{순수한 영혼만이 이 노래를 통해 세계를 정화할 수 있다고 전해진다.}
					\end{spoiler}
					
					\begin{spoiler}[search:strange-bag]{수상한 주머니}{[튜토리얼:조사대상]}
						\entry[\hline\hline]{주머니를 열어보면 도둑들이나 들고 다닐만한 단도와 락픽이 잔뜩 들어있습니다.}
						
						\entry[\hline]{\textbf{이단심판관에게만 전달}: 이단심판관의 칭호 \storyref{rogue:hurt}{부상당한}이 제거되고, 언제든지 본인이 도둑이라는 진실을 밝히며 \storyref{rogue:dagger}{단도 투척}과 \storyref{rogue:lockpick}{자물쇠 따기}를 얻습니다.}
					\end{spoiler}
				
			\subsection{마을 내부}
				마을 내부로 들어가기 위해서는 좀비를 피해다니며 30분의 시간이 소요됩니다. 마을 내부로 들어온 경우, 마을 외곽으로 다시 돌아나가기는 힘들 것 같다는 언급을 반드시 해주세요.
				
				이단심판관을 이야기꾼들이 의심하기 시작했다면, \storyref{search:bag}{이단심판관의 가방}을 조사할 수 있다는 점을 기억하세요.
				
				\hypertarget{search:zombie}{}
				\subsubsection*{좀비}
					모든 좀비의 체력은 10으로 취급합니다.
					
					\begin{spoiler}[search:undead]{좀비}{[튜토리얼:조사대상]}
						\entry{좀비를 사로잡아 분석하는데에 30분을 소모할 수 있습니다.}
						
						\entry{
							\begin{spoiler}{저주의 산물}{[저주]}
								\entry[\hline]{[신성] 피해가 아닌 피해로 피해를 받을 수 없으며, 이 서사 밖의 이야기로 인한 [신성] 피해로는 체력을 1 이하로 깎을 수 없다.}
							\end{spoiler}
						}
						
						\entry{
							\begin{spoiler}{마을 사람}{[과거]}
								\entry[\hline]{마을 주민이었던 사제와 성가단원, 성기사는 이들의 체력을 1 이하로 깎을 수 없다. 단, 체력이 1 이하인 경우 [신성] 피해를 주면 한턴간 [기절] 상태가 되어, 이동을 포함한 모든 행동을 할 수 없다.}
							\end{spoiler}
						}
						
						\entry[\hline]{
							\begin{spoiler}[zombie:anger-aroused]{점점 격렬해지는 분노}{[저주]}
								\entry[\hline\hline]{시간이 지날수록 점점 공격성이 강해지고 있다. 원인을 제거하지 않으면 이야기꾼 본인들이 위험해질 수 있다.}
								
								\entry[\hline]{\textbf{좀비의 공격성}: 이야기꾼에게는 아래 내용을 알리지 않는다.
									
									\begin{tabularx}{\linewidth}{c|X}
										\textbf{시간} & \makecell{\centering\textbf{공격성}}\\\hline\hline
										6시\textasciitilde8시 & 선제공격을 해도 공격성이 없음 \\\hline
										8시\textasciitilde10시 & 선제공격을 하면 반격함 \\\hline
										10시\textasciitilde 자정 & 보이면 공격성을 보임 \\\hline
										자정 이후 & 이야기꾼들을 적극적으로 찾아나섬
									\end{tabularx}
								}
							\end{spoiler}
						}
					\end{spoiler}
				
				\hypertarget{search:choir-leader}{}
				\subsubsection*{성가단장의 집}
					\begin{spoiler}[search:diary]{책장}{[튜토리얼:조사대상]}
						\entry[\hline\hline]{이 책장에 있는 사용할 수 있을 만한 정보를 한 가지 얻는 데에 10분이 소요됩니다. 얻겠다고 선언한다면, 아래 정보 중 무작위 정보를 전달해주세요. 1df를 굴려, +/0/-가 나올때마다 서로 다른 정보를 주고, 중복되는 정보를 줘야 할 때 마다 이미 아는 정보라는 사실을 전달해주는 것을 추천합니다.}
						
						\entry[\hline\hline]{\textbf{일기장}: 성가단원 꼬맹이는 입양아입니다. 아기 때에 풀숲 안에 숨겨져 있던 아기를 찾아 데리고 와서 키우고 있죠.}
						
						\entry[\hline\hline]{\textbf{음악 연습 기록지}: 이 마을에 과거에 살던 천재적인 음악가가 작곡한 전설의 노래가 있다고 성기사가 당신에게 얘기해주었습니다. 최근, 성기사가 이 노래로 추정되는 곡을 어떻게인지 찾아왔군요.}
						
						\entry[\hline]{\textbf{오래된 책}: 순수한 영혼에 대한 과거의 연구 자료가 있습니다. 순수한 영혼의 피끼리는 서로 섞인다는 말이 적혀 있군요.}
					\end{spoiler}
				
				\hypertarget{search:paladin}{}
				\subsubsection*{성기사의 집}
					\storyref{search:shovel}{흙이 묻은 삽}은 성기사의 집을 나서려고 할 때 현관문 뒤에 서 있는 것을 발견하도록 하는 것을 권장합니다.
					
					\begin{spoiler}[search:paladin-shield]{낡은 방패}{[튜토리얼:조사대상]}
						\entry[\hline]{손잡이 부분에 다음이 숨겨져 있습니다:
							\begin{spoiler}{흑마법사의 피}{[타락]}
								\entry[\hline]{인접한 구역에 이 피를 흩뿌릴 수 있다. 다음 두 턴 동안, 좀비는 반드시 해당 칸을 향해서 이동하며, 공격하지 않는다.}
							\end{spoiler}
						}
					\end{spoiler}
					
					\begin{spoiler}[search:paladin-cross]{십자가}{[튜토리얼:조사대상]}
						\entry[\hline]{내부에 다음이 숨겨져 있습니다:
							\begin{spoiler}{순수한 피}{[신성]}
								\flavour[\hline]{순수한 인간의 피입니다. 과거, 성기사는 이 사람을 지키는 데에 실패했습니다.}
							\end{spoiler}
						}
					\end{spoiler}
					
					\begin{spoiler}[search:shovel]{흙이 묻은 삽}{[튜토리얼:조사대상]}
						\flavour[\hline]{젖어있는 흙이 묻은 삽입니다.}
					\end{spoiler}
				
			\subsection{교회}
				교회 주변 구역으로 들어가기 위해서는 좀비를 피해다니며 30분의 시간이 소요됩니다. 이곳으로 들어온 경우, 마을쪽으로 다시 돌아나가기는 힘들 것 같다는 언급을 반드시 해주세요.
				
				이야기꾼들은 \storyref{search:zombie}{좀비}와 \storyref{search:bag}{이단심판관의 가방}를 조사하지 않았다면 조사할 수 있습니다. 이들을 조사할 수 있다는 점을 상기시켜주세요.
				
				\hypertarget{search:choir-practice}{}
				\subsubsection*{성가단의 연습장}
					예배당 바로 옆에 붙어있는, 방음이 잘 된 독립된 공간입니다.
					
					\begin{spoiler}[search:sheet-music]{악보}{[튜토리얼:조사대상]}
						\entry{1시간을 소모하여 악보를 정리할 수 있습니다. 또는, [지식:음악] 판정을 통해 2 이상이 값이 나오는 한 사람당 10분을 절약할 수 있습니다. [지식:음악]이 0인 사람은 이 판정을 할 수 없습니다. 정리가 끝나면, 아래 음악들의 악보를 획득할 수 있습니다.}
						
						\entry{
							\begin{spoiler}{속도의 노래}{[노래]}
								\entry[\hline]{이 노래를 부르는 이와 같은 구역에서 출발한 이들은 1회 더 이동할 수 있다.}
							\end{spoiler}
						}
						
						\entry{
							\begin{spoiler}{변화의 노래}{[노래]}
								\entry[\hline]{이 노래를 부르는 이와 같은 구역에 있는 동안, 개연성 1을 깎고 체력과 정신력을 1씩 회복할 수 있다.}
							\end{spoiler}
						}
						
						\entry{
							\begin{spoiler}{조화의 노래}{[노래]}
								\entry[\hline]{이 노래를 부르는 이와 같은 구역에 있는 동안, 한 턴에 한 번 공격을 회피할 수 있다.}
							\end{spoiler}
						}
						
						\entry[\hline\hline]{성가로 한번도 들어보지 못한 곡이 섞여있습니다. 성가단장의 필체로 [정화의 노래]라고 적혀 있는 곡입니다.}
						
						\entry[\hline]{성기사는 자신이 이 곡을 발견하기는 했지만, 성가단장이 악보를 현대적인 악보로 옮겼기 때문에 바로 알아보지는 못합니다.}
					\end{spoiler}
					
					\begin{spoiler}[search:organ]{소형 오르간}{[튜토리얼:조사대상]}
						\entry[\hline]{성가책이 펼쳐져 있는 오르간입니다. 한가지 음악을 연주하기 위해서는 15분이 소모됩니다.}
					\end{spoiler}
					
					\begin{spoiler}[search:singer-identifications]{신분 증명서 더미}{[튜토리얼:조사대상]}
						\entry{성가단장과 성가단원의 신분 증명서입니다. 정독하는데에 30분을 쓸 수 있습니다.}
						
						\entry[\hline]{정독하면, 성가단장이 결혼한 적이 없으며, 성가단원이 입양되었다는 사실을 알 수 있습니다.}
					\end{spoiler}
				
				\hypertarget{search:graveyard}{}
				\subsubsection*{공동묘지}
					\begin{spoiler}[search:musician-grave]{파헤쳐진 묘지}{[튜토리얼:조사대상]}
						\entry{모든 무덤은 얌전합니다. 30분을 소모해 무덤들을 관찰할 수 있습니다.}
							
						\entry[\hline]{한 무덤에 파헤쳐진 자국이 있습니다. 이 무덤의 비석에 ``내 음악이 곧 모두를 살릴 것이다"라는 엄청난 포부가 적혀있습니다.}
					\end{spoiler}
				
				\hypertarget{search:cleric-bedroom}{}
				\subsubsection*{사제의 침소}
					
					\begin{spoiler}[search:table]{탁자}{[튜토리얼:조사대상]}
						\entry{
							\begin{spoiler}[search:bible]{성경}{[튜토리얼:조사대상]}
								\entry{1시간을 소모해 성경을 정독할 수 있습니다. [지식:종교] 또는 [지식:서적] 판정을 해 4 이상의 값이 나오는 한 사람당 20분을 절약할 수 있습니다. 사제는 이 판정을 도울 수 없습니다.}
								
								\entry[\hline]{표지는 성경처럼 보이지만, 사실 악마를 숭배하는 책입니다!}
							\end{spoiler}
						}
						
						\entry[\hline]{
							\begin{spoiler}[search:bell]{종}{[튜토리얼:조사대상]}
								\entry[\hline]{종을 뒤집어 안을 보면, 종을 타종하는 금속이 열쇠라는 사실을 알 수 있습니다.}
							\end{spoiler}
						}
					\end{spoiler}
					
					\hypertarget{search:bed}{}
					\begin{spoiler}[search:bed]{침대}{[튜토리얼:조사대상]}
						\entry[\hline]{
							\begin{spoiler}[search:strange-box]{수상한 상자}{[튜토리얼:조사대상]}
								\entry{열쇠구멍이 하나 있을 뿐, 이음새도 없는 수상한 상자입니다. 안쪽에 무언가 들어있는 느낌은 납니다.}
								
								\entry{이 상자를 부숴서 열려고 하면, 수면 가스가 새어 나와 모두 잠에 빠져듭니다. 1시간 후, 다시 깨어나지만 상자는 비어있습니다.}
								
								\entry{상자를 여는 방법은 세 가지가 있습니다:
									\begin{enumerate}
										\item \storyref{search:bell}{종} 안에 있는 열쇠로 열기.
										\item 사제의 피를 한 방울 이상 열쇠 구멍 안에 넣기.
										\item \storyref{rogue:lockpick}{자물쇠 따기}로 열기.
									\end{enumerate}
									이 이외의 방법으로 열고자 시도하는 것은 통하지 않거나, 부수는 것으로 취급합니다.
								}
								
								\entry[\hline]{
									\begin{spoiler}[search:pentagram-note]{오각별이 그려진 책}{[튜토리얼:조사대상]}
										\entry[\hline]{[타락의 노래]라는 제목의 악보입니다. 엄청나게 두꺼워, 연주하는데에 적게 잡아도 여섯시간정도는 걸릴 것 같습니다.}
									\end{spoiler}
								}
							\end{spoiler}}
					\end{spoiler}
					
					\storyref{search:strange-box}{수상한 상자}를 부숴서 열었고, 수면 가스가 새어 나와 모두 잠에 빠져들었다면, 사제가 \storyref{search:pentagram-note}{오각별이 그려진 책}을 빼돌렸습니다. 다른 이야기꾼들은 이 책을 조사할 수 없습니다.
				
				\subsubsection*{예배당}
					예배당 안에서는 수상한 음악소리가 흘러나옵니다. 또한, 안쪽에는 좀비가 너무 많아, 들어가면 좀비를 어떻게든 처리하기 전에는 돌아나오기 힘들것 같다고 강조해주세요.
					
					들어가면 최종 전투가 시작됩니다.
	
	\pagebreak
	\section{예배당}
		최종 전투가 시작됩니다. \hyperlink{alternative:no-criminal}{사제 역할의 소유자가 흑막 역할을 거부한 경우}에는 아래 내용을 기본으로 하여, 바뀌는 부분에 대해서는 해당 부분을 따릅니다. 이 부분에서는 사제 역할의 소유자가 흑막 역할을 수용한 경우를 가정합니다.
		
		전투의 순서는 기민 스탯과 상관 없이 이단심판관(도적) $\rightarrow$ 성기사 $\rightarrow$ 사제 $\rightarrow$ 좀비 $\rightarrow$ 성가단원의 순서로 고정됩니다.
		
		예배당의 구조는 다음과 같습니다.
		
		\begin{center}
			\begin{tabular}{p{1.5cm}p{1.5cm}p{1.5cm}|p{1.5cm}|p{1.5cm}|p{1.5cm}p{1.5cm}p{1.5cm}}
				\cline{4-5}
				&                         &   & \multicolumn{2}{c|}{제대} &                       &                       &                       \\ \hline
				\multicolumn{1}{|c|}{} & \multicolumn{2}{c|}{좌측 오르간} &            &            & \multicolumn{2}{c|}{우측 오르간}                   & \multicolumn{1}{c|}{} \\ \hline
				\multicolumn{1}{|c|}{} & \multicolumn{2}{c|}{의자 1}   &            &            & \multicolumn{2}{c|}{의자 2}                     & \multicolumn{1}{c|}{} \\ \hline
				\multicolumn{1}{|c|}{} & \multicolumn{2}{c|}{의자 3}   &            &            & \multicolumn{2}{c|}{의자 4}                     & \multicolumn{1}{c|}{} \\ \hline
				\multicolumn{1}{|c|}{} & \multicolumn{2}{c|}{의자 5}   &            &            & \multicolumn{2}{c|}{의자 6}                     & \multicolumn{1}{c|}{} \\ \hline
				\multicolumn{1}{|c|}{} & \multicolumn{1}{c|}{}   &   & \makecell{\centering 입}          & \makecell{\centering 구}          & \multicolumn{1}{c|}{} & \multicolumn{1}{c|}{} & \multicolumn{1}{c|}{} \\ \hline
			\end{tabular}
		\end{center}
		
		\begin{spoiler}{밀어내기}{[행위:이야기꾼]}
			\entry[\hline]{[기절] 상태인 좀비를 원하는 방향으로 한 칸 밀어낼 수 있다.}
		\end{spoiler}
		
		\begin{spoiler}{제대}{[신성]}
			\entry[\hline]{제대 구역으로 밀려난 좀비는 즉시 무력화되어, 아무 행동도 할 수 없다.}
		\end{spoiler}
		
		\begin{spoiler}{의자}{[물체]}
			\entry[\hline]{마을 주민들이 빽빽하게 앉아있는 의자. 좀비는 이 의자를 마음대로 드나들 수 있지만, 이야기꾼은 드나들 수 없다.}
		\end{spoiler}
		
		\begin{spoiler}{오르간}{[물체][음악]}
			\entry[\hline]{오르간에서 연주된 음악의 효과는 예배당 전체에 영향을 끼친다.}
		\end{spoiler}
		
		사제는 예배당에 들어온 순간, 본색을 드러내며 충격에 빠진 이야기꾼들을 제치고 좌측 오르간으로 가 앉아, [타락의 노래]를 연주합니다(\storyref{priest:fallen-song}{타락의 연주자}).
		
		이야기꾼들은 입구칸 중 원하는 칸에서 시작합니다. 이야기꾼이나 흑막이 아닌 등장인물은 시스템의 가호를 받지 못해 좀비로 변화합니다.
		
		좀비는 최초에는 6시로부터 지난 시간 1시간당 한 개체씩, 최대 6개체까지 서로 다른 의자에서 일어난 채로 시작합니다. 좀비가 없는 의자는 주사위를 굴려 결정합니다.
		
		\begin{spoiler}[zombie:fallen]{타락의 손길}{[좀비]}
			\entry{좀비의 턴에, 좀비는 가장 가까이에 있는 이야기꾼을 향해 한 칸 이동한다.}
			
			\entry{좀비와 같은 칸에서 턴을 끝내는 이야기꾼은 좀비 하나당 체력, 정신력, 또는 개연성을 1 잃는다. 이는 물리적인 공격이 아니고 타락으로 인한 오염으로, 방지되거나 회피될 수 없다.}
			
			\entry[\hline]{턴을 시작할 때에 같은 칸에 있는 좀비의 수를 본인을 포함한 이야기꾼의 수로 나눈 수가 이야기꾼의 [근력] 수치의 두 배 초과인 경우, 이야기꾼은 이동할 수 없다. 좀비가 없을 때에는 언제든지 이동할 수 있다.}
		\end{spoiler}
		
		이 상황을 타개할 수 있는 방법은 두 가지입니다:
		\begin{enumerate}
			\item 흑막을 죽음에 빠트리기
				\subitem{$\rightarrow$} 흑막을 죽이면 모든 마을 주민은 가사상태에 빠지지만, 살아는 있습니다. 이 중 일부는 다시 살아갈 것입니다.
			\item{} [정화의 노래]를 오른쪽 오르간에서 세 턴간 연주하기
				\subitem{$\rightarrow$} [정화의 노래]는 [순수한 영혼]을 가진 성가대원이 연주해야만 효과를 발휘합니다. [정화의 노래]를 연주하는 동안, 연주자는 \storyref{zombie:fallen}{타락의 손길}으로 인한 피해량을 무시합니다. 세 턴이 완료되는 순간, 모든 마을 주민은 좀비 상태에서 다시 인간 상태로 되돌아오는 상태가 됩니다. 이 때에 흑막은 이야기 밖으로 추방됩니다.
		\end{enumerate}
		
		이가 일어나기 전에, 흑막을 제외한 모든 이야기꾼의 개연성이 0으로 떨어진다면 이 이야기 속의 인류는 모두 좀비가 되어 멸망하게 됩니다.
	
	\section{다시, 태초의 이야기(선택)}
		이야기꾼들은 태초의 이야기에서 서로가 가진, 하지만 말할 수 없었던 정보로 진상을 다시 한번 짚어가며 마무리합니다.
	
\end{document}
	
	\chapter{등장인물들의 진실된 이야기}
		\documentclass{report}

\begin{document}
	개별 이야기꾼들이 이미 알고 있는 정보를 전달하고, 얻을 수 있는 이야기들과 획득을 위한 필요 조건, 그리고 칭호의 효과와 칭호의 \textbf{공개 조건}과 \textbf{제거 조건}을 명백하게 밝혀야 합니다. 이 이야기들은 페이지 단위로 잘려 있어, 정보를 제공할 때 페이지 단위로 제공할 수 있게 했습니다.
	
	\pagebreak \hypertarget{cursed-bard}{}
	\section{저주받은 성가단원}
		\documentclass{report}

\begin{document}
	\subsection*{알고 있는 정보}
		당신은 사람들이 처음으로 언데드로 변하는 것을 본 꼬마아이입니다. 당신의 보호자인 성가단장 역시 좀비로 변해버렸고요.
		
		최근 들어 성가단의 단장이자 당신에게 오르간을 연주하는 법을 가르쳐준 당신의 보호자는 당신이 잘 때 밤늦게 어딘가로 향하는 일이 잦아졌습니다. 공책을 들고가는거로 봐서는 뭔가 적을것이 있는 것 같은데, 그게 무엇일까요?
	
	\subsection*{가지고 시작하는 이야기}
		\begin{story}[choir:bard]{성가대}{[역할]}
			\entry{노래를 부르거나 악기를 연주하여 음악을 통해 마법을 사용할 수 있다. 아래 노래들을 사용할 수 있다.}
			
			\entry{
				\begin{story}[music:holy]{성가}{[노래]}
					\entry[\hline]{한 턴에 이 노래를 부르기로 선택한다면 이동을 제외한 다른 행동을 할 수 없다. 턴이 종료될 때, 같은 구역에 있는 모든 생명체의 체력을 1 회복하고, 언데드에게 [신성] 피해를 1 준다.}
				\end{story}
			}
			
			\entry[\hline]{\statchange{+}{지식:음악, 제작:음악}}
		\end{story}
	
	\subsection*{획득 가능한 이야기}
		\begin{story}[bard:cursed]{저주받은}{[공포][칭호]}
			\pre{언데드에게 효과가 적용되도록 \storyref{choir:bard}{성가대}의 노래를 부른다.}
			
			\limitedtrauma{공포}{\storyref{choir:bard}{성가대}의 노래들의 효과가 언데드를 상대로는 적용되지 않는다.}
			
			\entry[\hline]{\textbf{제거 조건}: 특정 아이템을 소유하고 있는 동안, 또는 어떤 사실을 알게 되면 이 칭호를 무시한다. 이 시점은 시스템이 알려준다.}
		\end{story}
	
	\subsection*{직업 퀘스트}
		\begin{story}{성가단원}{[직업]}
			\entry{\statchange{+}{지식:음악, 제작:음악}}
			
			\entry{\begin{story}[bard:blessed]{축복}{[퀘스트]}
				
				\entry{\textbf{성공 조건}
				
				칭호 \storyref{bard:cursed}{저주받은} 제거}
				
				\entry[\hline]{\textbf{보상}
				
				\statchange{+}{지식:신성}
				
				\storyref{music:holy}{성가}의 회복량이 1 증가하고, 언데드에게 [신성] 피해를 주는 대신 인접한 구역으로 밀쳐내며 [기절: 1턴] 상태에 빠트릴 수 있다.}
			
			\end{story}}
			
			\entry[\hline]{\begin{story}{음악}{[퀘스트]}
					\entry{\textbf{성공 조건}
					
					성가단의 연습 장소에 있는 악보를 정리하여, 음악을 연주한다.}
					
					\entry[\hline]{\textbf{보상}
					
					연주한 음악을 \storyref{music:holy}{성가}와 더불어 기억할 수 있다.
					
					그 이외의 곡들은 악보를 직접 소유하고 있다면 연주하거나 노래할 수 있다.}
				
			\end{story}}
		\end{story}
\end{document}%
	
	\pagebreak \hypertarget{corrupt-paladin}{}
	\section{타락한 성기사}
		\documentclass{report}

\begin{document}
	\subsection*{알고 있는 정보}
		이 마을에 오래전에 살고 있던 사람 중에는 전설적인 작곡가이자 사제가 있었습니다. 마을의 오르간 연주자의 부탁을 받아 그 사람의 유작을 찾아나선 당신은 그의 무덤을 파헤쳤고, 그의 관에 새겨져있던 노래를 하나 찾게 되었습니다. 좀비 사태가 발생했을 때, 당신은 무덤의 뒤처리를 하고 있었고요. 당신은 이로 인해 당신이 타락하게 되었다는 사실을 알고 있기 때문에, 다른 등장인물들이 무덤을 발견하기 전에는 이 사실을 최대한 숨기고자 합니다.
		
		당신의 집에는 이 무덤을 파헤쳐서 젖은 흙이 묻은 삽이 현관문 뒤에 있고, 당신이 사용했던 방패와 십자가에는 각각 [흑마법사의 피]와 [순수한 피]가 숨겨져 있습니다. [순수한 피]는 수년 전, 당신의 보호 하에 있었지만 지키지 못했던 첫 번째 이의 것으로, [순수한 피] 끼리는 잘 섞인다는 성질을 가지고 있기에 이를 이용해 숨겨두었던 자신의 아기를 찾고, 아기를 보호해달라는 부탁을 받았습니다.
	
	\subsection*{가지고 시작하는 이야기}
		\begin{spoiler}[paladin:fallen]{타락한}{[공포][칭호]}
			\limitedtrauma{공포}{\storyref{paladin:smite}{신성한 일격}을 사용할 수 없다.}
			
			\entry[\hline]{\textbf{제거 조건}: 자신의 집 안에 있는 [순수한 피]를 마시거나, 특정한 음악을 듣는다.}
		\end{spoiler}
		
		\begin{spoiler}[paladin:smite]{신성한 일격}{[역할]}
			\entry[\hline]{사거리에 상관 없이 대상을 정한다. 대상에게 [신성] 피해를 2 주고, [기절] 상태에 빠트린다. [기절] 상태의 상대는 다음 턴 모든 행동이 불가능하다.}
		\end{spoiler}
	
	\subsection*{직업 퀘스트}
		\begin{spoiler}{성기사}{[직업]}
			\entry{\statchange{+}{근력, 의지}}
			
			\entry{\begin{spoiler}{순수한 지식}{[퀘스트]}
						
						\entry{\textbf{성공 조건}
							
						[순수한 피]를 얻는다.}
						
						\entry[\hline]{\textbf{보상}
							
						\statchange{+}{지식:신성[2]}}
						
			\end{spoiler}}
			
			\entry[\hline]{\begin{spoiler}{타락한 지식}{[퀘스트]}
						
						\entry{\textbf{성공 조건}
							
						[흑마법사의 피]를 얻는다.}
						
						\entry[\hline]{\textbf{보상}
							
						\statchange{+}{지식:흑마법[2]}}
						
			\end{spoiler}}
		\end{spoiler}
	
	\subsection*{획득 가능한 이야기}
		\begin{spoiler}{흑마법사의 피}{[타락]}
			\entry[\hline]{인접한 구역에 이 피를 흩뿌릴 수 있다. 다음 두 턴 동안, 좀비는 반드시 해당 칸을 향해서 이동하며, 이야기꾼들을 공격하지 않는다.}
		\end{spoiler}
		
		\begin{spoiler}[paladin:protect]{권능의 보호막}{[신성]}
			\pre{\storyref{paladin:fallen}{타락한} 칭호 제거, \storyref{search:paladin-shield}{낡은 방패} 장착}
			
			\entry[\hline]{세 턴에 한 번 사용할 수 있다. 다음 자신의 턴까지, 타락한 존재들은 드나들 수 없는 방벽을 자신이 있는 구역에 칠 수 있다. 해당 칸에 있는 타락한 존재는 무작위 인접한 구역으로 밀려난다.}
		\end{spoiler}
\end{document}%
	
	\pagebreak \hypertarget{cowardly-priest}{}
	\section{겁에 질린 사제}
		\documentclass{report}

\begin{document}
	\subsection*{알고 있는 정보}
		당신이 이 좀비 사태의 원흉입니다. 당신은 사실 악마를 숭배하는 이교도이며, 이 악마를 숭배하고 소환하기 위해 악마의 가르침을 담은 책을 성경 표지만 덧씌워두었습니다. 당신은 이런 노력을 통해 악마와 계약해 [타락의 노래]를 얻었고, 이를 이용해 오르간 연주자부터 시작하여 사람들을 좀비로 바꾸었습니다.
		
		당신의 방 안의 침대 밑에는 당신의 피를 이용하거나, 종 안에 숨겨진 열쇠를 사용해서만 안에 들어있는 해골과 악마에게서 받은 [타락의 노래]를 발견할 수 있다는 사실을 알고 있습니다. 상자가 부서지면 수면가스가 나오도록 되어 있어 어느 정도의 보호조치를 해 두었습니다.
		
		이 모든 이야기는 예배당에 진입하기 이전까지 자의적으로 공개할 수 없습니다.
	
	\subsection*{가지고 시작하는 이야기}
		\begin{spoiler}{겁에 질린}{[공포][칭호]}
			\limitedtrauma{공포}{\storyref{cleric:prayer}{기도}로 언데드에게 피해를 줄 수 없다.}
			
			\entry{\textbf{공개조건}: 공개 불가.}
			
			\entry[\hline]{\textbf{제거 조건}: 예배당에 진입한다.}
		\end{spoiler}
		
		\begin{spoiler}[cleric:prayer]{기도}{[역할]}
			\entry[\hline]{사거리에 상관 없이 대상을 정한다. 대상의 다음 턴이 시작될 때, 생명체인 대상의 체력을 2 회복시키거나, 언데드인 대상에게 [신성] 피해를 2 준다.}
		\end{spoiler}
	
	\subsection*{직업 퀘스트}
		사제의 직업 퀘스트는 이야기의 의지가 내린 것이 아니고, 악마로서 강림하여 이야기를 오염시켜 자신의 것으로 만든 [타락한 자]가 내린 것입니다.
		
		\begin{spoiler}{대악마의 사제}{[직업]}
			\entry{\statchange{+}{지식:신성[2], 지식:흑마법[2]}}
			
			\entry[\hline]{\begin{spoiler}{비밀 숭배}{[퀘스트]}
						
						\entry{\textbf{성공 조건}
							
							예배당 돌입 전까지, 악마 숭배 사실을 들키지 않는다.
						}
						
						\entry[\hline]{\textbf{보상}
							
							오르간을 연주하는 동안, 매 턴 개연성을 1d6 회복한다.
							
							흑막을 거부했다면, \storyref{cleric:prayer}{기도}의 이름이 [절박한 기도]로 바뀌며, 이를 통해 이제 아무 대상의 개연성을 2 회복시키거나, [신성] 피해를 2 주거나, [타락] 피해를 2 줄 수 있다.
						}
						
			\end{spoiler}}
		\end{spoiler}
	
	\subsection*{획득 가능한 이야기}
		악마 숭배 사실을 들키면, \storyref{cleric:prayer}{기도}의 이름이 [저주]로 바뀌고, 이를 포함한 언데드와 생명체에 서로 다른 효과를 주는 능력 모두의 언데드에 대한 효과와 생명체에 대한 효과가 뒤바뀌며, 언데드에게 [신성] 피해를 주는 기술의 경우 생명체에게 [타락] 피해를 줍니다\footnote{[저주]의 경우, 언데드의 체력을 2 회복하거나, 생명체인 대상에게 [타락] 피해를 2 줍니다.}.
		
		또한, 흑막을 거부하지 않았다면 예배당에 진입하면 자동으로 왼쪽의 오르간에 앉으며, \storyref{dark-age}{어둠의 시대}의 효과를 전부 무시하고, 다음 이야기를 얻습니다:
		\begin{spoiler}[priest:fallen-song]{타락의 연주자}{[타락의 노래:2악장]}
			\entry{자신에게 적용되는 특정 이야기의 효과가 생명체와 언데드에게 서로 다른 효과를 발휘한다면, 어느 쪽을 따를지를 선택할 수 있다.}
			
			\entry{자신의 턴이 진행되는 중, 방어적인 행동 또는 이동만 한 채로 오르간 앞에 있다면 다음 자신의 턴까지 [타락의 노래]를 연주하기로 선택할 수 있다. 이를 연주한다면 자신의 턴을 즉시 종료하지만, d6을 굴려, 숫자에 해당하는 의자에서 좀비가 하나 추가로 소환된다.}
			
			\entry{턴 종료시 오르간 앞에서 벗어나있다면 개연성에 피해를 1 받는다.}
			
			\entry[\hline]{체력, 정신력, 개연성이 모두 개연성으로 통합된다.}
		\end{spoiler}
\end{document}%
	
	\pagebreak \hypertarget{hurt-rogue}{}
	\section{부상당한 이단심판관}
		\documentclass{report}

\begin{document}
	\subsection*{알고 있는 정보}
		당신은 도둑입니다. 여러 마을을 돌아다니며 이단심판관인척 하며 그들의 재산을 훔쳤죠.
		
		이번에 당신이 노리는 것은 어떤 작곡가의 유작입니다. [정화의 노래]로 알려진 이 노래는 어느샌가 기억에서 사라졌지만, 이 마을 출신이었고 여기에 묻히기까지 했으니 흔적은 남아 있겠죠. 이 노래에 대해 조사한 결과, "순수한 피"를 가진 이가 연주해야만 효과가 있다고 하는데, 마침 이 마을에 "순수한 피"를 가진 이와 접촉한 성기사가 있다는 사실을 이전에 있던 마을의 흑마술사로부터 알아냈습니다. 이 마을에서 성기사의 도움을 얻고자 마을의 수장이나 다름없는 사제에게 도움을 청하기 위해, 이 두 명에 대한 뒷조사를 간략하게나마 하고 왔습니다. 물론, 불법이니만큼 들키기 전에는 말할 생각은 없지만요.
		
		당신의 가방 안에는 이 뒷조사 자료와 [정화의 노래]에 대한 자료, 그리고 당신이 사용하는 락픽과 단도, 만능툴들이 무기 주머니에 들어있습니다. 다른 사람들이 당신을 알아보지 못하기를, 그리고 이 가방을 열지 않기를 바라는수밖에는 없겠군요.
	
	\subsection*{가지고 시작하는 이야기}
		\begin{spoiler}[rogue:hurt]{부상당한}{[공포][칭호]}
			\limitedtrauma{공포}{언데드를 대상으로 한 무기와 \storyref{rogue:judgment}{심판}의 사용이 불가능하다.}
			
			\entry[\hline]{\textbf{제거 조건}: 자신의 진정한 정체를 다른 사람들이 추궁한다. 이 제거 조건은 공개할 수 없다.}
		\end{spoiler}
		
		\begin{spoiler}[rogue:judgment]{심판}{[역할]}
			\entry{\storyref{rogue:hurt}{부상당한}이 제거되면, 이 이야기를 잃는다. 이 조건은 공개할 수 없다.}
			
			\entry[\hline]{씬이 시작할 때, [심판]의 대상을 하나 정한다. 해당 대상에게 주는 모든 피해가 1 증가한다.}
		\end{spoiler}
	
	\subsection*{획득 가능한 이야기}
	\begin{spoiler}[rogue:dagger]{단도 투척}{[역할]}
		\pre{칭호 \storyref{rogue:hurt}{부상당한} 제거}
		
		\entry[\hline]{매 턴 한 번, 단도를 던질 수 있다. 단도는 떨어진 구역당 피해 1을 준다.}
	\end{spoiler}
	
	\begin{spoiler}[rogue:lockpick]{자물쇠 따기}{[역할]}
		\pre{칭호 \storyref{rogue:hurt}{부상당한} 제거}
		
		\entry[\hline]{잠겨있는 물체를 30분을 소모하여 열 수 있습니다.}
	\end{spoiler}
	
	\subsection*{직업 퀘스트}
		\begin{spoiler}{이단심판관}{[직업]}
			\entry{\statchange{+}{기민[2]}}
			
			\entry{\begin{spoiler}{비밀의 수호자}{[퀘스트]}
						
						\entry{\textbf{성공 조건}
							\storyref{rogue:hurt}{부상당한}이 마을 내부에서 교회 구역으로 들어가기 전까지 제거당하지 않는다.
						}
						
						\entry[\hline]{\textbf{보상}
							\storyref{rogue:judgment}{심판}을 \storyref{rogue:hurt}{부상당한}을 잃더라도 사용할 수 있으나, 이름이 [쇠약의 독]으로 바뀐다.
						}
						
			\end{spoiler}}
			
			\entry[\hline]{\begin{spoiler}{영원한 비밀}{[퀘스트]}
					
					\entry{\textbf{성공 조건}
						\storyref{rogue:hurt}{부상당한}을 교회 내부로 들어갈 때 까지 제거당하지 않는다.
					}
					
					\entry[\hline]{\textbf{보상}
						\storyref{rogue:judgment}{심판}의 이름이 [이단의 심판]으로 바뀌며, 피해 증가량이 피해량의 50\%(최소 1)로 증가한다.
					}
					
			\end{spoiler}}
		\end{spoiler}
\end{document}%
	
\end{document}%
	
	\chapter{대체 세계선}
		\documentclass{report}

\begin{document}
	\hypertarget{alternative:no-criminal}{}
	\section{사제 역할의 소유자가 흑막 역할을 거부한 경우}
		이야기의 진행은 기본적으로 동일합니다. 하지만, 최종 전투에서 드러나는 흑막이 사제가 아닌 사제가 숭배하는 악마로 변경되어, 이야기를 조금 다르게 풀어나가게 됩니다.
		
		좌측 오르간에는 인간으로 보이지 않는 존재가 앉아있습니다. 이는 악마로서 이야기에 들어온 타락한 자로, \storyref{priest:fallen-song}{타락의 연주자} 이야기를 사제로부터 강제로 빼앗고, 자신이 원천이 되어 지급한 이야기이기 때문에 좀비를 소환할 때 3d10이 아닌 5d10을 굴려 좀비가 소환됩니다. 또한, 전투를 시작할 때에 좀비의 스폰 방법은 동일하지만, 즉시 \storyref{priest:fallen-song}{타락의 연주자}로 좀비를 1회 소환합니다.
		
		타락한 자는 위와 같이 좀비를 더 많이 소환하는 대신, 아무 행동도 하지 않으며, \storyref{priest:fallen-song}{타락의 연주자}의 마지막 효과인 자신의 가호를 받을 수 없기 때문에 총 75의 피해를 받으면 사망합니다. 종료 조건은 전원 사망 조건에서 사제까지 개연성이 0이 되어야 한다는 점을 제외한다면 이전과 동일합니다.
	
	\hypertarget{alternative:war-ready}{}
	\section{호전적인 이야기꾼들이 등장인물로서 들어온 경우}
		구역을 넘어갈 때 소요되는 30분을 전투로 대체합니다. 이 경우, 다음 구역으로 넘어가는 길을 다음과 같이 4 $\times$ 7의 구역으로 표현합니다:
		
		\begin{center}
			\begin{tabular}{|c|c|c|c|c|c|c|}
				\hline
				이 & 01 & 02 & 03 & 04 & 05 & 도 \\\hline
				야 & 06 & 07 & 08 & 09 & 10 & 착 \\\hline
				기 & 11 & 12 & 13 & 14 & 15 & 지 \\\hline
				꾼 & 16 & 17 & 18 & 19 & 20 & 점 \\\hline
			\end{tabular}
		\end{center}
		
		이야기꾼이 아닌 등장인물은 도착지점까지 들키지 않고 빠르게 갈 수 있는 길을 각자 정확하게 알고 있기 때문에, 이야기꾼들만 이 좀비가 가득한 길을 통해서 도착지점까지 도착하면 됩니다.
		
		이야기꾼의 순서는 최종 전투와 같이, 이단심판관(도적) $\rightarrow$ 성기사 $\rightarrow$ 사제 $\rightarrow$ 좀비 $\rightarrow$ 성가단원의 순서대로 진행합니다. 이야기꾼들은 한 턴에 두 칸 이동할 수 있지만, 좀비가 있는 칸에 들어선다면 반드시 이동을 멈춰야 합니다.
		
		좀비를 소환할 때에는, d20을 굴려 해당하는 숫자의 칸에 소환됩니다. 해당 칸에 좀비가 있어도 중첩되어 소환됩니다.
		
		좀비가 공격할 때에는 공격을 받는 이야기꾼의 선택에 따라 체력 또는 정신력 중 하나를 1 감합니다. 이 피해는 경감될 수 없으며, 체력과 정신력 양쪽 모두가 0일 때에는 개연성을 감합니다.
		
		좀비의 턴에는, \storyref{zombie:anger-aroused}{점점 격렬해지는 분노}의 ``시간이 지날수록 공격성이 강해진다"는 문구에 따라, 전투가 시작할 때 [기본] 항목만큼의 좀비가 소환되어, 턴이 시작할 때 [좀비 스폰] 수에 따라 좀비를 소환하고, [공격성] 규칙에 따라 공격하고 이동합니다.
		
		\begin{tabularx}{\linewidth}{c|c|l|X}
			\makecell{\centering\textbf{시간}} & \textbf{기본} & \makecell{\centering\textbf{좀비 스폰}} & \makecell{\centering\textbf{공격성}}\\\hline\hline
			6시\textasciitilde8시 & 0 & 좀비 한 구 & 좀비를 마지막으로 공격한 사람 쪽으로 이동합니다. 좀비를 공격하지 않는다면, 이동하지 않습니다. \\\hline
			8시\textasciitilde10시 & 2 & 좀비 두 구 & 좀비를 마지막으로 공격한 사람 쪽으로 이동합니다. 좀비를 공격하지 않는다면, 무작위로 한 칸 이동합니다. \\\hline
			10시\textasciitilde 자정 & 5 & 좀비 세 구 & 가장 가까운 이야기꾼에게 한 칸 이동합니다. \\\hline
			자정 이후 & 10 & 좀비 세 구 & 가장 가까운 이야기꾼에게 한 칸 이동합니다.
		\end{tabularx}
		
		좀비의 턴이 끝날 때, 좀비와 같은 칸에 있는 모든 이야기꾼은 [의지] 판정을 합니다. 판정값이 (좀비의 수)-1 미만이라면, 타락으로 인한 방지 또는 회피될 수 없는 피해를 체력, 정신력을 합쳐 3 받거나 개연성에 피해를 1 받습니다.
		
		네 명의 이야기꾼이 모두 도착지점에 도착하면 다음 구역으로 넘어갑니다. 시간은 소모된 턴 수에 따라 계산해도 무방하고, 10분\textasciitilde15분으로 고정해도 괜찮습니다.
		
		네 명의 이야기꾼 중 하나라도 도착지점에 도착하기 전에 개연성이 0이 되어 추방된다면, 모든 이야기꾼이 안쪽 구역에 도착한 채로 1시간이 지난 후 기절했다 깨어납니다.
		
		이야기꾼이 네명 미만이고 나머지를 NPC로 대체한 경우, 해당 인원들은 마을 구조나 기술을 잘 사용해서 알아서 빠져나간 것으로 취급해도 무방합니다. 이렇게 하지 않는다면, NPC까지 같이 통과해야 하는, 난이도가 더 높은 전투가 됩니다.
	
\end{document}
		
	\chapter*{스포일러 방지 및 메모용 빈 페이지 입니다.}
\end{document}
	
	\label{endof_Unliving}
	
\end{document}
