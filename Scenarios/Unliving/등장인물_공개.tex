\documentclass{report}

\begin{document}
	1인인 경우에는 성가단원과 성기사 중 한 명을 선택합니다.
	
	2\textasciitilde3인인 경우에는 사제를 제외한 등장인물들 중 한 명씩을 선택합니다.
	
	\section*{성가단원}
	
	\begin{story}{성가대}{[역할]}
		\entry[\hline]{노래를 부르거나 악기를 연주하여 음악을 통해 마법을 사용할 수 있다. 다음 노래를 사용할 수 있다:
		\begin{story}{성가}{[노래]}
			\entry[\hline]{한 턴에 이 노래를 부르기로 선택한다면 이동을 제외한 다른 행동을 할 수 없다. 턴이 종료될 때, 같은 구역에 있는 모든 생명체의 체력을 1 회복하고, 언데드에게 [신성] 피해를 1 준다.}
		\end{story}
		}
	\end{story}
	
	\section*{성기사}
	
	\begin{story}{신성한 일격}{[역할]}
		\entry[\hline]{사거리에 상관 없이 대상을 정한다. 대상에게 [신성] 피해를 2 주고, [기절] 상태에 빠트린다. [기절] 상태의 상대는 다음 턴 모든 행동이 불가능하다.}
	\end{story}
	
	\section*{사제}
	
	\begin{story}{기도}{[역할]}
		\entry[\hline]{사거리에 상관 없이 대상을 정한다. 대상의 다음 턴이 시작될 때, 대상의 체력을 2 회복시키거나, [신성] 피해를 2 준다.}
	\end{story}
	
	\section*{이단심판관}
	
	\begin{story}{심판}{[역할]}
		\entry[\hline]{씬이 시작할 때, [심판]의 대상을 하나 정한다. 해당 대상에게 주는 모든 피해가 1 증가한다.}
	\end{story}
\end{document}