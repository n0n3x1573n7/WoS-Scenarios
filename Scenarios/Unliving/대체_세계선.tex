\documentclass{report}

\begin{document}
	\hypertarget{alternative:no-criminal}{}
	\section{사제 역할의 소유자가 흑막 역할을 거부한 경우}
		이야기의 진행은 기본적으로 동일합니다. 하지만, 최종 전투에서 드러나는 흑막이 사제가 아닌 사제가 숭배하는 악마로 변경되어, 이야기를 조금 다르게 풀어나가게 됩니다.
		
		좌측 오르간에는 인간으로 보이지 않는 존재가 앉아있습니다. 이는 악마로서 이야기에 들어온 타락한 자로, \storyref{priest:fallen-song}{타락의 연주자} 이야기를 사제로부터 강제로 빼앗고, 자신이 원천이 되어 지급한 이야기이기 때문에 좀비를 소환할 때 3d10이 아닌 5d10을 굴려 좀비가 소환됩니다. 단, 셋 이상의 좀비가 한 의자에서 일어나야 입구에서 좀비가 추가로 소환됩니다. 또한, 전투를 시작할 때에 좀비의 스폰 방법은 동일하지만, 즉시 \storyref{priest:fallen-song}{타락의 연주자}로 좀비를 1회 소환합니다.
		
		타락한 자는 위와 같이 좀비를 더 많이 소환하는 대신, 아무 행동도 하지 않으며, \storyref{priest:fallen-song}{타락의 연주자}의 마지막 효과인 자신의 가호를 받을 수 없기 때문에 총 75의 피해를 받으면 사망합니다. 종료 조건은 전원 사망 조건에서 사제까지 개연성이 0이 되어야 한다는 점을 제외한다면 이전과 동일합니다.
	
	\hypertarget{alternative:war-ready}{}
	\section{호전적인 이야기꾼들이 등장인물로서 들어온 경우}
		구역을 넘어갈 때 소요되는 30분을 전투로 대체합니다. 이 경우, 다음 구역으로 넘어가는 길을 다음과 같이 4 $\times$ 7의 구역으로 표현합니다:
		
		\begin{center}
			\begin{tabular}{|c|c|c|c|c|c|c|}
				\hline
				이 & 01 & 02 & 03 & 04 & 05 & 도 \\\hline
				야 & 06 & 07 & 08 & 09 & 10 & 착 \\\hline
				기 & 11 & 12 & 13 & 14 & 15 & 지 \\\hline
				꾼 & 16 & 17 & 18 & 19 & 20 & 점 \\\hline
			\end{tabular}
		\end{center}
		
		이야기꾼이 아닌 등장인물은 도착지점까지 들키지 않고 빠르게 갈 수 있는 길을 각자 정확하게 알고 있기 때문에, 이야기꾼들만 이 좀비가 가득한 길을 통해서 도착지점까지 도착하면 됩니다.
		
		이야기꾼의 순서는 최종 전투와 같이, 이단심판관(도적) $\rightarrow$ 성기사 $\rightarrow$ 사제 $\rightarrow$ 좀비 $\rightarrow$ 성가단원의 순서대로 진행합니다. 이야기꾼들은 한 턴에 두 칸 이동할 수 있지만, 좀비가 있는 칸에 들어선다면 반드시 이동을 멈춰야 합니다.
		
		좀비를 소환할 때에는, d20을 굴려 해당하는 숫자의 칸에 소환됩니다. 해당 칸에 좀비가 있어도 중첩되어 소환됩니다.
		
		좀비가 공격할 때에는 공격을 받는 이야기꾼의 선택에 따라 체력 또는 정신력 중 하나를 1 감합니다. 이 피해는 경감될 수 없으며, 체력과 정신력 양쪽 모두가 0일 때에는 개연성을 감합니다.
		
		좀비의 턴에는, \storyref{zombie:anger-aroused}{점점 격렬해지는 분노}의 ``시간이 지날수록 공격성이 강해진다"는 문구에 따라, 전투가 시작할 때 [기본] 항목만큼의 좀비가 소환되어, 턴이 시작할 때 [좀비 스폰] 수에 따라 좀비를 소환하고, [공격성] 규칙에 따라 공격하고 이동합니다.
		
		\begin{tabularx}{\linewidth}{c|c|l|X}
			\makecell{\centering\textbf{시간}} & \textbf{기본} & \makecell{\centering\textbf{좀비 스폰}} & \makecell{\centering\textbf{공격성}}\\\hline\hline
			6시\textasciitilde8시 & 0 & 좀비 한 구 & 좀비를 마지막으로 공격한 사람 쪽으로 이동합니다. 좀비를 공격하지 않는다면, 이동하지 않습니다. \\\hline
			8시\textasciitilde10시 & 2 & 좀비 두 구 & 좀비를 마지막으로 공격한 사람 쪽으로 이동합니다. 좀비를 공격하지 않는다면, 무작위로 한 칸 이동합니다. \\\hline
			10시\textasciitilde 자정 & 5 & 좀비 세 구 & 가장 가까운 이야기꾼에게 한 칸 이동합니다. \\\hline
			자정 이후 & 10 & 좀비 세 구 & 가장 가까운 이야기꾼에게 한 칸 이동합니다.
		\end{tabularx}
		
		좀비의 턴이 끝날 때, 좀비와 같은 칸에 있는 모든 이야기꾼은 [의지] 판정을 합니다. 판정값이 (좀비의 수)-1 미만이라면, 타락으로 인한 방지 또는 회피될 수 없는 피해를 체력, 정신력을 합쳐 3 받거나 개연성에 피해를 1 받습니다.
		
		네 명의 이야기꾼이 모두 도착지점에 도착하면 다음 구역으로 넘어갑니다. 시간은 소모된 턴 수에 따라 계산해도 무방하고, 10분\textasciitilde15분으로 고정해도 괜찮습니다.
		
		네 명의 이야기꾼 중 하나라도 도착지점에 도착하기 전에 개연성이 0이 되어 추방된다면, 모든 이야기꾼이 안쪽 구역에 도착한 채로 1시간이 지난 후 기절했다 깨어납니다.
		
		이야기꾼이 네명 미만이고 나머지를 NPC로 대체한 경우, 해당 인원들은 마을 구조나 기술을 잘 사용해서 알아서 빠져나간 것으로 취급해도 무방합니다. 이렇게 하지 않는다면, NPC까지 같이 통과해야 하는, 난이도가 더 높은 전투가 됩니다.
	
\end{document}