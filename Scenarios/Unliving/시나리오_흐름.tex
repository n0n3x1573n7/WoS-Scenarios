\documentclass{report}

\begin{document}
	\section{태초의 이야기(선택)}
		태초의 이야기에서 시스템은 네 이야기꾼을 불러, 서사의 기본 정보와 공개 이야기에 대해 이야기해줍니다. 여기에서 네 가지 역할, 즉 \hyperlink{cursed-bard}{성가단원}, \hyperlink{cowardly-priest}{사제}, \hyperlink{corrupt-paladin}{타락한}, \hyperlink{hurt-rogue}{이단심판관}을 부여합니다. 이 때, 네 가지 역할을 네 명이 시스템과 따로 만나서 나눠줘도 되고, 네명이 같이 시스템과 모여서 각자 고르도록 해도 상관 없습니다.
		
		이 이야기에 왜 들어가야 하는지를 
		
		개인적으로는 시스템이 네 개의 역할을 지정하는 것을 추천드립니다. \hyperlink{cowardly-priest}{사제}를 받은 이에게 흑막이 본인이라는 사실을 밝혀야 하기 때문이기도 하고, 동의를 구해야 하기 때문입니다. 반드시 동의하지 않아도 된다는 점을 강조해주세요.
		
		동의하지 않은 경우에는 "역할은 역할일 뿐"이라는 점을 명시하고, \hyperlink{alternative:no-criminal}{흑막 거부시의 대체 세계선}을 따라서 진행하시면 됩니다.
	
	\section{마을의 구조}
		마을은 마을 외곽, 마을 내부, 교회의 총 세 겹으로 구성되어 있습니다. 이야기꾼들은 마을로 들어가는 길에 서 있습니다. 사제는 마을에서 나와 이단심판관을 맞이하러 나와있었고, 그 과정에서 좀비로 변하지 않은 성가단원 꼬마와 전직 성기사를 데리고 나왔습니다.
		
		현재 시각은 해가 뉘엿뉘엿 지고 있는 저녁 6시입니다. 이야기꾼들에게 시간이 흐르고 있다는 사실을 반드시 상기시켜주세요.
		
			\subsection{마을 외곽}
				\hypertarget{search:newspaper-stand}{}
				\subsubsection*{신문 가판대}
					\begin{spoiler}[search:news]{신문}{[튜토리얼:조사대상]}
						\entry{신문을 읽는 데에 30분을 사용할 수 있습니다.}
						
						\entry[\hline\hline]{인근 마을들에 도둑이 들었다는 소식이 신문 1면에 대서특필 되어 있습니다.}
						
						\entry[\hline]{\textbf{누군가가 이단심판관(외부인)을 의심할 때 공개}: \hyperlink{search:rogue-bag}{이단심판관의 가방}을 이제 조사할 수 있습니다.}
					\end{spoiler}
				
				\hypertarget{search:rogue-bag}{}
				\subsubsection*{이단심판관의 가방}
					이 가방은 이단심판관이 항상 들고 다니는 가방입니다. 누군가가 이단심판관을 의심하기 시작할 때에야 조사할 수 있도록 해 주세요.
					
					\begin{spoiler}[search:villager-identifications]{신분 증명서 더미}{[튜토리얼:조사대상]}
						\entry[\hline]{이단 심판관 증명서와 함께, 다른 세 명의 신분 증명서 요약본이 들어 있습니다.}
					\end{spoiler}
					
					\begin{spoiler}[search:files]{서류철}{[튜토리얼:조사대상]}
						\entry{서류더미를 읽는 데에 30분을 사용할 수 있습니다.}
						
						\entry{[정화의 노래]에 관련된 정보가 정리된 서류철입니다. 이 문장이 특히 눈에 띄는군요.}
						
						\flavour[\hline]{순수한 영혼만이 이 노래를 통해 세계를 정화할 수 있다고 전해진다.}
					\end{spoiler}
					
					\begin{spoiler}[search:strange-bag]{수상한 주머니}{[튜토리얼:조사대상]}
						\entry[\hline\hline]{주머니를 열어보면 도둑들이나 들고 다닐만한 단도와 락픽이 잔뜩 들어있습니다.}
						
						\entry[\hline]{\textbf{이단심판관에게만 전달}: 이단심판관의 칭호 \storyref{rogue:hurt}{부상당한}이 제거되고, 언제든지 본인이 도둑이라는 진실을 밝히며 \storyref{rogue:dagger}{단도 투척}과 \storyref{rogue:lockpick}{자물쇠 따기}를 얻습니다.}
					\end{spoiler}
				
			\subsection{마을 내부}
				마을 내부로 들어가기 위해서는 좀비를 피해다니며 30분의 시간이 소요됩니다. 마을 내부로 들어온 경우, 마을 외곽으로 다시 돌아나가기는 힘들 것 같다는 언급을 반드시 해주세요.
				
				이단심판관을 이야기꾼들이 의심하기 시작했다면, \storyref{search:bag}{이단심판관의 가방}을 조사할 수 있다는 점을 기억하세요.
				
				\hypertarget{search:zombie}{}
				\subsubsection*{좀비}
					모든 좀비의 체력은 10으로 취급합니다.
					
					\begin{spoiler}[search:undead]{좀비}{[튜토리얼:조사대상]}
						\entry{좀비를 사로잡아 분석하는데에 30분을 소모할 수 있습니다.}
						
						\entry{
							\begin{spoiler}{저주의 산물}{[저주]}
								\entry[\hline]{[신성] 피해가 아닌 피해로 피해를 받을 수 없으며, 이 서사 밖의 이야기로 인한 [신성] 피해로는 체력을 1 이하로 깎을 수 없다.}
							\end{spoiler}
						}
						
						\entry{
							\begin{spoiler}{마을 사람}{[과거]}
								\entry[\hline]{마을 주민이었던 사제와 성가단원, 성기사는 이들의 체력을 1 이하로 깎을 수 없다. 단, 체력이 1 이하인 경우 [신성] 피해를 주면 한턴간 [기절] 상태가 되어, 이동을 포함한 모든 행동을 할 수 없다.}
							\end{spoiler}
						}
						
						\entry[\hline]{
							\begin{spoiler}[zombie:anger-aroused]{점점 격렬해지는 분노}{[저주]}
								\entry[\hline\hline]{시간이 지날수록 점점 공격성이 강해지고 있다. 원인을 제거하지 않으면 이야기꾼 본인들이 위험해질 수 있다.}
								
								\entry[\hline]{\textbf{좀비의 공격성}: 이야기꾼에게는 아래 내용을 알리지 않는다.
									
									\begin{tabularx}{\linewidth}{c|X}
										\textbf{시간} & \makecell{\centering\textbf{공격성}}\\\hline\hline
										6시\textasciitilde8시 & 선제공격을 해도 공격성이 없음 \\\hline
										8시\textasciitilde10시 & 선제공격을 하면 반격함 \\\hline
										10시\textasciitilde 자정 & 보이면 공격성을 보임 \\\hline
										자정 이후 & 이야기꾼들을 적극적으로 찾아나섬
									\end{tabularx}
								}
							\end{spoiler}
						}
					\end{spoiler}
				
				\hypertarget{search:choir-leader}{}
				\subsubsection*{성가단장의 집}
					\begin{spoiler}[search:diary]{책장}{[튜토리얼:조사대상]}
						\entry[\hline\hline]{이 책장에 있는 사용할 수 있을 만한 정보를 한 가지 얻는 데에 10분이 소요됩니다. 얻겠다고 선언한다면, 아래 정보 중 무작위 정보를 전달해주세요. 1df를 굴려, +/0/-가 나올때마다 서로 다른 정보를 주고, 중복되는 정보를 줘야 할 때 마다 이미 아는 정보라는 사실을 전달해주는 것을 추천합니다.}
						
						\entry[\hline\hline]{\textbf{일기장}: 성가단원 꼬맹이는 입양아입니다. 아기 때에 풀숲 안에 숨겨져 있던 아기를 찾아 데리고 와서 키우고 있죠.}
						
						\entry[\hline\hline]{\textbf{음악 연습 기록지}: 이 마을에 과거에 살던 천재적인 음악가가 작곡한 전설의 노래가 있다고 성기사가 당신에게 얘기해주었습니다. 최근, 성기사가 이 노래로 추정되는 곡을 어떻게인지 찾아왔군요.}
						
						\entry[\hline]{\textbf{오래된 책}: 순수한 영혼에 대한 과거의 연구 자료가 있습니다. 순수한 영혼의 피끼리는 서로 섞인다는 말이 적혀 있군요.}
					\end{spoiler}
				
				\hypertarget{search:paladin}{}
				\subsubsection*{성기사의 집}
					\storyref{search:shovel}{흙이 묻은 삽}은 성기사의 집을 나서려고 할 때 현관문 뒤에 서 있는 것을 발견하도록 하는 것을 권장합니다.
					
					\begin{spoiler}[search:paladin-shield]{낡은 방패}{[튜토리얼:조사대상]}
						\entry[\hline]{손잡이 부분에 다음이 숨겨져 있습니다:
							\begin{spoiler}{흑마법사의 피}{[타락]}
								\entry[\hline]{인접한 구역에 이 피를 흩뿌릴 수 있다. 다음 두 턴 동안, 좀비는 반드시 해당 칸을 향해서 이동하며, 공격하지 않는다.}
							\end{spoiler}
						}
					\end{spoiler}
					
					\begin{spoiler}[search:paladin-cross]{십자가}{[튜토리얼:조사대상]}
						\entry[\hline]{내부에 다음이 숨겨져 있습니다:
							\begin{spoiler}{순수한 피}{[신성]}
								\flavour[\hline]{순수한 인간의 피입니다. 과거, 성기사는 이 사람을 지키는 데에 실패했습니다.}
							\end{spoiler}
						}
					\end{spoiler}
					
					\begin{spoiler}[search:shovel]{흙이 묻은 삽}{[튜토리얼:조사대상]}
						\flavour[\hline]{젖어있는 흙이 묻은 삽입니다.}
					\end{spoiler}
				
			\subsection{교회}
				교회 주변 구역으로 들어가기 위해서는 좀비를 피해다니며 30분의 시간이 소요됩니다. 이곳으로 들어온 경우, 마을쪽으로 다시 돌아나가기는 힘들 것 같다는 언급을 반드시 해주세요.
				
				이야기꾼들은 \storyref{search:zombie}{좀비}와 \storyref{search:bag}{이단심판관의 가방}를 조사하지 않았다면 조사할 수 있습니다. 이들을 조사할 수 있다는 점을 상기시켜주세요.
				
				\hypertarget{search:choir-practice}{}
				\subsubsection*{성가단의 연습장}
					예배당 바로 옆에 붙어있는, 방음이 잘 된 독립된 공간입니다.
					
					\begin{spoiler}[search:sheet-music]{악보}{[튜토리얼:조사대상]}
						\entry{1시간을 소모하여 악보를 정리할 수 있습니다. 또는, [지식:음악] 판정을 통해 2 이상이 값이 나오는 한 사람당 10분을 절약할 수 있습니다. [지식:음악]이 0인 사람은 이 판정을 할 수 없습니다.}
						
						\entry[\hline\hline]{성가로 한번도 들어보지 못한 곡이 섞여있습니다. 성가단장의 필체로 [정화의 노래]라고 적혀 있는 곡입니다.}
						
						\entry[\hline]{성기사는 자신이 이 곡을 발견하기는 했지만, 성가단장이 악보를 현대적인 악보로 옮겼기 때문에 바로 알아보지는 못합니다.}
					\end{spoiler}
					
					\begin{spoiler}[search:organ]{소형 오르간}{[튜토리얼:조사대상]}
						\entry{성가책이 펼쳐져 있는 오르간입니다.}
						
						\entry{
							\begin{spoiler}{속도의 노래}{[노래]}
								\entry[\hline]{이 노래를 부르는 이와 같은 구역에서 출발한 이들은 1회 더 이동할 수 있다.}
							\end{spoiler}
						}
						
						\entry{
							\begin{spoiler}{변화의 노래}{[노래]}
								\entry[\hline]{이 노래를 부르는 이와 같은 구역에 있는 동안, 개연성 1을 깎고 체력과 정신력을 1씩 회복할 수 있다.}
							\end{spoiler}
						}
						
						\entry[\hline]{
							\begin{spoiler}{조화의 노래}{[노래]}
								\entry[\hline]{이 노래를 부르는 이와 같은 구역에 있는 동안, 한 턴에 한 번 공격을 회피할 수 있다.}
							\end{spoiler}
						}
					\end{spoiler}
					
					\begin{spoiler}[search:singer-identifications]{신분 증명서 더미}{[튜토리얼:조사대상]}
						\entry{성가단장과 성가단원의 신분 증명서입니다. 정독하는데에 30분을 쓸 수 있습니다.}
						
						\entry[\hline]{정독하면, 성가단장이 결혼한 적이 없으며, 성가단원이 입양되었다는 사실을 알 수 있습니다.}
					\end{spoiler}
				
				\hypertarget{search:graveyard}{}
				\subsubsection*{공동묘지}
					\begin{spoiler}[search:musician-grave]{파헤쳐진 묘지}{[튜토리얼:조사대상]}
						\entry{모든 무덤은 얌전합니다. 30분을 소모해 무덤을 관찰할 수 있습니다.}
							
						\entry[\hline]{한 무덤에 파헤쳐진 자국이 있습니다. 이 무덤의 비석에 ``내 음악이 곧 모두를 살릴 것이다"라는 엄청난 포부가 적혀있습니다.}
					\end{spoiler}
				
				\hypertarget{search:cleric-bedroom}{}
				\subsubsection*{사제의 침소}
					
					\begin{spoiler}[search:table]{탁자}{[튜토리얼:조사대상]}
						\entry{
							\begin{spoiler}[search:bible]{성경}{[튜토리얼:조사대상]}
								\entry{1시간을 소모해 성경을 정독할 수 있습니다. [지식:종교] 또는 [지식:서적] 판정을 해 4 이상의 값이 나오는 한 사람당 20분을 절약할 수 있습니다. 사제는 이 판정을 도울 수 없습니다.}
								
								\entry[\hline]{표지는 성경처럼 보이지만, 사실 악마를 숭배하는 책입니다!}
							\end{spoiler}
						}
						
						\entry[\hline]{
							\begin{spoiler}[search:bell]{종}{[튜토리얼:조사대상]}
								\entry[\hline]{종을 뒤집어 안을 보면, 종을 타종하는 금속이 열쇠라는 사실을 알 수 있습니다.}
							\end{spoiler}
						}
					\end{spoiler}
					
					\hypertarget{search:bed}{}
					\begin{spoiler}[search:bed]{침대}{[튜토리얼:조사대상]}
						\entry[\hline]{
							\begin{spoiler}[search:strange-box]{수상한 상자}{[튜토리얼:조사대상]}
								\entry{열쇠구멍이 하나 있을 뿐, 이음새도 없는 수상한 상자입니다. 안쪽에 무언가 들어있는 느낌은 납니다.}
								
								\entry{이 상자를 부숴서 열려고 하면, 수면 가스가 새어 나와 모두 잠에 빠져듭니다. 1시간 후, 다시 깨어나지만 내용물(\storyref{search:pentagram-note}{오각별이 그려진 책})은 어디론가 사라져있습니다.}
								
								\entry{상자를 여는 방법은 세 가지가 있습니다:
									\begin{enumerate}
										\item \storyref{search:bell}{종} 안에 있는 열쇠로 열기.
										\item 사제의 피를 한 방울 열쇠 구멍 안에 넣기.
										\item \storyref{rogue:lockpick}{자물쇠 따기}로 열기.
									\end{enumerate}
									이 이외의 방법으로 열고자 시도하는 것은 통하지 않습니다.
								}
								
								\entry[\hline]{
									\begin{spoiler}[search:pentagram-note]{오각별이 그려진 책}{[튜토리얼:조사대상]}
										\entry[\hline]{[타락의 노래]라는 제목의 악보입니다. 엄청나게 두꺼워, 연주하는데에 한참 걸릴 것 같군요.}
									\end{spoiler}
								}
							\end{spoiler}}
					\end{spoiler}
				
				\subsubsection*{예배당}
					예배당 안에서는 수상한 음악소리가 흘러나옵니다. 또한, 안쪽에는 좀비가 너무 많아, 들어가면 좀비를 어떻게든 처리하기 전에는 돌아나오기 힘들것 같다고 강조해주세요.
					
					들어가면 최종 전투가 시작됩니다.
	
	\pagebreak
	\section{예배당}
		최종 전투가 시작됩니다. \hyperlink{alternative:no-criminal}{사제 역할의 소유자가 흑막 역할을 거부한 경우}에는 아래 내용을 기본으로 하여, 바뀌는 부분에 대해서는 해당 부분을 따릅니다. 이 부분에서는 사제 역할의 소유자가 흑막 역할을 수용한 경우를 가정합니다.
		
		전투의 순서는 기민과 상관 없이 이단심판관(도적) $\rightarrow$ 성기사 $\rightarrow$ 사제 $\rightarrow$ 좀비 $\rightarrow$ 성가단원의 순서로 고정됩니다.
		
		예배당의 구조는 다음과 같습니다.
		
		\begin{center}
			\begin{tabular}{p{1.5cm}p{1.5cm}p{1.5cm}|p{1.5cm}|p{1.5cm}|p{1.5cm}p{1.5cm}p{1.5cm}}
				\cline{4-5}
				&                         &   & \multicolumn{2}{c|}{제대} &                       &                       &                       \\ \hline
				\multicolumn{1}{|c|}{} & \multicolumn{2}{c|}{좌측 오르간} &            &            & \multicolumn{2}{c|}{우측 오르간}                   & \multicolumn{1}{c|}{} \\ \hline
				\multicolumn{1}{|c|}{} & \multicolumn{2}{c|}{의자 1}   &            &            & \multicolumn{2}{c|}{의자 2}                     & \multicolumn{1}{c|}{} \\ \hline
				\multicolumn{1}{|c|}{} & \multicolumn{2}{c|}{의자 3}   &            &            & \multicolumn{2}{c|}{의자 4}                     & \multicolumn{1}{c|}{} \\ \hline
				\multicolumn{1}{|c|}{} & \multicolumn{2}{c|}{의자 5}   &            &            & \multicolumn{2}{c|}{의자 6}                     & \multicolumn{1}{c|}{} \\ \hline
				\multicolumn{1}{|c|}{} & \multicolumn{1}{c|}{}   &   & \makecell{\centering 입}          & \makecell{\centering 구}          & \multicolumn{1}{c|}{} & \multicolumn{1}{c|}{} & \multicolumn{1}{c|}{} \\ \hline
			\end{tabular}
		\end{center}
		
		\begin{spoiler}{밀어내기}{[행위:이야기꾼]}
			\entry[\hline]{[기절] 상태인 좀비를 원하는 방향으로 한 칸 밀어낼 수 있다.}
		\end{spoiler}
		
		\begin{spoiler}{제대}{[신성]}
			\entry[\hline]{제대 구역으로 밀려난 좀비는 즉시 무력화되어, 아무 행동도 할 수 없다.}
		\end{spoiler}
		
		\begin{spoiler}{의자}{[물체]}
			\entry[\hline]{마을 주민들이 빽빽하게 앉아있는 의자. 좀비는 이 의자를 마음대로 드나들 수 있지만, 이야기꾼은 드나들 수 없다.}
		\end{spoiler}
		
		\begin{spoiler}{오르간}{[물체][음악]}
			\entry[\hline]{오르간에서 연주된 음악의 효과는 예배당 전체에 영향을 끼친다.}
		\end{spoiler}
		
		사제는 예배당에 들어온 순간, 본색을 드러내며 충격에 빠진 이야기꾼들을 제치고 좌측 오르간으로 가 앉아, [타락의 노래]를 연주합니다(\storyref{priest:fallen-song}{타락의 연주자}).
		
		이야기꾼들은 입구칸 중 원하는 칸에서 시작합니다.
		
		좀비는 최초에는 각 의자별로 하나씩 일어서서 공격 준비 태세를 하고 있습니다.
		
		\begin{spoiler}[zombie:fallen]{타락의 손길}{[좀비]}
			\entry{좀비의 턴에, 좀비는 가장 가까이에 있는 이야기꾼을 향해 한 칸 이동한다.}
			
			\entry{좀비와 같은 칸에서 턴을 끝내는 이야기꾼은 좀비 하나당 체력, 정신력, 또는 개연성을 1 잃는다.}
			
			\entry[\hline]{같은 칸에 있는 좀비의 수가 이야기꾼의 [근력] 수치의 두 배 이상인 경우, 이야기꾼은 이동할 수 없다.}
		\end{spoiler}
		
		이 상황을 타개할 수 있는 방법은 두 가지입니다:
		\begin{enumerate}
			\item 사제를 죽음에 빠트리기
				\subitem{$\rightarrow$} 사제를 죽이면 모든 마을 주민은 가사상태에 빠지지만, 살아는 있습니다. 이 중 일부는 다시 살아갈 것입니다.
			\item{} [정화의 노래]를 오른쪽 오르간에서 세 턴간 연주하기
				\subitem{$\rightarrow$} [정화의 노래]는 [순수한 영혼]을 가진 성가대원이 연주해야만 효과를 발휘합니다. [정화의 노래]를 연주하는 동안, 연주자는 \storyref{zombie:fallen}{타락의 손길}으로 인한 피해량을 무시합니다. 세 턴이 완료되는 순간, 모든 마을 주민은 좀비 상태에서 다시 인간 상태로 되돌아오게 됩니다.
		\end{enumerate}
		
		이가 일어나기 전에, 사제를 제외한 모든 이야기꾼의 개연성이 0으로 떨어진다면 이 이야기 속의 인류는 모두 좀비가 되어 멸망하게 됩니다.
	
	\section{다시, 태초의 이야기(선택)}
		이야기꾼들은 태초의 이야기에서 서로가 가진, 하지만 말할 수 없었던 정보로 진상을 다시 한번 짚어가며 마무리합니다.
	
\end{document}