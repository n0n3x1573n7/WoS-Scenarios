\documentclass{report}

\begin{document}
	\section{태초의 이야기(선택)}
		태초의 이야기에서 시스템은 네 이야기꾼을 불러, 서사의 기본 정보와 공개 이야기에 대해 이야기해줍니다. 여기에서 네 가지 역할, 즉 \hyperlink{cursed-bard}{성가단원}, \hyperlink{cowardly-priest}{사제}, \hyperlink{corrupt-paladin}{타락한}, \hyperlink{hurt-rogue}{이단심판관}을 부여합니다. 이 때, 네 가지 역할을 네 명이 시스템과 따로 만나서 나눠줘도 되고, 네명이 같이 시스템과 모여서 각자 고르도록 해도 상관 없습니다.
		
		이 이야기에 왜 들어가야 하는지를 
		
		개인적으로는 시스템이 네 개의 역할을 지정하는 것을 추천드립니다. \hyperlink{cowardly-priest}{사제}를 받은 이에게 흑막이 본인이라는 사실을 밝혀야 하기 때문이기도 하고, 동의를 구해야 하기 때문입니다. 반드시 동의하지 않아도 된다는 점을 강조해주세요.
		
		동의하지 않은 경우에는 "역할은 역할일 뿐"이라는 점을 명시하고, \hyperlink{alternative:no-criminal}{흑막 거부시의 대체 세계선}을 따라서 진행하시면 됩니다.
	
	\section{마을의 구조}
		마을은 마을 외곽, 마을 내부, 교회의 총 세 겹으로 구성되어 있습니다. 이야기꾼들은 마을로 들어가는 길에 서 있습니다. 사제는 마을에서 나와 이단심판관을 맞이하러 나와있었고, 그 과정에서 좀비로 변하지 않은 성가단원 꼬마와 전직 성기사를 데리고 나왔습니다.
		
		현재 시각은 해가 뉘엿뉘엿 지고 있는 저녁 6시입니다. 이야기꾼들에게 시간이 흐르고 있다는 사실을 반드시 상기시켜주세요.
		
			\subsection{마을 외곽}
				\hypertarget{search:newspaper-stand}{}
				\subsubsection*{신문 가판대}
					\begin{spoiler}[search:news]{신문}{[튜토리얼:조사대상]}
						\entry[\hline\hline]{인근 마을들에 도둑이 들었다는 소식이 신문 1면에 대서특필 되어 있습니다.}
						
						\entry[\hline]{\textbf{누군가가 이단심판관(외부인)을 의심할 때 공개}: \hyperlink{search:rogue-bag}{이단심판관의 가방}을 이제 조사할 수 있습니다.}
					\end{spoiler}
				
				\hypertarget{search:rogue-bag}{}
				\subsubsection*{이단심판관의 가방}
					이 가방은 이단심판관이 항상 들고 다니는 가방입니다. 누군가가 이단심판관을 의심하기 시작할 때에야 조사할 수 있도록 해 주세요.
					
					\begin{spoiler}[search:villager-identifications]{신분 증명서 더미}{[튜토리얼:조사대상]}
						\entry[\hline]{이단 심판관 증명서와 함께, 다른 세 명의 신분 증명서 요약본이 들어 있습니다.}
					\end{spoiler}
					
					\begin{spoiler}[search:files]{서류철}{[튜토리얼:조사대상]}
						\entry{[정화의 노래]에 관련된 정보가 정리된 서류철입니다. 이 문장이 특히 눈에 띄는군요.}
						
						\flavour[\hline]{순수한 영혼만이 이 노래를 통해 세계를 정화할 수 있다고 전해진다.}
					\end{spoiler}
					
					\begin{spoiler}[search:strange-bag]{수상한 주머니}{[튜토리얼:조사대상]}
						\entry[\hline\hline]{주머니를 열어보면 도둑들이나 들고 다닐만한 단도와 락픽이 잔뜩 들어있습니다.}
						
						\entry[\hline]{\textbf{이단심판관에게만 전달}: 이단심판관의 칭호 \storyref{rogue:hurt}{부상당한}이 제거되고, 언제든지 본인이 도둑이라는 진실을 밝히며 \storyref{rogue:dagger}{단도 투척}과 \storyref{rogue:lockpick}{자물쇠 따기}를 얻습니다.}
					\end{spoiler}
				
			\subsection{마을 내부}
				마을 내부로 들어가기 위해서는 좀비를 피해다니며 30분의 시간이 소요됩니다. 마을 내부로 들어온 경우, 마을 외곽으로 다시 돌아나가기는 힘들 것 같다는 언급을 반드시 해주세요.
				
				이단심판관을 이야기꾼들이 의심하기 시작했다면, \storyref{search:bag}{이단심판관의 가방}을 조사할 수 있다는 점을 기억하세요.
				
				\hypertarget{search:zombie}{}
				\subsubsection*{좀비}
					모든 좀비의 체력은 10으로 취급합니다.
					
					\begin{spoiler}[search:]{좀비}{[튜토리얼:조사대상]}
						\entry{
							\begin{spoiler}{저주의 산물}{[저주]}
								\entry[\hline]{[신성] 피해가 아닌 피해로 피해를 받을 수 없으며, 이 서사 밖의 이야기로 인한 [신성] 피해로는 체력을 1 이하로 깎을 수 없다.}
							\end{spoiler}
						}
						
						\entry{
							\begin{spoiler}{마을 사람}{[과거]}
								\entry[\hline]{마을 주민이었던 사제와 성가단원, 성기사는 이들의 체력을 1 이하로 깎을 수 없다. 단, 체력이 1 이하인 경우 [신성] 피해를 주면 한턴간 [기절] 상태가 되어, 이동을 포함한 모든 행동을 할 수 없다.}
							\end{spoiler}
						}
						
						\entry[\hline]{
							\begin{spoiler}[zombie:anger-aroused]{점점 격렬해지는 분노}{[저주]}
								\entry[\hline\hline]{시간이 지날수록 점점 공격성이 강해지고 있다. 원인을 제거하지 않으면 이야기꾼 본인들이 위험해질 수 있다.}
								
								\entry[\hline]{\textbf{좀비의 공격성}: 이야기꾼에게는 아래 내용을 알리지 않는다.
									
									\begin{tabularx}{\linewidth}{c|X}
										\textbf{시간} & \makecell{\centering\textbf{공격성}}\\\hline\hline
										6시\textasciitilde8시 & 선제공격을 해도 공격성이 없음 \\\hline
										8시\textasciitilde10시 & 선제공격을 하면 반격함 \\\hline
										10시\textasciitilde 자정 & 보이면 공격성을 보임 \\\hline
										자정 이후 & 이야기꾼들을 적극적으로 찾아나섬
									\end{tabularx}
								}
							\end{spoiler}
						}
					\end{spoiler}
				
				\hypertarget{search:choir-leader}{}
				\subsubsection*{성가단장의 집}
					\begin{spoiler}[search:diary]{책장}{[튜토리얼:조사대상]}
						\entry[\hline\hline]{이 책장에 있는 사용할 수 있을 만한 정보를 한 가지 얻는 데에 10분이 소요됩니다. 얻겠다고 선언한다면, 아래 정보 중 무작위 정보를 전달해주세요. 1df를 굴려, +/0/-가 나올때마다 서로 다른 정보를 주고, 중복되는 정보를 줘야 할 때 마다 이미 아는 정보라는 사실을 전달해주는 것을 추천합니다.}
						
						\entry[\hline\hline]{\textbf{일기장}: 성가단원 꼬맹이는 입양아입니다. 아기 때에 풀숲 안에 숨겨져 있던 아기를 찾아 데리고 와서 키우고 있죠.}
						
						\entry[\hline\hline]{\textbf{음악 연습 기록지}: 이 마을에 과거에 살던 천재적인 음악가가 작곡한 전설의 노래가 있다고 성기사가 당신에게 얘기해주었습니다. 최근, 성기사가 이 노래로 추정되는 곡을 어떻게인지 찾아왔군요.}
						
						\entry[\hline]{\textbf{오래된 책}: 순수한 영혼에 대한 과거의 연구 자료가 있습니다. 순수한 영혼의 피끼리는 서로 섞인다는 말이 적혀 있군요.}
					\end{spoiler}
				
				\hypertarget{search:paladin}{}
				\subsubsection*{성기사의 집}
					\storyref{search:shovel}{흙이 묻은 삽}은 성기사의 집을 나서려고 할 때 현관문 뒤에 서 있는 것을 발견하도록 하는 것을 권장합니다.
					
					\begin{spoiler}[search:paladin-shield]{낡은 방패}{[튜토리얼:조사대상]}
						\entry[\hline]{손잡이 부분에 다음이 숨겨져 있습니다:
							\begin{spoiler}{흑마법사의 피}{[타락]}
								\entry[\hline]{인접한 구역에 이 피를 흩뿌릴 수 있다. 다음 두 턴 동안, 좀비는 반드시 해당 칸을 향해서 이동하며, 공격하지 않는다.}
							\end{spoiler}
						}
					\end{spoiler}
					
					\begin{spoiler}[search:paladin-cross]{십자가}{[튜토리얼:조사대상]}
						\entry[\hline]{내부에 다음이 숨겨져 있습니다:
							\begin{spoiler}{순수한 피}{[신성]}
								\flavour[\hline]{순수한 인간의 피입니다. 과거, 성기사는 이 사람을 지키는 데에 실패했습니다.}
							\end{spoiler}
						}
					\end{spoiler}
					
					\begin{spoiler}[search:shovel]{흙이 묻은 삽}{[튜토리얼:조사대상]}
						\flavour[\hline]{젖어있는 흙이 묻은 삽입니다.}
					\end{spoiler}
				
			\subsection{교회}
				교회 주변 구역으로 들어가기 위해서는 좀비를 피해다니며 30분의 시간이 소요됩니다. 이곳으로 들어온 경우, 마을쪽으로 다시 돌아나가기는 힘들 것 같다는 언급을 반드시 해주세요.
				
				이야기꾼들은 \storyref{search:zombie}{좀비}와 \storyref{search:bag}{이단심판관의 가방}를 조사하지 않았다면 조사할 수 있습니다. 이들을 조사할 수 있다는 점을 상기시켜주세요.
				
				\hypertarget{search:choir-practice}{}
				\subsubsection*{성가단의 연습장}
					예배당 바로 옆에 붙어있는, 방음이 잘 된 독립된 공간입니다.
					
					\begin{spoiler}[search:sheet-music]{악보}{[튜토리얼:조사대상]}
						\entry[\hline]{content...}
					\end{spoiler}
					
					\begin{spoiler}[search:organ]{소형 오르간}{[튜토리얼:조사대상]}
						\entry[\hline]{content...}
					\end{spoiler}
					
					\begin{spoiler}[search:singer-identifications]{신분 증명서 더미}{[튜토리얼:조사대상]}
						\entry[\hline]{content...}
					\end{spoiler}
				
				\hypertarget{search:graveyard}{}
				\subsubsection*{공동묘지}
					\begin{spoiler}[search:musician-grave]{파헤쳐진 묘지}{[튜토리얼:조사대상]}
						\entry[\hline]{content...}
					\end{spoiler}
				
				\hypertarget{search:cleric-bedroom}{}
				\subsubsection*{사제의 침소}
					
					\begin{spoiler}[search:table]{탁자}{[튜토리얼:조사대상]}
						\entry{
							\begin{spoiler}[search:bible]{성경}{[튜토리얼:조사대상]}
								\entry[\hline]{content...}
							\end{spoiler}
						}
						
						\entry[\hline]{
							\begin{spoiler}[search:bell]{종}{[튜토리얼:조사대상]}
								\entry[\hline]{content...}
							\end{spoiler}
						}
					\end{spoiler}
					
					\hypertarget{search:bed}{}
					\begin{spoiler}[search:bed]{침대}{[튜토리얼:조사대상]}
						\entry[\hline]{
							\begin{spoiler}[search:strange-box]{수상한 상자}{[튜토리얼:조사대상]}
								\entry{
									\begin{spoiler}[search:skull]{해골}{[튜토리얼:조사대상]}
										\entry[\hline]{content...}
									\end{spoiler}
								}
								
								\entry[\hline]{
									\begin{spoiler}[search:pentagram-note]{오각별이 그려진 책}{[튜토리얼:조사대상]}
										\entry[\hline]{content...}
									\end{spoiler}
								}
							\end{spoiler}}
					\end{spoiler}
				
				\subsubsection*{예배당}
					
	\section{예배당}
	
	\section{다시, 태초의 이야기(선택)}
	
\end{document}