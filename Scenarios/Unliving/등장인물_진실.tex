\documentclass{report}

\begin{document}
	개별 이야기꾼들이 이미 알고 있는 정보를 전달하고, 얻을 수 있는 이야기들과 획득을 위한 필요 조건, 그리고 칭호의 효과와 칭호의 \textbf{공개 조건}과 \textbf{제거 조건}을 명백하게 밝혀야 합니다. 이 이야기들은 페이지 단위로 잘려 있어, 정보를 제공할 때 페이지 단위로 제공할 수 있게 했습니다.
	
	직업 퀘스트는 시스템이 내린 것으로, 이야기꾼들을 돕기 위해 시스템이 이야기에 간섭할 수 있는 최대한입니다.
	
	\pagebreak \hypertarget{cursed-bard}{}
	\section{저주받은 성가단원}
		\documentclass{report}

\begin{document}
	\subsection*{알고 있는 정보}
		당신은 사람들이 처음으로 언데드로 변하는 것을 본 꼬마아이입니다. 당신의 보호자인 성가단장 역시 좀비로 변해버렸고요.
		
		최근 들어 성가단의 단장이자 당신에게 오르간을 연주하는 법을 가르쳐준 당신의 보호자는 당신이 잘 때 밤늦게 어딘가로 향하는 일이 잦아졌습니다. 공책을 들고가는거로 봐서는 뭔가 적을것이 있는 것 같은데, 그게 무엇일까요?
	
	\subsection*{가지고 시작하는 이야기}
		\begin{story}[choir:bard]{성가대}{[역할]}
			\entry{노래를 부르거나 악기를 연주하여 음악을 통해 마법을 사용할 수 있다. 아래 노래들을 사용할 수 있다.}
			
			\entry{
				\begin{story}[music:holy]{성가}{[노래]}
					\entry[\hline]{한 턴에 이 노래를 부르기로 선택한다면 이동을 제외한 다른 행동을 할 수 없다. 턴이 종료될 때, 같은 구역에 있는 모든 생명체의 체력을 1 회복하고, 언데드에게 [신성] 피해를 1 준다.}
				\end{story}
			}
			
			\entry[\hline]{\statchange{+}{지식:음악, 제작:음악}}
		\end{story}
	
	\subsection*{획득 가능한 이야기}
		\begin{story}[bard:cursed]{저주받은}{[공포][칭호]}
			\pre{언데드에게 효과가 적용되도록 \storyref{choir:bard}{성가대}의 노래를 부른다.}
			
			\limitedtrauma{공포}{\storyref{choir:bard}{성가대}의 노래들의 효과가 언데드를 상대로는 적용되지 않는다.}
			
			\entry[\hline]{\textbf{제거 조건}: 특정 아이템을 소유하고 있는 동안, 또는 어떤 사실을 알게 되면 이 칭호를 무시한다. 이 시점은 시스템이 알려준다.}
		\end{story}
	
	\subsection*{직업 퀘스트}
		\begin{story}{성가단원}{[직업]}
			\entry{\statchange{+}{지식:음악, 제작:음악}}
			
			\entry{\begin{story}[bard:blessed]{축복}{[퀘스트]}
				
				\entry{\textbf{성공 조건}
				
				칭호 \storyref{bard:cursed}{저주받은} 제거}
				
				\entry[\hline]{\textbf{보상}
				
				\statchange{+}{지식:신성}
				
				\storyref{music:holy}{성가}의 회복량이 1 증가하고, 언데드에게 [신성] 피해를 주는 대신 인접한 구역으로 밀쳐내며 [기절: 1턴] 상태에 빠트릴 수 있다.}
			
			\end{story}}
			
			\entry[\hline]{\begin{story}{음악}{[퀘스트]}
					\entry{\textbf{성공 조건}
					
					성가단의 연습 장소에 있는 악보를 정리하여, 음악을 연주한다.}
					
					\entry[\hline]{\textbf{보상}
					
					연주한 음악을 \storyref{music:holy}{성가}와 더불어 기억할 수 있다.
					
					그 이외의 곡들은 악보를 직접 소유하고 있다면 연주하거나 노래할 수 있다.}
				
			\end{story}}
		\end{story}
\end{document}%
	
	\pagebreak \hypertarget{corrupt-paladin}{}
	\section{타락한 성기사}
		\documentclass{report}

\begin{document}
	\subsection*{알고 있는 정보}
		이 마을에 오래전에 살고 있던 사람 중에는 전설적인 작곡가이자 사제가 있었습니다. 마을의 오르간 연주자의 부탁을 받아 그 사람의 유작을 찾아나선 당신은 그의 무덤을 파헤쳤고, 그의 관에 새겨져있던 노래를 하나 찾게 되었습니다. 좀비 사태가 발생했을 때, 당신은 무덤의 뒤처리를 하고 있었고요. 당신은 이로 인해 당신이 타락하게 되었다는 사실을 알고 있기 때문에, 다른 등장인물들이 무덤을 발견하기 전에는 이 사실을 최대한 숨기고자 합니다.
		
		당신의 집에는 이 무덤을 파헤쳐서 젖은 흙이 묻은 삽이 현관문 뒤에 있고, 당신이 사용했던 방패와 십자가에는 각각 [흑마법사의 피]와 [순수한 피]가 숨겨져 있습니다. [순수한 피]는 수년 전, 당신의 보호 하에 있었지만 지키지 못했던 첫 번째 이의 것으로, [순수한 피] 끼리는 잘 섞인다는 성질을 가지고 있기에 이를 이용해 숨겨두었던 자신의 아기를 찾고, 아기를 보호해달라는 부탁을 받았습니다.
	
	\subsection*{가지고 시작하는 이야기}
		\begin{spoiler}[paladin:fallen]{타락한}{[공포][칭호]}
			\limitedtrauma{공포}{\storyref{paladin:smite}{신성한 일격}을 사용할 수 없다.}
			
			\entry[\hline]{\textbf{제거 조건}: 자신의 집 안에 있는 [순수한 피]를 마시거나, 특정한 음악을 듣는다.}
		\end{spoiler}
		
		\begin{spoiler}[paladin:smite]{신성한 일격}{[역할]}
			\entry[\hline]{사거리에 상관 없이 대상을 정한다. 대상에게 [신성] 피해를 2 주고, [기절] 상태에 빠트린다. [기절] 상태의 상대는 다음 턴 모든 행동이 불가능하다.}
		\end{spoiler}
	
	\subsection*{직업 퀘스트}
		\begin{spoiler}{성기사}{[직업]}
			\entry{\statchange{+}{근력, 의지}}
			
			\entry{\begin{spoiler}{순수한 지식}{[퀘스트]}
						
						\entry{\textbf{성공 조건}
							
						[순수한 피]를 얻는다.}
						
						\entry[\hline]{\textbf{보상}
							
						\statchange{+}{지식:신성[2]}}
						
			\end{spoiler}}
			
			\entry[\hline]{\begin{spoiler}{타락한 지식}{[퀘스트]}
						
						\entry{\textbf{성공 조건}
							
						[흑마법사의 피]를 얻는다.}
						
						\entry[\hline]{\textbf{보상}
							
						\statchange{+}{지식:흑마법[2]}}
						
			\end{spoiler}}
		\end{spoiler}
	
	\subsection*{획득 가능한 이야기}
		\begin{spoiler}{흑마법사의 피}{[타락]}
			\entry[\hline]{인접한 구역에 이 피를 흩뿌릴 수 있다. 다음 두 턴 동안, 좀비는 반드시 해당 칸을 향해서 이동하며, 이야기꾼들을 공격하지 않는다.}
		\end{spoiler}
		
		\begin{spoiler}[paladin:protect]{권능의 보호막}{[신성]}
			\pre{\storyref{paladin:fallen}{타락한} 칭호 제거, \storyref{search:paladin-shield}{낡은 방패} 장착}
			
			\entry[\hline]{세 턴에 한 번 사용할 수 있다. 다음 자신의 턴까지, 타락한 존재들은 드나들 수 없는 방벽을 자신이 있는 구역에 칠 수 있다. 해당 칸에 있는 타락한 존재는 무작위 인접한 구역으로 밀려난다.}
		\end{spoiler}
\end{document}%
	
	\pagebreak \hypertarget{cowardly-priest}{}
	\section{겁에 질린 사제}
		\documentclass{report}

\begin{document}
	\subsection*{알고 있는 정보}
		당신이 이 좀비 사태의 원흉입니다. 당신은 사실 악마를 숭배하는 이교도이며, 이 악마를 숭배하고 소환하기 위해 악마의 가르침을 담은 책을 성경 표지만 덧씌워두었습니다. 당신은 이런 노력을 통해 악마와 계약해 [타락의 노래]를 얻었고, 이를 이용해 오르간 연주자부터 시작하여 사람들을 좀비로 바꾸었습니다.
		
		당신의 방 안의 침대 밑에는 당신의 피를 이용하거나, 종 안에 숨겨진 열쇠를 사용해서만 안에 들어있는 해골과 악마에게서 받은 [타락의 노래]를 발견할 수 있다는 사실을 알고 있습니다. 상자가 부서지면 수면가스가 나오도록 되어 있어 어느 정도의 보호조치를 해 두었습니다.
		
		이 모든 이야기는 예배당에 진입하기 이전까지 자의적으로 공개할 수 없습니다.
	
	\subsection*{가지고 시작하는 이야기}
		\begin{spoiler}{겁에 질린}{[공포][칭호]}
			\limitedtrauma{공포}{\storyref{cleric:prayer}{기도}로 언데드에게 피해를 줄 수 없다.}
			
			\entry{\textbf{공개조건}: 공개 불가.}
			
			\entry[\hline]{\textbf{제거 조건}: 예배당에 진입한다.}
		\end{spoiler}
		
		\begin{spoiler}[cleric:prayer]{기도}{[역할]}
			\entry[\hline]{사거리에 상관 없이 대상을 정한다. 대상의 다음 턴이 시작될 때, 생명체인 대상의 체력을 2 회복시키거나, 언데드인 대상에게 [신성] 피해를 2 준다.}
		\end{spoiler}
	
	\subsection*{직업 퀘스트}
		사제의 직업 퀘스트는 이야기의 의지가 내린 것이 아니고, 악마로서 강림하여 이야기를 오염시켜 자신의 것으로 만든 [타락한 자]가 내린 것입니다.
		
		\begin{spoiler}{대악마의 사제}{[직업]}
			\entry{\statchange{+}{지식:신성[2], 지식:흑마법[2]}}
			
			\entry[\hline]{\begin{spoiler}{비밀 숭배}{[퀘스트]}
						
						\entry{\textbf{성공 조건}
							
							예배당 돌입 전까지, 악마 숭배 사실을 들키지 않는다.
						}
						
						\entry[\hline]{\textbf{보상}
							
							오르간을 연주하는 동안, 매 턴 개연성을 1d6 회복한다.
							
							흑막을 거부했다면, \storyref{cleric:prayer}{기도}의 이름이 [절박한 기도]로 바뀌며, 이를 통해 이제 아무 대상의 개연성을 2 회복시키거나, [신성] 피해를 2 주거나, [타락] 피해를 2 줄 수 있다.
						}
						
			\end{spoiler}}
		\end{spoiler}
	
	\subsection*{획득 가능한 이야기}
		악마 숭배 사실을 들키면, \storyref{cleric:prayer}{기도}의 이름이 [저주]로 바뀌고, 이를 포함한 언데드와 생명체에 서로 다른 효과를 주는 능력 모두의 언데드에 대한 효과와 생명체에 대한 효과가 뒤바뀌며, 언데드에게 [신성] 피해를 주는 기술의 경우 생명체에게 [타락] 피해를 줍니다\footnote{[저주]의 경우, 언데드의 체력을 2 회복하거나, 생명체인 대상에게 [타락] 피해를 2 줍니다.}.
		
		또한, 흑막을 거부하지 않았다면 예배당에 진입하면 자동으로 왼쪽의 오르간에 앉으며, \storyref{dark-age}{어둠의 시대}의 효과를 전부 무시하고, 다음 이야기를 얻습니다:
		\begin{spoiler}[priest:fallen-song]{타락의 연주자}{[타락의 노래:2악장]}
			\entry{자신에게 적용되는 특정 이야기의 효과가 생명체와 언데드에게 서로 다른 효과를 발휘한다면, 어느 쪽을 따를지를 선택할 수 있다.}
			
			\entry{자신의 턴이 진행되는 중, 방어적인 행동 또는 이동만 한 채로 오르간 앞에 있다면 다음 자신의 턴까지 [타락의 노래]를 연주하기로 선택할 수 있다. 이를 연주한다면 자신의 턴을 즉시 종료하지만, d6을 굴려, 숫자에 해당하는 의자에서 좀비가 하나 추가로 소환된다.}
			
			\entry{턴 종료시 오르간 앞에서 벗어나있다면 개연성에 피해를 1 받는다.}
			
			\entry[\hline]{체력, 정신력, 개연성이 모두 개연성으로 통합된다.}
		\end{spoiler}
\end{document}%
	
	\pagebreak \hypertarget{hurt-rogue}{}
	\section{부상당한 이단심판관}
		\documentclass{report}

\begin{document}
	\subsection*{알고 있는 정보}
		당신은 도둑입니다. 여러 마을을 돌아다니며 이단심판관인척 하며 그들의 재산을 훔쳤죠.
		
		이번에 당신이 노리는 것은 어떤 작곡가의 유작입니다. [정화의 노래]로 알려진 이 노래는 어느샌가 기억에서 사라졌지만, 이 마을 출신이었고 여기에 묻히기까지 했으니 흔적은 남아 있겠죠. 이 노래에 대해 조사한 결과, "순수한 피"를 가진 이가 연주해야만 효과가 있다고 하는데, 마침 이 마을에 "순수한 피"를 가진 이와 접촉한 성기사가 있다는 사실을 이전에 있던 마을의 흑마술사로부터 알아냈습니다. 이 마을에서 성기사의 도움을 얻고자 마을의 수장이나 다름없는 사제에게 도움을 청하기 위해, 이 두 명에 대한 뒷조사를 간략하게나마 하고 왔습니다. 물론, 불법이니만큼 들키기 전에는 말할 생각은 없지만요.
		
		당신의 가방 안에는 이 뒷조사 자료와 [정화의 노래]에 대한 자료, 그리고 당신이 사용하는 락픽과 단도, 만능툴들이 무기 주머니에 들어있습니다. 다른 사람들이 당신을 알아보지 못하기를, 그리고 이 가방을 열지 않기를 바라는수밖에는 없겠군요.
	
	\subsection*{가지고 시작하는 이야기}
		\begin{spoiler}[rogue:hurt]{부상당한}{[공포][칭호]}
			\limitedtrauma{공포}{언데드를 대상으로 한 무기와 \storyref{rogue:judgment}{심판}의 사용이 불가능하다.}
			
			\entry[\hline]{\textbf{제거 조건}: 자신의 진정한 정체를 다른 사람들이 추궁한다. 이 제거 조건은 공개할 수 없다.}
		\end{spoiler}
		
		\begin{spoiler}[rogue:judgment]{심판}{[역할]}
			\entry{\storyref{rogue:hurt}{부상당한}이 제거되면, 이 이야기를 잃는다. 이 조건은 공개할 수 없다.}
			
			\entry[\hline]{씬이 시작할 때, [심판]의 대상을 하나 정한다. 해당 대상에게 주는 모든 피해가 1 증가한다.}
		\end{spoiler}
	
	\subsection*{획득 가능한 이야기}
	\begin{spoiler}[rogue:dagger]{단도 투척}{[역할]}
		\pre{칭호 \storyref{rogue:hurt}{부상당한} 제거}
		
		\entry[\hline]{매 턴 한 번, 단도를 던질 수 있다. 단도는 떨어진 구역당 피해 1을 준다.}
	\end{spoiler}
	
	\begin{spoiler}[rogue:lockpick]{자물쇠 따기}{[역할]}
		\pre{칭호 \storyref{rogue:hurt}{부상당한} 제거}
		
		\entry[\hline]{잠겨있는 물체를 30분을 소모하여 열 수 있습니다.}
	\end{spoiler}
	
	\subsection*{직업 퀘스트}
		\begin{spoiler}{이단심판관}{[직업]}
			\entry{\statchange{+}{기민[2]}}
			
			\entry{\begin{spoiler}{비밀의 수호자}{[퀘스트]}
						
						\entry{\textbf{성공 조건}
							\storyref{rogue:hurt}{부상당한}이 마을 내부에서 교회 구역으로 들어가기 전까지 제거당하지 않는다.
						}
						
						\entry[\hline]{\textbf{보상}
							\storyref{rogue:judgment}{심판}을 \storyref{rogue:hurt}{부상당한}을 잃더라도 사용할 수 있으나, 이름이 [쇠약의 독]으로 바뀐다.
						}
						
			\end{spoiler}}
			
			\entry[\hline]{\begin{spoiler}{영원한 비밀}{[퀘스트]}
					
					\entry{\textbf{성공 조건}
						\storyref{rogue:hurt}{부상당한}을 교회 내부로 들어갈 때 까지 제거당하지 않는다.
					}
					
					\entry[\hline]{\textbf{보상}
						\storyref{rogue:judgment}{심판}의 이름이 [이단의 심판]으로 바뀌며, 피해 증가량이 피해량의 50\%(최소 1)로 증가한다.
					}
					
			\end{spoiler}}
		\end{spoiler}
\end{document}%
		
	\pagebreak
	\section{고의로 누락된 정보}
		\documentclass{report}

\begin{document}
	아래는 각 캐릭터의 정보에서 고의적으로 누락된 부분에 대한 추가적 설명입니다.
	
	\subsection*{\large \hyperlink{cursed-bard}{저주받은 성가단원}}
		\subsubsection*{\normalsize \storyref{bard:cursed}{저주받은}의 제거조건: ``특정 아이템을 소유", ``어떤 사실을 알게 되면"}
			$\Rightarrow$ ``특정 아이템"은 성기사의 집에 있는 [순수한 피]이고, ``어떤 사실"은 자신이 ``순수한 영혼"이라는 사실입니다. [순수한 피]를 얻은 경우에는 반드시 본인에게만 해당 사실을 알리나, ``순수한 영혼"으로 인한 해제는 해당 사실을 파티 전원이 알게 되고, \storyref{bard:cursed}{저주받은}이 공개된 상태라면 전체 공개하지만, 본인만 알게 되거나 \storyref{bard:cursed}{저주받은}이 공개되지 않은 상태더라도 개인적으로 이 사실을 전달합니다.
		
		\subsubsection*{\normalsize 자신이 ``순수한 영혼"임을 확실히 알게 되었을 때}
			$\Rightarrow$ 성기사의 [순수한 피]를 사용해 ``순수한 영혼"을 감별하고자 한다면, 감별 대상자의 체력 1을 감하고 판정을 단 한 번 시행할 수 있습니다. 이 때 성가단원이 자신이 ``순수한 영혼"임을 알게 된다면, 다음 이야기를 얻습니다:
			\begin{spoiler}{순수한 영혼의 피}{[신성]}
				\entry[\hline]{피를 담을 수 있는 용기가 있을 때, 체력 1d3을 소모하고 \storyref{paladin-cross:pure-blood}{순수한 피}를 획득할 수 있다.}
			\end{spoiler}
	
	\subsection*{\large \hyperlink{corrupt-paladin}{타락한 성기사}}
		\subsubsection*{\normalsize \storyref{paladin:fallen}{타락한}의 제거조건: ``특정한 음악을 듣는다"}
			$\Rightarrow$ 순수한 영혼의 소유자(성가단원)이 연주하는 [정화의 음악]입니다. 파티 전원이 알게 되고, \storyref{paladin:fallen}{타락한}이 공개된 상태라면 전체 공개하지만, 본인만 듣게 되거나 \storyref{paladin:fallen}{타락한}이 공개되지 않은 상태더라도 개인적으로 이 사실을 전달합니다.
			
		\subsubsection*{\normalsize \storyref{paladin:protection}{권능의 보호막}: ``타락한 존재"}
			$\Rightarrow$ 좀비와 사제입니다. 사제가 흑막이 아니라 할지라도 악마 숭배로 인해 타락했다는 사실은 변함이 없습니다. 만약 사제가 흑막 역할을 거부한 경우, 악마 역시 포함됩니다. 물론, 오르간에 앉아있는 동안에는 \storyref{priest:fallen-song}{타락의 연주자}로 인해 벗어날 수 없기 때문에 실제 흑막에게는 통하지 않습니다.
\end{document}
	
\end{document}