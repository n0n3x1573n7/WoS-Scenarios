\documentclass{report}

\begin{document}
	개별 이야기꾼들이 이미 알고 있는 정보를 전달하고, 얻을 수 있는 이야기들과 획득을 위한 필요 조건, 그리고 칭호의 효과와 칭호의 \textbf{공개 조건}과 \textbf{제거 조건}을 명백하게 밝혀야 합니다.
	
	
	
	
	
	\pagebreak \hypertarget{cursed-bard}{}
	\section{저주받은 성가단원}
		\subsection*{알고 있는 정보}
			당신은 사람들이 처음으로 언데드로 변하는 것을 본 장본인입니다. 당신의 보호자 역시 좀비로 변해버렸고요.
			
			최근 들어 성가단의 단장이자 당신에게 오르간을 연주하는 법을 가르쳐준 당신의 보호자는 당신이 잘 때 밤늦게 어딘가로 향하는 일이 잦아졌습니다. 공책을 들고가는거로 봐서는 뭔가 적을것이 있는 것 같은데, 그게 무엇일까요?
			
		\subsection*{가지고 시작하는 이야기}
			\begin{spoiler}[choir:bard]{성가대}{[역할]}
				\entry{노래를 부르거나 악기를 연주하여 음악을 통해 마법을 사용할 수 있다. 아래 노래들을 사용할 수 있다.}
				
				\entry{
				\begin{spoiler}{성가}{[노래]}
					\entry[\hline]{한 턴에 이 노래를 부르기로 선택한다면 이동을 제외한 다른 행동을 할 수 없다. 턴이 종료될 때, 같은 구역에 있는 모든 생명체의 체력을 1 회복하고, 언데드에게 [신성] 피해를 1 준다.}
				\end{spoiler}
				}
				
				\entry[\hline]{\statchange{+}{지식:음악[2]}}
			\end{spoiler}
		
		\subsection*{획득 가능한 이야기}
			\begin{spoiler}{저주받은}{[공포][칭호]}
				\pre{언데드에게 효과가 적용되도록 \storyref{choir:bard}{성가대}의 노래를 부른다.}
				
				\limitedtrauma{공포}{\storyref{choir:bard}{성가대}의 노래들의 효과가 언데드를 상대로는 적용되지 않는다.}
				
				\entry[\hline]{\textbf{제거 조건}: 특정 아이템을 소유하고 있는 동안 이 칭호를 무시한다.}
			\end{spoiler}
			
			\bigskip
			
			\storyref{choir:bard}{성가대}로 사용할 수 있는 더 많은 노래는 성가단의 연습장소에 있는 악보들으로 알 수 있습니다. 최대 한 곡을 추가로 기억할 수 있고, 그 이상을 기억하기 위해서는 악보를 직접 소유하고 있어야만 합니다.
	
	
	
	
	
	\pagebreak \hypertarget{cowardly-priest}{}
	\section{겁에 질린 사제}
		\subsection*{알고 있는 정보}
			당신이 이 좀비 사태의 원흉입니다. 당신은 사실 악마를 숭배하는 이교도이며, 이 악마를 숭배하고 소환하기 위해 악마의 가르침을 담은 책을 성경 표지만 덧씌워두었습니다. 당신은 이런 노력을 통해 악마와 계약해 [타락의 노래]를 얻었고, 이를 이용해 사람들을 좀비로 바꾸었습니다.
			
			당신의 방 안의 침대 밑에는 당신의 피를 이용하거나, 종 안에 숨겨진 열쇠를 사용해서만 안에 들어있는 해골과 악마에게서 받은 [타락의 노래]를 발견할 수 있다는 사실을 알고 있습니다. 상자가 부서지면 수면가스가 나오도록 되어 있어 어느 정도의 보호조치를 해 두었습니다.
			
			이 모든 이야기는 예배당에 진입하기 이전까지 자의적으로 공개할 수 없습니다.
			
		\subsection*{가지고 시작하는 이야기}
			\begin{spoiler}{겁에 질린}{[공포][칭호]}
				\limitedtrauma{공포}{\storyref{cleric:prayer}{기도}로 언데드에게 피해를 줄 수 없다.}
				
				\entry[\hline]{\textbf{제거 조건}: 예배당에 진입한다.}
			\end{spoiler}
			
			\begin{spoiler}[cleric:prayer]{기도}{[역할]}
				\entry[\hline]{사거리에 상관 없이 대상을 정한다. 대상의 다음 턴이 시작될 때, 생명체인 대상의 체력을 2 회복시키거나, 언데드인 대상에게 [신성] 피해를 2 준다.}
			\end{spoiler}
		
		\subsection*{획득 가능한 이야기}
			악마 숭배 사실을 들키면, \storyref{cleric:prayer}{기도}의 언데드에 대한 효과와 생명체에 대한 효과가 뒤바뀝니다\footnote{[기도]가 이제 언데드의 체력을 2 회복하거나, 생명체인 대상에게 [타락] 피해를 2 줍니다.}.
			
			또한, 예배당에 진입하면 자동으로 왼쪽의 오르간에 앉으며, 다음 이야기를 얻습니다:
			\begin{spoiler}{타락의 연주자}{[역할]}
				\entry{좀비는 당신의 명령을 따르며, 당신은 특정 효과가 생명체와 언데드에게 서로 다른 효과를 발휘한다면, 어느 쪽을 따를지를 선택할 수 있다.}
				
				\entry{매 자신의 턴에, 예배당의 의자에 앉은 좀비 하나를 일으켜 세워, 좀비 대열에 합류시킬 수 있다.}
				
				\entry[\hline]{방어적인 행동을 제외한 모든 행동을 할 수 없다.}
			\end{spoiler}
	
	
	
	
	
	\pagebreak \hypertarget{corrupt-paladin}{}
	\section{타락한 성기사}
		\subsection*{알고 있는 정보}
			이 마을에 오래전에 살고 있던 사람 중에는 전설적인 작곡가이자 사제가 있었습니다. 마을의 오르간 연주자의 부탁을 받아 그 사람의 유작을 찾아나선 당신은 그의 무덤을 파헤쳤고, 그의 관에 새겨져있던 노래를 하나 찾게 되었습니다. 좀비 사태가 발생했을 때, 당신은 무덤의 뒤처리를 하고 있었고요.
			
			당신의 집에는 이 무덤을 파헤쳐서 젖은 흙이 묻은 삽이 현관문 뒤에 있고, 당신이 사용했던 방패와 십자가에는 각각 [흑마법사의 피]와 [순수한 피]가 숨겨져 있습니다. [순수한 피]는 당신의 보호 하에 있었지만 지키지 못했던 첫 번째 이의 것으로, [순수한 피] 끼리는 잘 섞인다는 성질을 가지고 있기에 자신의 아기를 찾아 그를 보호하는데에 사용해 달라는 부탁을 받았습니다.
			
		\subsection*{가지고 시작하는 이야기}
			\begin{spoiler}{타락한}{[공포][칭호]}
				\limitedtrauma{공포}{\storyref{paladin:smite}{신성한 일격}을 사용할 수 없다.}
				
				\entry[\hline]{\textbf{제거 조건}: 자신의 집 안에 있는 [순수한 피]를 마시거나, 특정한 음악을 듣는다.}
			\end{spoiler}
			
			\begin{spoiler}[paladin:smite]{신성한 일격}{[역할]}
				\entry[\hline]{사거리에 상관 없이 대상을 정한다. 대상에게 [신성] 피해를 2 주고, [기절] 상태에 빠트린다. [기절] 상태의 상대는 다음 턴 모든 행동이 불가능하다.}
			\end{spoiler}
		
		\subsection*{획득 가능한 이야기}
			\begin{spoiler}{흑마법사의 피}{[혈액]}
				\entry[\hline]{인접한 구역을 정해, 이 피가 들어 있는 병을 깨트린다. 앞으로 두 턴간, 좀비는 해당 칸을 향해서만 움직일 수 있고, 이야기꾼들을 공격할 수 없다.}
			\end{spoiler}
	
	
	
	
	\pagebreak \hypertarget{hurt-rogue}{}
	\section{부상당한 이단심판관}
		\subsection*{알고 있는 정보}
			당신은 도둑입니다. 여러 마을을 돌아다니며 이단심판관인척 하며 그들의 재산을 훔쳤죠.
			
			이번에 당신이 노리는 것은 어떤 작곡가의 유작입니다. [정화의 노래]로 알려진 이 노래는 어느샌가 기억에서 사라졌지만, 이 마을 출신이었고 여기에 묻히기까지 했으니 흔적은 남아 있겠죠. 이 노래에 대해 조사한 결과, "순수한 피"를 가진 이가 연주해야만 효과가 있다고 하는데, 마침 이 마을에 "순수한 피"를 가진 이와 접촉한 이도, "순수한 피"를 가진 어린아이가 있다는 사실을 어둠의 경로로 정보를 알아냈습니다.
			
			이 마을에서 이 둘의 도움을 얻기 위해 마을의 수장이나 다름없는 사제에게 도움을 청하기 위해, 이 세 명에 대한 뒷조사를 단단히 하고 왔습니다.
			
			당신의 가방 안에는 이 뒷조사 자료와 [정화의 노래]에 대한 자료, 그리고 당신이 사용하는 락픽과 단도, 만능툴들이 무기 주머니에 들어있습니다. 다른 사람들이 당신을 알아보지 못하기를, 그리고 이 가방을 열지 않기를 바라는수밖에는 없겠군요.
			
		\subsection*{가지고 시작하는 이야기}
			\begin{spoiler}[rogue:hurt]{부상당한}{[공포][칭호]}
				\limitedtrauma{공포}{언데드를 대상으로 한 무기의 사용이 불가능하다.}
				
				\entry[\hline]{\textbf{제거 조건}: 자신의 진정한 정체를 다른 사람들이 추궁한다. 이 제거 조건은 공개할 수 없다.}
			\end{spoiler}
			
			\begin{spoiler}{심판}{[역할]}
				\entry[\hline]{씬이 시작할 때, [심판]의 대상을 하나 정한다. 해당 대상에게 주는 모든 피해가 1 증가한다.}
			\end{spoiler}
		
		\subsection*{획득 가능한 이야기}
			\begin{spoiler}{단도 투척}{[역할]}
				\pre{칭호 \storyref{rogue:hurt}{부상당한} 제거}
				
				\entry[\hline]{매 턴 한 번, 단도를 던질 수 있다. 단도는 떨어진 구역당 피해 1을 준다.}
			\end{spoiler}
	
\end{document}