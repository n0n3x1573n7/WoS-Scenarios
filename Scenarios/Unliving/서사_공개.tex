\documentclass{report}

\begin{document}
	\begin{story}{어둠의 시대}{[중세]}
		\flavour{마녀 사냥이 일어나는 어둠의 시대.}
		
		\entry{현대적인 기술력을 사용하거나, 신성력 외의 마법을 사용하기 위해서는 반드시 [추방] 판정을 거쳐야 한다.}
		
		\entry[\hline]{[추방] 판정에 실패하면, 개연성에 피해를 받는 대신 해당 이야기를 이 서사의 결말을 맞기 전까지 봉인하는 것으로 대체할 수 있다.}
	\end{story}
	
	\begin{story}{몰입}{[이야기의 의지]}
		\entry[\hline]{등장인물들이 알고 있는 것들 중 \textbf{공개 조건}이 존재하는 지식 또는 이야기들은 등장인물의 자존심이나 비밀과 매우 크게 연관이 있기 때문에 서사 상으로 스스로 밝힐 이유가 없는 정보들이기 때문에, 스스로 밝힐 수 없다. 이를 스스로 밝힌다면, [침범] 판정의 난이도가 영구적으로 1 증가하며, 즉시 [침범] 판정을 해, 실패한다면 해당 씬 동안 서사 속에서 얻은 것이 아닌 무작위 이야기가 하나 봉인된다.}
	\end{story}
	
	\section*{선택 이야기}
	
	이 시나리오에는 조사해야할 것 같은 대상은 많아보이지만 실제 진상에 접근하는 데에 사용할 수 있는 정보는 적습니다. 그렇기 때문에, 아래의 \storyref{system:tutorial}{튜토리얼} 이야기의 사용을 권장합니다.
	
	\begin{story}[system:tutorial]{튜토리얼}{[도움: 시스템]\ifDLC[inSANe]\fi}
		\entry[\hline]{서사 속에서 만나는 중요해보이는 인물, 사물, 상황들을 묘사하는 이야기를 알 수 있다. 이런 이야기들을 직접 조사하여 해당 이야기에 가려서 숨겨져있던 사실을 밝혀내고, 상황을 진행시킬 수 있다. 이런 조사를 하기 위해서는 상황에 맞는 어떠한 방법을 사용해도 무방하다. 이 세계의 서사를 풀어나가기 위해서는, 이 이야기들만을 조사해도 충분하다.}
	\end{story}
\end{document}