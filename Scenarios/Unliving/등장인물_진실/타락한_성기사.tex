\documentclass{report}

\begin{document}
	\subsection*{알고 있는 정보}
		이 마을에 오래전에 살고 있던 사람 중에는 전설적인 작곡가이자 사제가 있었습니다. 마을의 오르간 연주자의 부탁을 받아 그 사람의 유작을 찾아나선 당신은 그의 무덤을 파헤쳤고, 그의 관에 새겨져있던 노래를 하나 찾게 되었습니다. 좀비 사태가 발생했을 때, 당신은 무덤의 뒤처리를 하고 있었고요. 당신은 이로 인해 당신이 타락하게 되었다는 사실을 알고 있기 때문에, 다른 등장인물들이 무덤을 발견하기 전에는 이 사실을 최대한 숨기고자 합니다.
		
		당신의 집에는 이 무덤을 파헤쳐서 젖은 흙이 묻은 삽이 현관문 뒤에 있고, 당신이 사용했던 방패와 십자가에는 각각 [흑마법사의 피]와 [순수한 피]가 숨겨져 있습니다. [순수한 피]는 당신의 보호 하에 있었지만 지키지 못했던 첫 번째 이의 것으로, [순수한 피] 끼리는 잘 섞인다는 성질을 가지고 있기에 자신의 아기를 찾아 그를 보호하는데에 사용해 달라는 부탁을 받았습니다.
	
	\subsection*{가지고 시작하는 이야기}
		\begin{spoiler}[paladin:fallen]{타락한}{[공포][칭호]}
			\limitedtrauma{공포}{\storyref{paladin:smite}{신성한 일격}을 사용할 수 없다.}
			
			\entry[\hline]{\textbf{제거 조건}: 자신의 집 안에 있는 [순수한 피]를 마시거나, 특정한 음악을 듣는다.}
		\end{spoiler}
		
		\begin{spoiler}[paladin:smite]{신성한 일격}{[역할]}
			\entry[\hline]{사거리에 상관 없이 대상을 정한다. 대상에게 [신성] 피해를 2 주고, [기절] 상태에 빠트린다. [기절] 상태의 상대는 다음 턴 모든 행동이 불가능하다.}
		\end{spoiler}
	
	\subsection*{획득 가능한 이야기}
		\begin{spoiler}{흑마법사의 피}{[타락]}
			\entry[\hline]{인접한 구역에 이 피를 흩뿌릴 수 있다. 다음 두 턴 동안, 좀비는 반드시 해당 칸을 향해서 이동하며, 공격하지 않는다.}
		\end{spoiler}
		
		\begin{spoiler}[paladin:protect]{권능의 보호막}{[신성]}
			\pre{\storyref{paladin:fallen}{타락한} 칭호 제거, \storyref{search:paladin-shield}{낡은 방패} 장착}
			
			\entry[\hline]{세 턴에 한 번 사용할 수 있다. 다음 자신의 턴까지, 타락한 존재들은 드나들 수 없는 방벽을 자신이 있는 구역에 칠 수 있다. 해당 칸에 있는 타락한 존재는 무작위 인접한 구역으로 밀려난다.}
		\end{spoiler}
\end{document}