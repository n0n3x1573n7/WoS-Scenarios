\documentclass{report}

\begin{document}
	\subsection*{알고 있는 정보}
		당신은 도둑입니다. 여러 마을을 돌아다니며 이단심판관인척 하며 그들의 재산을 훔쳤죠.
		
		이번에 당신이 노리는 것은 어떤 작곡가의 유작입니다. [정화의 노래]로 알려진 이 노래는 어느샌가 기억에서 사라졌지만, 이 마을 출신이었고 여기에 묻히기까지 했으니 흔적은 남아 있겠죠. 이 노래에 대해 조사한 결과, "순수한 피"를 가진 이가 연주해야만 효과가 있다고 하는데, 마침 이 마을에 "순수한 피"를 가진 이와 접촉한 성기사가 있다는 사실을 이전에 있던 마을의 흑마술사로부터 알아냈습니다. 이 마을에서 성기사의 도움을 얻고자 마을의 수장이나 다름없는 사제에게 도움을 청하기 위해, 이 두 명에 대한 뒷조사를 간략하게나마 하고 왔습니다. 물론, 불법이니만큼 들키기 전에는 말할 생각은 없지만요.
		
		당신의 가방 안에는 이 뒷조사 자료와 [정화의 노래]에 대한 자료, 그리고 당신이 사용하는 락픽과 단도, 만능툴들이 무기 주머니에 들어있습니다. 다른 사람들이 당신을 알아보지 못하기를, 그리고 이 가방을 열지 않기를 바라는수밖에는 없겠군요.
	
	\subsection*{가지고 시작하는 이야기}
		\begin{spoiler}[rogue:hurt]{부상당한}{[공포][칭호]}
			\limitedtrauma{공포}{언데드를 대상으로 한 무기와 \storyref{rogue:judgment}{심판}의 사용이 불가능하다.}
			
			\entry[\hline]{\textbf{제거 조건}: 자신의 진정한 정체를 다른 사람들이 추궁한다. 이 제거 조건은 공개할 수 없다.}
		\end{spoiler}
		
		\begin{spoiler}[rogue:judgment]{심판}{[역할]}
			\entry{\storyref{rogue:hurt}{부상당한}이 제거되면, 이 이야기를 잃는다. 이 조건은 공개할 수 없다.}
			
			\entry[\hline]{씬이 시작할 때, [심판]의 대상을 하나 정한다. 해당 대상에게 주는 모든 피해가 1 증가한다.}
		\end{spoiler}
	
	\subsection*{직업 퀘스트}
		\begin{spoiler}{이단심판관}{[직업]}
			\entry{\begin{spoiler}{비밀의 수호자}{[퀘스트]}
						
						\entry{\textbf{성공 조건}
							\storyref{rogue:hurt}{부상당한}이 마을 내부에서 교회 구역으로 들어가기 전까지 제거당하지 않는다.
						}
						
						\entry[\hline]{\textbf{보상}
							\storyref{rogue:judgment}{심판}을 \storyref{rogue:hurt}{부상당한}을 잃더라도 사용할 수 있다.
						}
						
			\end{spoiler}}
			
			\entry[\hline]{\begin{spoiler}{영원한 비밀}{[퀘스트]}
					
					\entry{\textbf{성공 조건}
						\storyref{rogue:hurt}{부상당한}을 교회 내부로 들어갈 때 까지 제거당하지 않는다.
					}
					
					\entry[\hline]{\textbf{보상}
						\storyref{rogue:judgment}{심판}의 피해 증가량이 피해량의 50\%(최소 1)로 증가한다.
					}
					
			\end{spoiler}}
		\end{spoiler}
	
	\subsection*{획득 가능한 이야기}
		\begin{spoiler}[rogue:dagger]{단도 투척}{[역할]}
			\pre{칭호 \storyref{rogue:hurt}{부상당한} 제거}
			
			\entry[\hline]{매 턴 한 번, 단도를 던질 수 있다. 단도는 떨어진 구역당 피해 1을 준다.}
		\end{spoiler}
		
		\begin{spoiler}[rogue:lockpick]{자물쇠 따기}{[역할]}
			\pre{칭호 \storyref{rogue:hurt}{부상당한} 제거}
			
			\entry[\hline]{잠겨있는 물체를 30분을 소모하여 열 수 있습니다.}
		\end{spoiler}
\end{document}