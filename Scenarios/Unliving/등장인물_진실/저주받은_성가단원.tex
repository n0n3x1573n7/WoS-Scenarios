\documentclass{report}

\begin{document}
	\subsection*{알고 있는 정보}
		당신은 사람들이 처음으로 언데드로 변하는 것을 본 꼬마아이입니다. 당신의 보호자인 성가단장 역시 좀비로 변해버렸고요.
		
		최근 들어 성가단의 단장이자 당신에게 오르간을 연주하는 법을 가르쳐준 당신의 보호자는 당신이 잘 때 밤늦게 어딘가로 향하는 일이 잦아졌습니다. 공책을 들고가는거로 봐서는 뭔가 적을것이 있는 것 같은데, 그게 무엇일까요?
	
	\subsection*{가지고 시작하는 이야기}
		\begin{spoiler}[choir:bard]{성가대}{[역할]}
			\entry{노래를 부르거나 악기를 연주하여 음악을 통해 마법을 사용할 수 있다. 아래 노래들을 사용할 수 있다.}
			
			\entry{
				\begin{spoiler}[music:holy]{성가}{[노래]}
					\entry[\hline]{한 턴에 이 노래를 부르기로 선택한다면 이동을 제외한 다른 행동을 할 수 없다. 턴이 종료될 때, 같은 구역에 있는 모든 생명체의 체력을 1 회복하고, 언데드에게 [신성] 피해를 1 준다.}
				\end{spoiler}
			}
			
			\entry[\hline]{\statchange{+}{지식:음악[2]}}
		\end{spoiler}
	
	\subsection*{직업 퀘스트}
		\begin{spoiler}{성가단원}{[직업]}
			\entry{\statchange{+}{지식:음악[2], 제작:음악[2]}}
			
			\entry{\begin{spoiler}{축복}{[퀘스트]}
				
				\entry{\textbf{성공 조건}
				
				칭호 \storyref{bard:cursed}{저주받은} 제거}
				
				\entry[\hline]{\textbf{보상}
				
				\statchange{+}{지식:신성}
				
				\storyref{music:holy}{성가}의 회복량이 1 증가하고, 언데드에게 [신성] 피해를 주는 대신 인접한 구역으로 밀쳐내며 [기절: 1턴] 상태에 빠트릴 수 있다.}
			
			\end{spoiler}}
			
			\entry[\hline]{\begin{spoiler}{음악}{[퀘스트]}
					\entry{\textbf{성공 조건}
					
					성가단의 연습 장소에 있는 악보를 정리하여, 음악을 연주한다.}
					
					\entry[\hline]{\textbf{보상}
					
					연주한 음악을 \storyref{music:holy}{성가}와 더불어 기억할 수 있다.
					
					그 이외의 곡들은 악보를 직접 소유하고 있다면 연주하거나 노래할 수 있다.}
				
			\end{spoiler}}
		\end{spoiler}
	
	\subsection*{획득 가능한 이야기}
		\begin{spoiler}[bard:cursed]{저주받은}{[공포][칭호]}
			\pre{언데드에게 효과가 적용되도록 \storyref{choir:bard}{성가대}의 노래를 부른다.}
			
			\limitedtrauma{공포}{\storyref{choir:bard}{성가대}의 노래들의 효과가 언데드를 상대로는 적용되지 않는다.}
			
			\entry[\hline]{\textbf{제거 조건}: 특정 아이템을 소유하고 있는 동안, 또는 어떤 사실을 알게 되면 이 칭호를 무시한다. 이 시점은 시스템이 알려준다.}
		\end{spoiler}
\end{document}