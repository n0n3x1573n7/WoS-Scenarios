\documentclass{report}

\begin{document}
	아래는 각 캐릭터의 정보에서 고의적으로 누락된 부분에 대한 추가적 설명입니다.
	
	\subsection*{\large \hyperlink{cursed-bard}{저주받은 성가단원}}
		\subsubsection*{\normalsize \storyref{bard:cursed}{저주받은}의 제거조건: ``특정 아이템을 소유", ``어떤 사실을 알게 되면"}
			$\Rightarrow$ ``특정 아이템"은 성기사의 집에 있는 [순수한 피]이고, ``어떤 사실"은 자신이 ``순수한 영혼"이라는 사실입니다. [순수한 피]를 얻은 경우에는 반드시 본인에게만 해당 사실을 알리나, ``순수한 영혼"으로 인한 해제는 해당 사실을 파티 전원이 알게 되고, \storyref{bard:cursed}{저주받은}이 공개된 상태라면 전체 공개하지만, 본인만 알게 되거나 \storyref{bard:cursed}{저주받은}이 공개되지 않은 상태더라도 개인적으로 이 사실을 전달합니다.
		
		\subsubsection*{\normalsize 자신이 ``순수한 영혼"임을 확실히 알게 되었을 때}
			$\Rightarrow$ 성기사의 [순수한 피]를 사용해 ``순수한 영혼"을 감별하고자 한다면, 감별 대상자의 체력 1을 감하고 판정을 단 한 번 시행할 수 있습니다. 이 때 성가단원이 자신이 ``순수한 영혼"임을 알게 된다면, 다음 이야기를 얻습니다:
			\begin{story}{순수한 영혼의 피}{[신성]}
				\entry{피를 담을 수 있는 용기가 있을 때, 체력 1d3을 소모하고 \storyref{paladin-cross:pure-blood}{순수한 피}를 획득할 수 있다.}
			\end{story}
	
	\subsection*{\large \hyperlink{corrupt-paladin}{타락한 성기사}}
		\subsubsection*{\normalsize \storyref{paladin:fallen}{타락한}의 제거조건: ``특정한 음악을 듣는다"}
			$\Rightarrow$ 순수한 영혼의 소유자(성가단원)이 연주하는 [정화의 음악]입니다. 파티 전원이 알게 되고, \storyref{paladin:fallen}{타락한}이 공개된 상태라면 전체 공개하지만, 본인만 듣게 되거나 \storyref{paladin:fallen}{타락한}이 공개되지 않은 상태더라도 개인적으로 이 사실을 전달합니다.
			
		\subsubsection*{\normalsize \storyref{paladin:protection}{권능의 보호막}: ``타락한 존재"}
			$\Rightarrow$ 좀비와 사제입니다. 사제가 흑막이 아니라 할지라도 악마 숭배로 인해 타락했다는 사실은 변함이 없습니다. 만약 사제가 흑막 역할을 거부한 경우, 악마 역시 포함됩니다. 물론, 오르간에 앉아 [타락의 노래]를 연주하는 동안에는 \storyref{priest:fallen-song}{타락의 연주자}로 인해 벗어날 수 없기 때문에 실제 흑막에게는 통하지 않습니다.
\end{document}