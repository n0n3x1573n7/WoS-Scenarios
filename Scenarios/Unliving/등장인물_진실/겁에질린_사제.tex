\documentclass{report}

\begin{document}
	\subsection*{알고 있는 정보}
		당신이 이 좀비 사태의 원흉입니다. 당신은 사실 악마를 숭배하는 이교도이며, 이 악마를 숭배하고 소환하기 위해 악마의 가르침을 담은 책을 성경 표지만 덧씌워두었습니다. 당신은 이런 노력을 통해 악마와 계약해 [타락의 노래]를 얻었고, 이를 이용해 사람들을 좀비로 바꾸었습니다.
		
		당신의 방 안의 침대 밑에는 당신의 피를 이용하거나, 종 안에 숨겨진 열쇠를 사용해서만 안에 들어있는 해골과 악마에게서 받은 [타락의 노래]를 발견할 수 있다는 사실을 알고 있습니다. 상자가 부서지면 수면가스가 나오도록 되어 있어 어느 정도의 보호조치를 해 두었습니다.
		
		이 모든 이야기는 예배당에 진입하기 이전까지 자의적으로 공개할 수 없습니다.
	
	\subsection*{가지고 시작하는 이야기}
		\begin{spoiler}{겁에 질린}{[공포][칭호]}
			\limitedtrauma{공포}{\storyref{cleric:prayer}{기도}로 언데드에게 피해를 줄 수 없다.}
			
			\entry[\hline]{\textbf{제거 조건}: 예배당에 진입한다.}
		\end{spoiler}
		
		\begin{spoiler}[cleric:prayer]{기도}{[역할]}
			\entry[\hline]{사거리에 상관 없이 대상을 정한다. 대상의 다음 턴이 시작될 때, 생명체인 대상의 체력을 2 회복시키거나, 언데드인 대상에게 [신성] 피해를 2 준다.}
		\end{spoiler}
	
	\subsection*{획득 가능한 이야기}
		악마 숭배 사실을 들키면, \storyref{cleric:prayer}{기도}를 포함한 언데드와 생명체에 서로 다른 효과를 주는 능력 모두의 언데드에 대한 효과와 생명체에 대한 효과가 뒤바뀌며, 언데드에게 [신성] 피해를 주는 기술의 경우 생명체에게 [타락] 피해를 줍니다\footnote{[기도]의 경우, 이제 언데드의 체력을 2 회복하거나, 생명체인 대상에게 [타락] 피해를 2 줍니다.}.
		
		또한, 예배당에 진입하면 자동으로 왼쪽의 오르간에 앉으며, 다음 이야기를 얻습니다:
		\begin{spoiler}{타락의 연주자}{[역할]}
			\entry{좀비는 당신의 명령을 따르며, 당신은 특정 효과가 생명체와 언데드에게 서로 다른 효과를 발휘한다면, 어느 쪽을 따를지를 선택할 수 있다.}
			
			\entry{매 자신의 턴에, 예배당의 의자에 앉은 좀비 하나를 일으켜 세워, 좀비 대열에 합류시킬 수 있다.}
			
			\entry[\hline]{방어적인 행동을 제외한 모든 행동을 할 수 없다.}
		\end{spoiler}
\end{document}