\documentclass{report}

\begin{document}
	\textbf{시나리오 이름}: 안녕히, 그리고 이번 생은 고마웠어요.
	
	\textbf{시나리오 작가}: None(\href{https://www.twitter.com/n0n3x1573n7_WS}{@n0n3x1573n7\_WS})
	
	\textbf{사용 룰}: 이야기의 방랑자들(Wanderers of the Tales)
	
	\textbf{권장 인원}: 1인. 2인 진행이 가능하나 시스템의 역할이 매우 중요하며 복잡합니다.
	
	\subsubsection*{주의사항}
	
	이야기의 흐름 특성상, \emph{\textbf{굉장히 어두운 분위기}}이며, 시놉시스에 적혀있듯 시나리오 진행 중 이야기꾼의 소중한 사람인 NPC가 사망하는 것으로 이야기가 시작됩니다. 또한 이야기꾼과 NPC가 1회 이상 사망할 수 있으며, 이야기꾼의 공포심을 직접적으로 사용하는 장면이 등장합니다. \textbf{이에 대하여 고지하지 않은 채로 시나리오를 진행하거나, 동의를 받았다 할지라도 이로 인한 문제가 발생했음에도 불구하고 세션을 강행하는 행위를 \emph{절대로} 금합니다.}
	
	\subsubsection*{시놉시스}
	
	NPC와 원래는 기분 좋게 놀러가기로 한 날, 이야기꾼은 NPC의 죽음을 마주합니다.
	
	\world{......이게 현실일 리 없어.}
	
	...하지만 이미 벌어진 일은 벌어진 일. 이야기꾼은 눈앞이 깜깜해지기 시작합니다.
	
	\subsubsection*{들어가기 전에}
	
	이 이야기의 역할에 따라 이야기꾼과 NPC는 일반인입니다. 특수한 능력을 가지고 있을 수는 있지만, 이를 사용할 수는 없습니다. 따라서, 이야기꾼의 성격과 지식 등을 기반으로 시트를 작성하는 것을 권장합니다.
	
	또한, 이야기꾼의 공포심 한가지를 반드시 정해 시트에 작성해주세요.
	
\end{document}