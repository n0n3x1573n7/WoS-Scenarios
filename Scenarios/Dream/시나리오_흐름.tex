\documentclass{report}

\begin{document}
	\section{들어가기 전에}
		이 시나리오는 뒷맛이 굉장히 찝찝하거나, 이야기가 덜 끝난 상태에서 종료된 것 처럼 보일 수 있습니다.
		
		시작하기 전에, 시스템은 NPC를 죽일 서로 다른 세 가지 이상의 방법을 생각해둡니다. 이와는 별개로, 약속을 취소했을 경우에 대한 방법 역시 생각해두는 것을 추천합니다. 교통사고, 공사장 관리부실, 난간 관리부실로 인한 추락, 가스 폭발 등 사고사로 보일 수 있는 방법을 권장하며, NPC의 백스토리에 따른 방법 역시 권장합니다.
		
		또한, 플레이테스트 중 받은 공통적 피드백이 힌트의 부족이었던 만큼, 진행 중에 진상과 분기에 대한 단서를 많이 주는 것을 권장합니다.
		
		마지막으로 본 시나리오의 스토리에 대한 영감을 아지랑이 데이즈(\href{https://youtu.be/EMGyiiTC7sg}{공식}, \href{https://youtu.be/VNTgXbCVSUc}{자막})에서 받았음을 밝힙니다.
		
	\section{도입}
		이야기꾼은 특정 날짜\footnote{이후에 이 날짜를 사용해야하므로, 세션 당일 등 기억하기 쉬운 날짜로 하는 것을 권장드립니다. 날짜 대신 NPC의 문자 등을 활용할 수도 있습니다.}에, 기분좋게 침대에서 일어납니다. 오늘은 NPC와 함께 놀러가기로 한 날이거든요! 기분좋게 일어난 이야기꾼은 NPC와의 약속 장소로 나갑니다. 놀러가기로 하던 곳으로 가던 중, 어떤 이유로 NPC는 사고를 당해 이야기꾼의 눈 앞에서 사망합니다. 차 사고가 날수도, 공사장에서 무언가 떨어져도, 이야기꾼은 이를 막기 위해 아무것도 할 수 없습니다. 그저 허망하게 바라만 보고 있을 뿐입니다. 이야기꾼은 눈에서 눈물이 흘러나오는 것을 막을 수 없습니다.
		
		이야기꾼의 시야가 점점 흐릿해지더니, 충격 때문인지 주변의 소리가 삐... 거리는 소리로만 들립니다.\footnote{후술하겠지만, 여러 텍스트에 걸쳐 총 7번 또는 그 이하로 언급하는 것을 추천합니다.} 하지만 이 소리는 점점 흐려지더니, 이내 주변의 소리가 전혀 들리지 않습니다. 이야기꾼이 다시 눈을 뜨면, 어딘가 굉장히 어두운 곳입니다. 이곳에서 이야기꾼이 어떤 조사를 해도 아무것도 알 수는 없습니다. 이내 어둠의 형체가 모여들더니 NPC의 죽음을 이야기꾼의 앞에 끝없이 다시 보여줍니다. 이후, 이야기꾼은 ``어떻게 하면 이를 막을 수 있었을까" 하는 생각을 하고\footnote{직접적으로 구하고 싶냐고 의문의 목소리가 물어보는 방법도 있습니다.}, 그러자 NPC의 죽음을 어둠의 형체가 이야기꾼에게 그대로 재현합니다.
		
		삐...\footnote{역시 후술하겠지만, 6번 또는 그 이내로 언급하는 것을 추천합니다.} 하는 소리가 들리다 잦아들었고, 다시금 일어났을 때, 이야기꾼은 처음에 일어난 곳에서 일어납니다.
		
	\section{진행}
		
		위의 도입부는 ``NPC와 놀러가기로 한 날의 아침부터 NPC의 죽음까지", 그리고 ``소름돋는 의문의 공간에서 눈을 뜬 이후 공포에 의해 잠식되기까지"의 두 부분으로 나뉘어집니다. 앞으로도 이 두 가지 부분이 계속해서 반복됩니다. 편의상 각각 ``낮"과 ``밤"으로 부르겠습니다. 낮과 밤이 전환될 때, ``삐..."하는 소리는 전환될때마다 한번씩 덜 들려옵니다. 첫 두 번의 경우 정신이 없어 덜 들었을 수는 있으나, 이후부터는 다른 소리에 섞이지 않고 ``삐..." 하는 소리를 연속적으로 들을 수 있게 하는 것을 추천합니다. 즉, 시점의 전환은 다음과 같이 진행됩니다:
		
		\begin{tightcenter}
			\begin{tabular}{!{\color{black}\vrule}c!{\color{black}\vrule}c!{\color{black}\vrule}c!{\color{black}\vrule}}
				\hline
				장면 구분        & 장면 전환 트리거 & 장면 전환                                  \\\hline\hline
				첫 번째 낮(도입) & NPC의 죽음       & ``삐" 소리 7회 이하                        \\\hline
				첫 번째 밤(도입) & 공포의 잠식      & ``삐" 소리 6회 이하                        \\\hline
				두 번째 낮       & NPC의 죽음       & ``삐" 소리 5회                             \\\hline
				두 번째 밤       & 공포의 잠식      & ``삐" 소리 4회                             \\\hline
				세 번째 낮       & NPC의 죽음       & ``삐" 소리 3회                             \\\hline
				세 번째 밤       & 공포의 잠식      & ``삐" 소리 2회                             \\\hline
				마지막 낮        & NPC의 죽음       & ``삐" 소리 1회                             \\\hline
				마지막 밤        & 공포의 잠식      & 분기 \hyperlink{dream-limbo}{``연옥"} 발동 \\\hline
			\end{tabular}
		\end{tightcenter}
		
		\subsection{``낮"의 이야기}
		
		낮에는 위와 같은 일이 반복됩니다. 같은 날에 일어나, NPC의 독촉 또는 기대하는 듯한 연락을 듣는 것으로 시작하는 것을 권장합니다. 이 때, 어떤 소재를 하나 잡아 점점 나빠지는 것을 연출해도 좋습니다. 예를 들어 날씨라면 처음에는 구름 한점 없이 맑은 날, 두번째에는 구름이 한두점정도 낀 날, 세번째는 조금 흐린듯 구름이 옅게 깔린 날, 마지막은 완전히 흐린 날으로 표현할 수 있습니다. 꽃말과 같은 것으로 가볍게 암시할 수도 있습니다. 예를 들어 처음에는 로즈마리(행복한 추억) 또는 캐모마일(고난 속의 힘), 다음날에는 천수국(이별의 슬픔과 절망, 죽은 사람에 대한 기억), 셋째날에는 시스투스(나는 내일 죽습니다), 그리고 마지막으로 흰 안개꽃(죽음)을 사용할 수 있겠죠.
		
		또한, 첫 번째 낮을 제외하고는 반드시 한 번 있어서는 안되는 곳에 병원 관련 물품을 등장시키는 것을 추천합니다. 가령, 거울을 봤더니 환자복을 입고 있다던지, 붕대를 감고 피투성이로 걸어가는 행인이라던지, 자신의 몸에 멍과 상처가 존재한다던지 하는 식입니다. 다만, 이러한 것들은 눈을 깜빡하면 원래대로 돌아가있습니다. 비슷한 아이디어로, 의사가 회진을 돌 때의 말이나 심폐소생술 또는 심장 재세동기(AED)의 안내멘트 등을 환청으로 듣게 하는 방법도 있습니다. 또한 일상적인 물품들을 관찰하려고 할 때 병원에서 사용되는 물품들을 암시해주어도 괜찮습니다. 가령, 옷의 경우 환자복, 팔찌는 환자팔찌, 반지는 산소포화도 측정기, 방 안 집기는 의료기기 등으로 암시할 수 있습니다.
		
		이야기꾼은 NPC의 사망 직전까지 일어나는 일을 바꿈으로서 NPC의 죽음을 막으려고 시도할 수 있지만, NPC는 결국 사망하고, 삐 소리가 난 후 밤으로 시점이 전환됩니다.
		
		분기 \hyperlink{dream-sacrifice}{``희생"}의 발동을 위해서는 NPC의 죽음을 이야기꾼이 직접 막아야 합니다. 이를 할 수 있다는 사실을 반드시 인지시키되, 이 경우 NPC 대신 이야기꾼이 사망한다는 사실을 고지해주세요. 또는 판정을 강제할 수는 있지만, 권장하지는 않습니다. 이를 위해서 반드시 판정을 하여 성공해야만 하며, 5+(이번 장면 전환시 나게 될 ``삐" 소리의 수) 이상으로 성공한다면, 이야기꾼이 NPC의 죽음을 막는 동시에 희생하며 이야기꾼이 사망합니다. 실패한다면, NPC가 사망합니다.
		
		분기 \hyperlink{dream-corrupt}{``공멸"}의 발동을 위해서는 함께 죽자고 NPC를 설득해야만 합니다. 설득은 한 낮에 한번만 할 수 있으며, 반드시 대항 판정을 하여 성공해야만 하며, 5+(이번 장면 전환시 나게 될 ``삐" 소리의 수) 이상으로 성공한다면, NPC는 이야기꾼과 함께 죽는데에 동의합니다. 단, 설득에 실패한다고 해서 \hyperlink{dream-sacrifice}{희생} 분기를 따를 수 없는 것은 아닙니다.
		
		두 번째 낮에는 분기의 발동이 불가합니다. \hyperlink{dream-sacrifice}{희생} 분기를 따른다면 판정에 성공한다 할지라도 이야기꾼이 이를 막기 이전에 NPC가 사망하며, \hyperlink{dream-corrupt}{공멸} 분기를 따른다면 어딘가 잘못되어 NPC만 사망하게 됩니다. 하지만 이 때의 성공을 발판으로, 이후 해당 분기에 대한 판정을 할때마다 판정에 +1을 받습니다.
		
		\subsection{``밤"의 이야기}
		
		두 번째 밤부터는 그르렁대던 것의 형체가 점점 드러납니다. 이는 이야기꾼 본인의 악몽이 실체화된 악몽 괴물으로, 이야기꾼의 이야기의 반대의 의미를 가진 이야기를 가지고 있습니다\footnote{가령, 이야기꾼이 `여린 마음'이라면 `강인한 마음', `거미 공포증'이라면 `거미는 내 친구' 등.}\footnote{부정적 서술이나 이야기의 경우, 반대의 의미를 가진 이야기 대신 더 심화된 이야기를 가지게 할 수 있습니다. `독설가'가 `욕쟁이'가 된다거나 하는 식이죠.}. 각 밤에 악몽 괴물이 분기에 대한 힌트를 주는 것을 권장합니다. \hyperlink{dream-corrupt}{공멸} 분기의 경우 ``같이 죽자"는 의미의 말이나 ``둘 다 살아날 길은 결국 둘 다 죽는 길이다"라는 말을 할 수 있고, \hyperlink{dream-sacrifice}{희생} 분기의 경우 ``혼자 살고자 죽인 것이냐"는 비난 등을 할 수 있습니다. 대신, 이러한 힌트가 감정적인 고통을 유발할 수 있는 경우 이후 이야기꾼의 감정적인 상태를 \emph{반드시} 확인해주세요.
		
		\begin{itemize}
			\item 두 번째 밤: 이야기꾼의 공포가 형체가 되어 나타나고, ``같이 죽자..."는 으스스한 목소리가 들립니다.
			\item 세 번째 밤: 그르렁거리는 소리가 점차 이야기꾼 본인의 목소리로 바뀌더니, 이야기꾼의 공포 또는 가장 깊고 어두운 비밀을 이야기꾼에게 속삭입니다.
			\item 마지막 밤: 공포의 형체 안에서 이야기꾼 본인의 모습이 보여집니다.
		\end{itemize}
		
		밤에 진입할 때에는 지난 밤이 종료했을때, 이야기꾼이 공포에 의해 잠식되기 직전의 상황으로 되돌아옵니다. 위의 내용을 그대로 따를 필요는 없으며, 알 수 없는 형태가 등장해 진상에 대한 힌트를 주거나 분기에 대한 암시를 주는 것으로 해도 괜찮습니다. 맞서 싸우려는 의향이 있다면 공포에 맞서 싸울 수 있으나, 이에 실패하거나 그러한 의향이 없다면 공포는 이야기꾼을 집어삼키고 낮으로 넘어갑니다.
		
		분기 \hyperlink{dream-fight}{``대항"}의 발동을 위해서는 공포에 맞섰을 때 포기하지 않고, 이를 받아들이거나 공포와 맞서 싸워 이겨야합니다. 이를 위해서 반드시 판정을 하여 성공해야만 하며, 5+(이번 장면 전환시 나게 될 ``삐" 소리의 수) 이상으로 성공한다면, 이야기꾼은 공포를 받아들이고, 마침내는 수용하게 됩니다.
		
		두 번째 밤에는 분기의 발동이 불가합니다. \hyperlink{dream-fight}{대항}을 위한 판정에 성공한다 할지라도, 공포가 이야기꾼을 집어삼켜버리지만, 이후 다시 \hyperlink{dream-fight}{대항}을 위한 판정을 할때마다 판정에 +1을 받습니다.
		
	\section{진상}
	이야기꾼과 NPC는 같이 사고를 당해 혼수상태에 빠져, 병원 중환자실에 입원해있습니다. 장면이 전환될 때 나는 삐 소리는 심박측정기의 소리입니다. 이야기꾼과 NPC의 꿈은 모종의 이유로 연결되어 있으며, 각자의 무의식 속에서 상대방은 계속해서 죽어갑니다. 자신이 ``낮"인 동안 상대는 ``밤"에서 공포에 맞서고 있고, ``밤"인 동안 상대는 ``낮"에서 상대의 죽음을 계속해서 보고 있는 것이죠\footnote{자신의 ``밤"은 자신의 죽음에 대한 공포에 대한 무의식, 자신의 ``낮"은 상대의 죽음에 대한 공포에 대한 무의식으로 해석할 수도 있습니다.}. ``낮"의 상태 악화, 그리고 줄어드는 ``삐" 소리는 이야기꾼과 NPC의 상태 악화와 직결됩니다. 엔딩에 대하여 미리 간단하게 설명하자면, 이 꿈 속에서 제 시간 안에 자발적으로 사망한다면 현실에서 깨어납니다. 상대가 먼저 나가거나 시간이 너무 오래 지나버린다면, 나갈 방법이 없이 악몽 속에 갇혀 공포에 떨고 있게 되는 것입니다.
		
\iffalse
		\begin{tightcenter}
			\begin{tabular}{!{\color{black}\vrule}c!{\color{black}\vrule}c!{\color{black}\vrule}}
				\hline
				일상적 물품 & 실제 물품 \\\hline\hline
				옷 & 환자복 \\\hline
				팔찌 & 환자팔찌 \\\hline
				반지 & 산소포화도 측정기\\\hline
				방 안의 집기 & 의료기기 \\\hline
			\end{tabular}
		\end{tightcenter}
\fi
		
	\section{분기}
		이 시나리오에서는 엔딩으로 가는 분기가 네 가지 존재합니다. 낮에 두 가지 분기가, 밤에 두 가지 분기가 존재합니다. 분기의 조건을 충족하면, 서사의 끝에 다다릅니다.
		
		\subsection{``낮"의 분기}
			\hypertarget{dream-sacrifice}{}
			\subsubsection{희생}
			``낮"에 이야기꾼이 NPC 대신 죽은 경우, 현실에서 이야기꾼은 깨어나나 NPC는 사망합니다. 이야기꾼은 병실에서 일어나지만, NPC가 깨어나지 못했다는 소식을 듣습니다. 꿈 속에서 보았던 것이 실제로 NPC가 아니었을까 의심하기 시작하고, 그렇기에 NPC가 깨어나지 못할 것임을 직감합니다.
			
			\hypertarget{dream-corrupt}{}
			\subsubsection{공멸}
			``낮"에 이야기꾼과 NPC가 함께 죽은 경우, 현실에서 이야기꾼과 NPC는 함께 깨어납니다. 둘의 꿈이 연결되어 있었다는 것을 느끼지만, 왜 연결되었던 것인지, 그 안에서 쫓아오던 공포는 도대체 무엇인지는 알 수 없습니다.
		
		\subsection{``밤"의 분기}
			\hypertarget{dream-fight}{}
			\subsubsection{대항}
			``밤"에 이야기꾼이 악몽 괴물에 맞서 싸워 승리하거나 싸울 의사를 드러내지 않고 완전히 놓아버리는 듯한 분위기의 경우\footnote{이야기꾼이 NPC의 \hyperlink{dream-corrupt}{공멸}에 저항하고, \hyperlink{dream-sacrifice}{희생}을 받아들인 것입니다.}, 현실에서 NPC는 깨어나나 이야기꾼은 그렇지 못합니다. 이야기꾼은 영원한 악몽 속에서 살아가게 되고, 악몽은 이야기꾼을 영원히 쫓아옵니다.
			
			서사는 NPC의 시점에서 계속 진행되며, NPC는 이야기꾼이 깨어나지 못했다는 소식을 듣습니다. NPC는 이야기꾼이 깨어나지 못할 것임을 직감합니다.
			
			\hypertarget{dream-limbo}{}
			\subsubsection{연옥}
			밤에서 낮으로 전환될 때 더 이상 줄어들 ``삐..." 소리가 없는 경우, 현실에서 이야기꾼과 NPC는 함께 깨어나지 못합니다. 두 사람은 영원한 악몽 속에서 영원히 배회하며 공포와 고통 속에 있게 될 것입니다.
		
		\bigskip
		이야기꾼 또는 NPC 한 명만 깨어났을 때, 상대가 깨어나지 못했음을 직감하는 대신 이게 꿈인지 현실인지 분간하지 못하는 상태가 되도록 해도 괜찮습니다.
\end{document}