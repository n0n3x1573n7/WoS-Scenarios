\documentclass{report}

\begin{document}
	\section{의뢰}
	깨달은 자인 경우, 태초의 이야기에서 어떤 수상한 이야기꾼\footnote{\label{who-is-it}이 때 의뢰를 주는 이는 기본적으로는 정체를 알 수 없습니다. 하지만, 만약 길잡이꾼이 이 시나리오를 진행하여, \hyperlink{vamp-master}{[뱀파이어의 정신 지배]} 이야기를 받은 이야기꾼이 있다면, 해당 이야기꾼이 전달하는 의뢰로 하는 것을 추천드립니다.}의 의뢰를 받아 이 이야기 속으로 들어갑니다. 
	
	깨달은 자가 아닌 경우, 누군가\footnoteref{who-is-it}의 의뢰를 받아 이 마을 안으로 들어갑니다.
	
	처음 마을에 도착할 때, 해가 지고 안개가 막 끼려고 하는 상황에서 극도의 피로를 느끼는 이야기꾼이 도착하게 하는 것을 권고드리지만, 그렇기 하지 않더라도 이야기꾼이 적어도 하룻밤 정도는 여관에서 잠을 청하도록 하는 것을 추천드립니다. 여관에서 쫓겨난다면 주점으로 가도록 유도해, 그곳에서 곯아떨어지게 하는 방법도 괜찮습니다.
	
	혹시 이야기 진입시에 의상에 관한 질문을 한다면, \emph{최대한 촌스럽게}, 중세풍의 적절한 복장으로 환복시켜주시면 됩니다.
	
	또한 여관 또는 다른 곳에서 돈을 사용할 일이 있다면, 판정을 하여 그 판정치에 따라 다음 돈을 획득합니다:
	
	\begin{tabularx}{\linewidth}{c!{\color{black}\vrule}X}
		판정치 & \makecell[c]{소지한 돈의 양} \\\hline\hline
		0\textasciitilde1 & 비상금 정도의 돈밖에 없거나, 돈이 아예 없습니다. 여관에서는 특별한 이유가 없다면 이야기꾼을 들이지 않으려 할 것입니다.\\\hline
		2\textasciitilde3 & 여비 정도의 돈밖에 없습니다. 주점이나 식료품점 등에서 한두끼정도를 먹을 수 있을 것입니다. 여관에 간다면, 모든 돈을 받고도 잠자리가 불편한, 가장 나쁜 방을 내어 줄 것입니다.\\\hline
		4\textasciitilde5 & 하루 정도 넉넉히 묵을 돈이 있습니다. 여관에서는 모든 돈을 받는다면 평범한 방을 내어 줄 것이고, 아침식사 또한 제공할 것입니다.\\\hline
		6\textasciitilde7 & 3일에서 4일 정도 넉넉히 묵을 돈이 있습니다. 여관에서는 모든 돈을 받는다면 사나흘정도의 평범한 방을 내어 줄 것이고, 아침과 저녁 식사 또한 제공할 것입니다.\\\hline
		8\textasciitilde9 & 일주일 정도 넉넉히 묵을 돈이 있습니다. 여관에서는 모든 돈을 받는다면 일주일 정도 평범한 방을 내어 주거나, 사흘 정도 특실을 내어 줄 것이고, 그 기간동안 아침과 저녁 식사를 제공할 것입니다.\\\hline
		10\textasciitilde11 & 이야기꾼은 자신의 소지품에서 옆 마을 영주로부터 의뢰금으로 받은 무거운 돈주머니를 찾을 수 있습니다! 돈의 제약이 사라집니다.
	\end{tabularx}
	
	\section{낮의 마을}
	아이들, 마을 주민, 신부에게서 증언을 얻을 수 있습니다.
	
	\subsection{아이들}
	아이들은 마을 이곳저곳을 뛰어다니며 놀고 있습니다. 아이들은 각각 안개가 나타난 날 부터 밤에 사람으로 보이는 형체가 갑자기 다른 모습으로 변한 것을 목격했습니다. 늑대 모양으로 바뀌기도, 박쥐떼가 날아가자 사라지기도, 특히 보름달이 뜬 날에 붉은 빛 안개로 흩어지기도 하며 여러가지 모습으로 변한 것을 목격했습니다.
	
	\subsection{마을 주민}
	마을 주민들은 모두 피곤한 눈으로 돌아다니고 있습니다.
	
	\begin{lite}{마을 주민}
		\subject{수면 부족}
		
		\negative{피곤해 뒤지겠네}
	\end{lite}
	
	마을 주민에게서 정보를 얻기 위해서는 마을 주민과 대결해 승리해야 합니다. 마을 주민은 ``피곤해 뒤지겠네"를 이용하여 대결에 저항합니다. 이 대결은 하루에 한 번 할 수 있습니다.
	
	마을 주민은 대결에서 패배한다면 매일 밤 안개가 끼기 시작한 이후로 피가 빨려 죽은 동물이 꽤 자주 발견되었으며, 모든 사람들이 악몽을 꾸기 시작했다는 사실을 알려줍니다.
	
	대결에서 3 이상의 수치로 성공한다면, 악몽이 모두 다 같은 내용이며, 목을 어떤 짐승이 물어 뜯는 꿈으로, 자고 일어나서도 그 느낌이 조금은 남아있다고 얘기합니다.
	
	혹시 마을 주민의 목을 관찰하려고 한다면, 판정을 시도할 수 있습니다. 판정에 ``비협조적"이라는 이유와 ``자세히 관찰해야 한다"는 이유를 들어서 각각 -1의 보정치를 적용한다는 점을 미리 알려주세요. 이 판정에서 7 이상으로 성공한다면 목에 붉은 점 두 개가 있다는 사실을 알려주세요.
	
	\subsection{신부}
	\begin{lite}{마을의 신부}
		\subject{신성의 수호자}
		
		\negative{신성한 도구를 검증되지 않은 자에게 주려고 하지 않는다.}
		
		\positive{모든 공격에 신성함이 깃들어 있다.}
	\end{lite}
	
	마을의 신부는 성기사단장 출신으로, 뱀파이어에 대해 알고 있고, 이 사태를 신고한 장본인입니다.
	
	마을의 신부는 이야기꾼에게 \storyref{bloody:holy-water}{성수}와 \storyref{bloody:cross}{십자가}를 줄 수 있는 유일한 존재입니다. \storyref{bloody:holy-water}{성수}는 교회를 찾아온다면 신부가 지급하지만, 현재 남아 있는 성수의 양이 얼마 없어 미안하다고 합니다.
	
	\begin{lite}[bloody:holy-water]{성수}
		\subject{분산된 타락도 타격할 수 있게 해주는 것.}
		
		\negative{한번 쓸 정도의 작은 양 밖에 없다.}
	\end{lite}
	
	신부에게 다른 도구를 빌려달라고 요청한다면, 신부를 설득해야 합니다. 이를 위한 대결을 진행하여, 승리한다면 \storyref{bloody:cross}{십자가}를 지급합니다:
	
	\begin{lite}[bloody:cross]{십자가}
		\subject{타락한 이에게 영구적인 손상을 입히는 것.}
		
		\negative{나무로 만들어져 있고 낡아있어, 한번만 쓸 수 있다고 해도 운이 좋을 것 같다.}
	\end{lite}
	
	3 이상의 수치로 승리한다면, \storyref{bloody:cross}{십자가} 대신 \storyref{bloody:silver-cross}{은 십자가}를 지급합니다.
	
	\begin{lite}[bloody:silver-cross]{은 십자가}
		\subject{타락한 이에게 영구적인 손상을 입히는 것.}
	\end{lite}
	
	혹시 마을 주민과 같이 신부의 목을 관찰하려고 한다면, 신부의 목에도 역시 붉은 점 두개를 발견할 수 있습니다.
	
	\section{밤의 마을}
	밤에는 마을의 모든 집이 문을 걸어잠그고, 여관에서 잠을 청하거나 밖에서 밤을 새는 것 외의 다른 행동은 불가능합니다. 밤에는 안개가 바닥부터 점점 깔리기 시작하여, 자정이 되었을 때 최고조가 됩니다.
	
	여관 주인과 대화할 수 있습니다만, 알고 있는 것은 마을 주민과 크게 다르지 않습니다. 잠을 자고 일어나면 다시 낮이 됩니다.
	
	\begin{lite}{여관 주인}
		\subject[\solidsepline]{내 가게, 내 규칙}
		
		\subject{돈이 최고야}
	\end{lite}
	
	여관에서 잠을 자면, 여관 창문과 문 틈으로 안개가 새어들어와 안개를 마시게 됩니다.
	
	안개를 마시고 잠을 자게 되면, 누군가에게 목을 뜯기는 악몽을 꿉니다. 목이 뜯겨서 난도질당한 느낌은 잠에서 깨어난 후에도 그대로 유지됩니다. 이 때, 작은 부상인 \textcolor{Brown}{[강렬한 악몽]}과 부정적 상태인 \textcolor{RubineRed}{[목의 위화감]}을 획득합니다. \textcolor{Brown}{[강렬한 악몽]}은 최대 세 번 까지 중첩되며 중간 부상, 큰 부상으로 심화되며, \textcolor{RubineRed}{[목의 위화감]}은 무제한으로 중첩됩니다.
	
	더 좋은 방을 얻기 위한 협상, 또는 밤 수색을 위해 잠긴 문을 딸 목적으로 대결 판정이 가능합니다.
	
	혹시 여관에서 낮에 잠을 청하려고 한다면, 지난 안개를 마신 후 잠을 이미 잔 바 있다면, 아무 영향도 없습니다. 하지만 잠을 잤다면 \textcolor{Brown}{[강렬한 악몽]}이 중첩됩니다. \textcolor{RubineRed}{[목의 위화감]}은 중첩되지 않습니다.
	
	\section{영주의 성}
	영주의 성 근처에는 민가가 많습니다. 낮이라면, 아무리 큰 소리를 내도 민가의 주민들도, 영주의 성 안에서도 반응이 없습니다. 밤이라면, 바로 전투가 시작됩니다.
	
	\subsection{민가}
	낮이라면, 모든 민가가 문은 잠겨있고, 창문이 모두 판자로 막혀있습니다. 판자 틈 사이로 관찰을 위한 판정을 하여 5 이상의 값으로 성공하면 내부를 볼 수 있습니다.
	
	내부에는 미동없이 누워있는 한 사람을 볼 수 있습니다. 판자를 부수거나 문을 따는 식으로 방 안으로 들어간다면, 살아는 있지만 몸이 매우 차갑다는 사실을 알 수 있습니다.
	
	\subsection{영주의 성문}
	영주의 성문은 굳게 잠겨있습니다. 인기척조차 느껴지지 않습니다. 관찰을 위한 판정을 합니다. 5 이상의 값으로 성공한다면, 성 안으로 들어가는 핏자국의 길을 발견합니다. 그렇지 못했다면, 성문 앞에 있는 핏자국 하나만을 발견합니다.
	
	\section{뱀파이어}
	영주의 성 근처에서 밤이 되기를 기다린다면, 민가에서 사람들이 나와서 안개처럼 흩어지며 안개를 점점 짙게 만드는 것을 목격합니다.
	
	\begin{lite}{짙은 안개}
		\negative{안개로 인해서 앞이 잘 보이지 않는다.}
	\end{lite}
	
	마침내 안개 속을 뚫고 영주의 성의 문이 열리고, 뾰족한 송곳니를 가진 영주가 등장합니다. 영주는 핏빛 안개로 흩어지더니 이야기꾼의 눈 앞에 등장하고, 영주와 이야기꾼의 전투가 시작됩니다.
	
	\medskip
	
	이야기꾼이 성벽을 타는 중 밤을 맞이했다면, 영주는 동일한 곳에서 등장합니다. 핏빛 안개로 흩어진 영주는 커다란 박쥐 날개가 달린 채로 이야기꾼의 뒤에 날고있는채로 등장합니다. 이 연출은 영주의 \storyref{bloodsucking-bat}{흡혈 박쥐} 이야기의 응용으로, 이 응용법은 연출에서만 사용하고 전투가 시작하는 순간 더 이상 사용하지 않습니다.
	
	성벽에서 낙하를 하겠다는 선언을 한 경우, 판정을 통해 부상 정도를 결정해도 괜찮습니다.
	
	\medskip
	
	이야기꾼이 영주의 성 안으로 들어간 채로 밤이 되었다면, 영주는 성 앞이 아닌 이야기꾼의 등 뒤에서 나타납니다. 즉시, 영주는 이야기꾼의 목을 물고자 시도하고, 기습으로 인해 +1을 받습니다. 또한, 다음 이야기의 적용이 있다는 사실을 알려주세요:
	
	\begin{lite}{영주의 성}
		\positive{뱀파이어들의 홈그라운드}
	\end{lite}
	
	전투 중의 뱀파이어 영주의 이야기는 다음과 같습니다:
	
	\begin{lite}{뱀파이어 영주}
		\entry[\solidsepline]{다섯 개의 부상칸을 가집니다. 부상을 입을 때, 심화 단계에 상관없이 부상칸 하나에 부상 내용을 채워넣습니다. 한 턴에는 최대 한 칸의 부상칸만 채울 수 있습니다.}
		
		\subject{\hypertarget{existence-of-the-fallen}{타락의 존재}}
		
		\entry[\solidsepline]{\storyref{bloody:holy-water}{성수}, \storyref{bloody:cross}{십자가} 또는 \storyref{bloody:silver-cross}{은 십자가}로 인한 부상을 입는다면, 부상 내용을 채우는 대신 해당 부상칸을 완전히 삭제(파괴)합니다.}
		
		\subject{피의 안개}
		
		\entry[\solidsepline]{부상칸이 남아있지 않거나 모두 채워지면, 다음에 다시 보자는 이야기를 하며 안개가 되어 사라집니다.}
		
		\subject{\hypertarget{bloodthirstiness}{흡혈의 주술}}
		
		\positive{\hypertarget{fangs}{날카로운 송곳니}}
		
		\entry{\storyref{fangs}{날카로운 송곳니}의 도움을 받은 공격으로 이야기꾼의 피를 흡혈할 때 마다, 차 있는 부상칸이 있다면 하나를 비웁니다.}
		
		\entry{\textcolor{Periwinkle}{[뱀파이어의 송곳니]} 부상이 없는 이야기꾼의 피를 처음 흡혈할 때, \textcolor{RubineRed}{[목의 위화감]} 상태가 모두 사라지고, \textcolor{Periwinkle}{[뱀파이어의 송곳니]} 큰 부상을 줍니다.}
		
		\entry[\solidsepline]{상대에게 \textcolor{Periwinkle}{[뱀파이어의 송곳니]} 부상과 \textcolor{Brown}{[강렬한 악몽]} 부상이 있다면, \textcolor{Brown}{[강렬한 악몽]} 부상이 심화도에 따라 가벼운:\textcolor{CarnationPink}{[뱀파이어의 정신 침식]}, 중간:\textcolor{Rhodamine}{[뱀파이어의 정신 침범]}, 큰:\textcolor{Magenta}{[뱀파이어의 정신 지배]}로 바뀝니다.}
		
		\subject{\hypertarget{mind-control}{정신의 주술}}
		
		\entry{한 턴을 소모해 주술을 영창합니다. 이야기꾼은 영창을 막기 위해 시도할 수 있습니다.}
		
		\entry{영창에 성공했고, \textcolor{Rhodamine}{[뱀파이어의 정신 침식]},\textcolor{Rhodamine} {[뱀파이어의 정신 침범]}, 또는 \textcolor{Magenta}{[뱀파이어의 정신 지배]} 부상이 없다면, \textcolor{Brown}{[강렬한 악몽]} 부상을 줍니다.}
		
		\entry{영창에 성공했고, \textcolor{Rhodamine}{[뱀파이어의 정신 침식]} 또는\textcolor{Rhodamine} {[뱀파이어의 정신 침범]} 부상이 있다면, 부상의 단계가 한 단계 심화됩니다.
		
		또는, 전투의 난이도를 낮추기 위해 상대가 의지만으로 해당 주술을 떨쳐낼 수 있을지 대결할 수 있습니다. 이 경우, 뱀파이어가 승리했을 때에야 부상의 단계가 심화됩니다.}
		
		\entry{뱀파이어는 \textcolor{Magenta}{[뱀파이어의 정신 지배]} 부상을 주는데에 성공했다면, 다음 부상을 받기 전까지 이야기꾼을 내려다보며 웃는 행동 이외의 행동을 하지 않습니다.}
		
		\entry{\textcolor{Magenta}{[뱀파이어의 정신 지배]} 부상이 있는 대상에게는 매 턴 위와 같이 의지로 떨쳐내기 대결을 진행합니다. 뱀파이어가 승리한다면, \textcolor{Red}{[뱀파이어 로드에게 바치는 충성]} 작은 부상을 무제한으로 누적하여 획득합니다.}
		
		\entry[\solidsepline]{영창에 성공했고, \textcolor{Red}{[뱀파이어 로드에게 바치는 충성]} 부상이 하나 이상 있다면, 위와 같이 의지로 떨쳐내기 대결을 진행합니다. 성공했다면, 해당 부상 중 하나의 단계가 한 단계 심화됩니다.}
		
		\subject{\hypertarget{bloodsucking-bat}{흡혈 박쥐}}
		
		\positive{한 몸의 박쥐 군단}
		
		\entry[\solidsepline]{이 형태에 있는 동안, \storyref{bloody:holy-water}{성수}를 흩뿌리는 형태의 공격이 아닌 공격으로 피해를 받을 수 없지만, 피해를 주는 것 역시 불가능합니다.}
		
		\subject{\hypertarget{huge-wolf}{거대 늑대}}
		
		\positive{야생의 힘}
		
		\positive{육탄전 전문가}
		
		\entry{이 형태에서만 사용할 수 있는 세 개의 임시 부상칸을 얻습니다. 이는 다른 형태로 바뀌었다가 다시 늑대 형태로 돌아와도 똑같은 상태로 유지됩니다.}
		
		\entry{\storyref{existence-of-the-fallen}{타락의 존재}로 인해 부상칸이 파괴될 때 늑대 형태라면, 추가된 임시 부상칸이 먼저 파괴되며, 이로 인해 임시 부상칸이 모두 파괴된다면 \storyref{huge-wolf}{거대 늑대} 이야기의 도움을 더 이상 받을 수 없습니다.}
		
		\entry{늑대 형태일때도 \storyref{fangs}{날카로운 송곳니}의 도움을 받을 수 있고, \storyref{bloodthirstiness}{흡혈의 주술}을 사용할 수 있습니다. 만약 늑대 형태로 인해 추가된 임시 부상칸이 차 있다면 추가되지 않은 부상칸 한 칸에 더해 임시 부상칸들 중 한 칸을 추가로 비웁니다.}
	\end{lite}
	
	\medskip
	
	전투 중, 영주가 \storyref{mind-control}{정신의 주술}을 영창하는데에 성공한다면 달이 점점 붉어진다는 연출을 해주세요.
	
	\storyref{bloody:holy-water}{성수} 또는 \storyref{bloody:cross}{십자가}는 전투 중 한번 사용하면 파괴된다는 점을 기억하세요.
	
	만약 비행을 할 수 있는 이야기꾼을 상대한다면, \storyref{bloodsucking-bat}{흡혈 박쥐} 이야기를 본래대로 사용하여 이야기꾼의 시야를 가려 움직임을 제한한 뒤, 이야기꾼의 등 뒤에 탄 채로 목을 무는 연출을 사용하는 것을 권장합니다. 이 때에 영주로 인해 비행 능력에 제약이 걸리게 하는 것은 권장하지 않습니다.
	
	어떠한 이유로든, 뱀파이어로부터 최소한 한 번의 부상\footnote{꿈을 꾸면서 얻은 \textcolor{RubineRed}{[목의 위화감]} 역시 이 부상으로 취급합니다.}을 입은 뒤, 뱀파이어의 \storyref{mind-control}{정신의 주술}이 성공하여, 이야기꾼에게 \textcolor{Magenta}{[뱀파이어의 정신 지배]}를 부여하는데에 성공한 것으로 취급합니다. 이야기꾼은 영주의 주술의 힘으로 인해 이야기에서의 추방이 일시적으로 거부되어, 아래의 엔딩을 따릅니다.
	
	\medskip
	
	뱀파이어가 안개가 되어 흩어지게 했다면, 뱀파이어를 충분히 약화시키는데에 성공한 것입니다. 다음날, 뱀파이어들은 모두 영주의 성 안에서 칩거에 들어갑니다. 뱀파이어들은 앞으로 한동안은 이 마을을 괴롭히지 못할 것입니다. 서술이 없는 이야기 [뱀파이어 퇴치 전문가]를 얻습니다.
	
	만약 이야기꾼이 \textcolor{Magenta}{[뱀파이어의 정신 지배]} 부상을 받았다면, 뱀파이어가 마지막 영창을 하거나, 뱀파이어가 안개가 되흩어지게 한 동시에 이야기꾼의 정신이 혼미해지며 이야기꾼의 정신을 지배합니다. 이야기꾼은 달이 완전히 붉게 물드는 것을 보고, 자신의 것이 된 이야기꾼을 이용해 더 넓은 곳으로 이 저주를 퍼트릴 씨앗이 될 것이라고 하며 자신의 세상을 만들겠다는 뱀파이어의 목소리를 들으며 정신이 점점 더 멀리로 날아가기 시작하고, 이윽고 기절합니다. 다음날 아침에 이야기꾼은 여관에서 어젯밤 일에 대한 기억은 전혀 없는 채로 깨어납니다. 이야기꾼의 목에 생긴 두 개의 빨간 점은 이 일을 증명해주지만요. 이 이야기 속에서 뱀파이어의 능력과 관련하여 얻은 모든 부상이 후유증으로 다음 이야기로 남아있게 됩니다:
	
	\begin{lite}[vamp-master]{뱀파이어의 정신 지배}
		\negative{뱀파이어 로드에게 바치는 무의식적 충성}
	\end{lite}
	
	
\end{document}