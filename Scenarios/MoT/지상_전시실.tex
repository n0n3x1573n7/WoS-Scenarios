\documentclass{report}

\begin{document}
	
	\section*{지상 전시실 지도}
	\begin{tabular}{!{\color{black}\vrule}p{3cm}!{\color{black}\vrule}p{3cm}!{\color{black}\vrule}p{3cm}!{\color{black}\vrule}p{3cm}!{\color{black}\vrule}p{3cm}!{\color{black}\vrule}p{3cm}!{\color{black}\vrule}}
		\hline
		\multirow{5}{*}{로비} & \multicolumn{2}{p{3cm}!{\color{black}\vrule}}{\multirow{2}{*}{보석 전시실}} & \multicolumn{2}{p{3cm}!{\color{black}\vrule}}{\multirow{2}{*}{마법 전시실}} & \multirow{5}{*}{보안실} \\
		& \multicolumn{2}{p{3cm}!{\color{black}\vrule}}{}                        & \multicolumn{2}{p{3cm}!{\color{black}\vrule}}{}                        &                      \\ \cline{2-5}
		& \multicolumn{4}{p{6cm}!{\color{black}\vrule}}{복도}                                                                     &                      \\ \cline{2-5}
		& \multicolumn{2}{p{3cm}!{\color{black}\vrule}}{\multirow{2}{*}{기술 전시실}} & \multicolumn{2}{p{3cm}!{\color{black}\vrule}}{\multirow{2}{*}{종교 전시실}} &                      \\
		& \multicolumn{2}{p{3cm}!{\color{black}\vrule}}{}                        & \multicolumn{2}{c!{\color{black}\vrule}}{}                        &                      \\ \hline
	\end{tabular}
	
	\bigskip
	
	박물관의 네 가지 전시실에는 각각 네 가지씩의 아티팩트들이 특별 전시품으로서 전시되어 있습니다. 이 아티팩트들은 이야기꾼에 따라 변경하는 것을 권장하며, 이야기꾼들이 \emph{사용하고 싶도록} 만들어야 합니다. 예를 들어, 이야기꾼들의 능력의 페널티를 상쇄시킨다거나 하는 식으로요. 아래의 표에는 존재할만한 아티팩트들을 나열해두었습니다.
	
	\section*{보석 전시실}
	\begin{tabularx}{\textwidth}{l!{\color{black}\vrule}l!{\color{black}\vrule}X!{\color{black}\vrule}l!{\color{black}\vrule}l!{\color{black}\vrule}l}
		\textbf{속성} & \textbf{명칭} & \textbf{능력} & \textbf{스탯 +} & \textbf{스탯 -} & \textbf{코스트}\\ \hline \hline
		[저주][보석]& 루비   & 소유자는 한 턴에 한 번, 개연성을 1 소모하고 한 구역 내의 모든 대상에게 회피 불가능의 물리 또는 정신 피해를 1 줄 수 있다.   & 자본     & & 0    \\ \hline
		[저주][보석]& 사파이어   & 소유자는 한 턴에 한 번, 개연성을 1 소모하고 한 구역 내의 모든 대상에게 물리적 상태 [감전됨 □] 또는 정신적 상태 [멍해짐 □]을 줄 수 있다.   & 자본     & & 0    \\ \hline
		[저주][보석]& 오팔   &  소유자는 한 턴에 한 번, 개연성을 1 소모하고 한 구역 내의 자신을 제외한 모든 대상의 체력 또는 정신력을 1 회복시킬 수 있다. 개연성은 회복시킬 수 없다.  & 자본     & & 0    \\ \hline
		[저주][보석]& 에메랄드   & 소유자는 한 턴에 한 번, 개연성을 1 소모하고 이동을 1회 추가로 할 수 있다.   & 자본     & & 0    \\ 
	\end{tabularx}
	
	\section*{마법 전시실}
	\begin{tabularx}{\textwidth}{l!{\color{black}\vrule}l!{\color{black}\vrule}X!{\color{black}\vrule}l!{\color{black}\vrule}l!{\color{black}\vrule}l}
		\textbf{속성} & \textbf{명칭} & \textbf{능력} & \textbf{스탯 +} & \textbf{스탯 -} & \textbf{코스트} \\ \hline \hline
		[마법][마나]& 불안정한 수정 & 마나를 사용한다면, 최대 마나가 10\% 증가한다. 이 아티팩트를 공중으로 던지면 폭발하여 자신 외의 같은 구역 안에 있는 모든 이에게 [실명됨: 1턴]을 준다. &  & & 0 \\ \hline
		[마법][목걸이]& 예지의 목걸이 & 착용자는 회피와 조준 판정에 +1을 받는다. 이 목걸이를 파괴함으로서 자동 성공을 결과로 가질 수 있다. &  & & 0 \\ \hline
		[흑마법][혈액] & 응고된 혈액 & 매 턴 정신력 1을 소모한다. 정신력이 0이 되면 이 아티팩트는 영구히 소실된다. &  & & -10 \\ \hline
		[마법][시계] & 시간의 회중시계 & 자신의 턴에 주사위에 의한 판정([행운] 등)을 할 때, 두 번 굴려 그 중 하나를 선택할 수 있다.& 속도 & & 0 \\ \hline
		[마법][목걸이]& 민첩의 목걸이 & 착용자는 회피 판정이나 조준 판정을 함에 있어 +1을 받는다. &  & & 0 \\
	\end{tabularx}
	
	\section*{기술 전시실}
	\begin{tabularx}{\textwidth}{l!{\color{black}\vrule}l!{\color{black}\vrule}X!{\color{black}\vrule}l!{\color{black}\vrule}l!{\color{black}\vrule}l}
		\textbf{속성} & \textbf{명칭} & \textbf{능력} & \textbf{스탯 +} & \textbf{스탯 -} & \textbf{코스트}\\ \hline \hline
		[기술][생물]& 기계 공생체 & 한 턴을 소모해 혈액에 심을 수 있다.\newline 심기면 훔칠 수 없어지며, 이동을 포기하면 보호막 3을 얻을 수 있고, 다음 스탯에 변화를 준다: \newline \textbf{스탯+}: 기민, 근력 \newline \textbf{스탯-}: 의지, 공감, 인식  &   &     & 0 \\ \hline
		[기술][환상]& 테서렉트 & 누군가 개연성 판정에 실패할 때, 테서렉트의 코스트가 2 증가한다. \newline 테서렉트의 코스트가 0이 되면 테서렉트가 폭발하며 소유자를 제외한 이들의 시간이 잠시 멈춘다. 즉시 한 턴을 진행한다. &  &      & -10 \\ \hline
		[기술][안정]& 댐퍼 & 자신의 턴이 종료될 때, 4df를 굴려 해당 값의 절대값에 1을 뺀 만큼의 개연성을 회복할 수 있다.  &  &      & 0 \\ \hline
		[기술][무기]& 죽음의 키스 & 단 한 번 발사할 수 있는 저격총. 사격 또는 사격:총기의 두 배 중 높은 쪽으로 판정하고, 인식과 기민 중 낮은 쪽으로 회피한다. 적중한다면, 해당 적의 체력을 1 남기고 모두 잃게 한다.  &  &      & 0 \\ \hline
		[기술][무기]& 레이저 건 & 턴당 한 번, 시야가 확보된 대상에게 회피 불가능한 피해 1을 주는 레이저를 발사한다. &  & & 0\\
	\end{tabularx}
	
	\section*{종교 전시실}
	\begin{tabularx}{\textwidth}{l!{\color{black}\vrule}l!{\color{black}\vrule}X!{\color{black}\vrule}l!{\color{black}\vrule}l!{\color{black}\vrule}l}
		\textbf{속성} & \textbf{명칭} & \textbf{능력} & \textbf{스탯 +} & \textbf{스탯 -} & \textbf{코스트}\\ \hline \hline
		[신성][십자가]& 순교자의 십자가 & 십자가를 소유한 상태로 이야기가 봉쇄되면, 방어막 3을 얻는다. &  & & 0 \\ \hline
		[신성][묵주]& 대주교의 묵주 & 묵주를 소유한 상태로 이야기가 봉쇄되면, 자신을 포함한 한 대상의 체력 2를 회복시킨다. &  & & 0 \\ \hline
		[신성][기도]& 성기사의 방패 & 구역 내에서 방패를 들고 무릎을 꿇은 채로 정신을 집중하고 있는 동안, 해당 구역에서 나갈수도 들어올 수도 없는 방벽이 생성된다. 이 방벽은 정신집중을 해제하거나, 안팎을 통틀어 10의 피해를 받으면 사라진다.  &  & & 0 \\ \hline
		[신성][토템]& 대정령의 토템 & 한 턴을 소모해 토템을 설치하거나 철거할 수 있다. 설치된 상태에서 같은 구역에 있는 모든 이들은 체력 1을 정신력 1, 또는 정신력 1을 체력 1으로 바꿀 수 있다. &  & & 0 \\
	\end{tabularx}
	
	\section{특별 전시}
	위와 같이 했을 때 전시실당 전시물품 수가 너무 많다고 생각하는 경우, 전시실당 특별 전시 물품 수를 하나씩으로 하여 특정 테마의 특별전을 열 수 있습니다. 아래는 \href{https://twitter.com/knock_tr}{소낙}님의 개변으로, ``빛"을 테마로 한 특별전의 예시입니다.
	
	\subsection*{1. 빛: 지성체의 시작}
	{\storyfont\Large 지성체의 손에 들린 빛은 곧 이들의 힘이자 무기가 됩니다.}
	
	인간들이 빛을 이용해 만든 물품들. 화로, 등불이나 전구나 빛을 이용해 화면을 출력하는 디스플레이 등 빛과 열을 발생시키는 물품들이 전시되어 있습니다.
	
	\begin{story}{빛의 검}{[빛]}
		\entry{손잡이를 제외한  검신이 빛으로 이루어진 검. 같은 구역 또는 인접한 구역의 대상을 지정하여 2의 피해를 입힌다.}
		\flavour{전장을 섬광처럼 누비던 검사가 사용하던 것.}
	\end{story}
	
	\subsection*{2. 빛: 지성체의 길잡이}
	{\storyfont\Large 거의 모든 문화권에서 항성, 빛의 존재는 단순한 광원으로서의 기능을 넘어, 지성체의 삶을 인도하는 존재였습니다. 지성체는 태양의 움직임을 시간의 지침으로 삼으며, 별의 위치와 움직임을 통해 방향을 알아내고 항해술을 개발했습니다.}
	
	천체를 이용해 시간과 방향을 알아내는 시계와 나침반 등이 있습니다.
	
	\begin{story}{달빛 시계}{[빛]}
		\entry{방금 굴린 주사위에서 원하는 주사위를 택해 원하는 결과로 바꾼다. 한 세션에서 두 번 사용할 수 있다.}
		\flavour{은처럼 보이는 금속으로 섬세하게 만들어진 시계. 달을 숭배하는 문화권에서 만들어졌다. 변화하는 달의 위상과 월령에 맞추어 시간을 알아낼 수 있다. 경제적 측면에서 이와 같은 기술을 개발하는 것은 효율적이지는 못하나, 이는 실용성을 위해 만들어진 것이라기보다는 예술적인 기예에 가깝다.
		
		당신이 관조하고자 한다면 상관없지만, 나아가고자 한다면 적절한 때를 알아야 할 것이다.}
	\end{story}
	
	\subsection*{3. 빛 : 신앙과 종교}
	{\storyfont\Large 빛은 많은 문화권에서 신앙의 대상이기도 했습니다. 빛은 이들 지성체에게 눈으로 보이는 길 뿐만 아니라 영적인 길을 인도하고 나아갈 지침이 되기도 합니다.}
	
	태양신과 빛의 신을 기리는 종교적 물품들이 전시되어 있습니다.
	
	\begin{story}{광명을 비추는 손}{[빛]}
		\entry{인지하는 대상을 하나 지정하여 행운 판정을 한다. 성공시 2턴, 실패시 1턴 동안 [기절]상태를 부여한다.}
		\flavour{투명한 구슬을 길게 꿰어 만든 묵주. 태양빛을 상징하는 오브젝트가 걸려 있다. 본래 기도할 때 사용하는 것이지만 성직자가 몸을 지키기 위해 사용하는 주술의 매개체로서 이용되었다.}
	\end{story}
	
	\subsection*{4. 빛의 그림자 : 어둠}
	{\storyfont\Large 빛과 어둠은 뗄레야 뗄 수 없는 관계입니다. 빛을 중심으로 발전한 문화권에서 어둠은 두려움의 대상이자 휴식과 사색의 영역이었습니다.}
	
	한층 더 어두운 전시실. 빛을 피하기 위해 만들어진 물품이나 수면과 관련된 물품 또는 모형이 전시되어 있습니다.
	
	특별 전시 물품은 잠든 것 처럼 눈을 감고 몸을 웅크린 채로 전시장 안에 둥둥 떠 있는 검은 동물. 귀가 길고 고양이나 강아지 정도의 크기를 하고 있으며, 털이 푹신한 꼬리가 천천히 흔들립니다. 그 움직임을 자세히 보면 진짜 털이 아니라 그런 모습을 하고 있을 뿐인 다른 물질이라는 것을 알 수 있습니다.
	
	\begin{story}{드리우는 어둠}{[빛]}
		\entry{받는 피해 2 감소}
		\flavour{어둠의 권능에서 영감을 받아 만들어진 마도구. 착용하지 않을 때에는 짐승의 형상을 하고 지켜야 하는 대상의 주위를 맴돌며 경계한다.}
	\end{story}
\end{document}