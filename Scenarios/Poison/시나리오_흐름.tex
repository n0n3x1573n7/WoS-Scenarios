\documentclass{report}

\begin{document}
	\section{진행하며 주의할 점}
	공개되지 않은 사실은 진실이 아닌 것으로 간주됩니다. 밝혀지지 않은 사실에 대해 진행 중에 이야기꾼에게 언급해 혼선을 빚지 않도록 주의해주세요.
	
	\section{태초의 이야기}
	이야기꾼은 눈을 뜹니다. 여기는 어디죠? 아무것도 없는 공간 상에서, 눈 앞에 문이 하나 보일 뿐입니다.
	
	이야기꾼이 문을 열고 들어가자, 적당한 크기의 방 하나가 나옵니다. 방 안에는 이야기꾼이 좋아하는 음료 두 잔이 준비되어 있습니다. NPC는 이야기꾼에게 음료 한 잔을 내밀며 권합니다.
	
	이야기꾼이 그 음료를 마시려는 순간, NPC는 ``그 잔에 독을 탔다"고 말합니다. NPC는 모종의 일로 이야기꾼에게 원한을 가지고 있었고, 그 때문에 지금 독살하려고 하는 것이라고 설명합니다.
	
	이 원한의 이유는 살인까지 벌이기에는 조금 미묘한 내용으로 설정해주세요.
	
	이야기꾼은 NPC가 장난을 치고 있다고 생각한다면, NPC의 표정 변화는 없고, 매우 진지해 보인다고 언급해주세요.
	
	NPC에게 다음 이야기를 지급합니다.
	
	\begin{lite}[poison:poison-in-yours]{네 잔에 독을 탔어.}
		\negative{너 때문에.}
		\negative{어떻게든 네가 이 잔을 들게 하겠어.}
	\end{lite}
	
	이야기꾼이 태초의 이야기에서 시작했다면 다음 이야기를 지급합니다.
	
	\begin{lite}[poison:too-deep-inside]{심각한 몰입}
		\flavour{여기가 마치 이야기 속이 아니라 현실만 같이 느껴져.}
		
		\entry{역할에 깊게 이입한 상태. 마치 원래부터 이 이야기에 속했던 것처럼, 이 세계와 당신의 역할을 당신이 처한 현실로 여긴다. `깨달은 자'로서의 정체성과 정보는 기억나지 않거나 와 닿지 않는다.}
	\end{lite}
	
	이야기꾼이 어디에서 시작했든, 다음 이야기를 지급합니다.
	
	\begin{lite}[poison:poison-in-mine]{잔에 독을 탔다고...?}
		\negative{NPC가 권한 잔이긴 하지만...}
		\negative{독이 든 잔은 마시기 싫어.}
	\end{lite}
	
	\storyref{poison:too-deep-inside}{심각한 몰입} 이야기 때문인지, 이 이야기 속으로 왜 들어왔는지, 어떻게 들어오게 되었는지가 가물가물합니다. 애초에, 여기가 이야기 속이 맞는지도 잘 모르겟네요. 그런데 그 전에, 이야기가 뭐죠?
	
	\section{잔을 권하는 NPC}
	
	방 안은 극도로 평범해보입니다.
	
	NPC는 이야기꾼에게 이 이야기를 끝낼 방법은 단 한 가지, 두 개의 잔을 누군가 모두 비우는 것 뿐이라고 이야기합니다.
	
	이야기꾼이 아래 내용들이 아닌 방 안을 조사하면 정말 평범하게 아무것도 없는 방이라는 사실을 알려주세요. 방 안에서 눈에 띌 만한것은 NPC, NPC와 이야기꾼의 앞에 놓여 있는 두 개의 잔, 그리고 NPC의 뒤에 있는 거울정도가 다입니다.
	
	모든 판정은 기본적으로 5 이상이 나오면 성공합니다. 하지만, 시간이 많아지고 판정을 많이 할수록 NPC의 마음은 초조해져만 갑니다. 모든 판정을 한 후, NPC는 빨리 잔을 들어 마시라는 식으로 독촉을 합니다. 또한, 이렇게 독촉을 받은 이후로 판정을 한 번 할 때 마다 성공치가 1씩 증가합니다. 즉, 두 번째 판정에서는 6, 세 번째 판정에서는 7, 이런 식으로 증가하는거죠.
	
	\subsection{NPC}
	
	이야기꾼이 NPC의 마음을 떠본다면, 판정을 해주세요. 다른 방법으로 마음을 떠본다면 여러 번 판정을 할 수 있게 해도 괜찮습니다.
	
	이야기꾼이 판정에서 성공한다면, 사실 NPC가 PC가 죽기를 바라지 않는다는 사실을 알 수 있습니다.
	
	\storyref{poison:poison-in-yours}{네 잔에 독을 탔어.} 이야기가 다음 이야기로 바뀝니다:
	
	\begin{lite}[poison:dont-die]{제발.}
		\negative{이게 연극이라는 사실을 몰랐더라면.}
	\end{lite}
	
	\subsection{음료 잔}
	
	이야기꾼이 음료 잔을 자세히 살펴보려고 한다면, 판정을 해주세요. 관찰의 방법이 다르다면, 여러 번 판정을 할 수 있게 해도 괜찮습니다.
	
	이야기꾼이 판정을 성공한다면, 어떤 이유에서인지는 몰라도 독이 NPC의 잔에 들어있다는 사실을 알게 됩니다!
	
	이야기꾼의 \storyref{poison:poison-in-mine}{잔에 독을 탔다고...?} 이야기가 다음과 같이 변합니다:
	
	\begin{lite}[poison:poison-in-theirs]{잔에 독을 탔구나.}
		\negative{NPC의 잔에 든 독...}
		\negative{독이 든 잔은 마시기 싫어.}
	\end{lite}
	
	이 내용이 추가되기 전까지는 이야기꾼의 잔에 독이 있는 것으로 취급합니다. 하지만, 이 내용이 추가되는 순간 NPC의 잔에 독이 있는 것으로 취급됩니다.
	
	이 내용이 추가되기 전, NPC의 \storyref{poison:poison-in-yours}{네 잔에 독을 탔어.} 이야기가 \storyref{poison:dont-die}{제발.} 이야기로 바뀌었다면 독은 일부러 NPC의 잔에 넣은 것입니다. 하지만 바뀌기 전이라면 실수로 NPC와 잔이 바뀐 것입니다.
	
	이 내용을 NPC에게 RP로 전달할 수 있습니다.
	
	\subsection{공간}
	
	이야기꾼이 태초의 이야기에서 시작했다면 \storyref{poison:mirror}{거울}을 조사하기 전 공간 자체를 먼저 조사해야 합니다.
	
	이야기꾼이 이 공간 자체에 대해 궁금해하여 조사하려고 하면, 판정을 해주세요. 이는 NPC를 통해서, 또는 직접 조사 등으로 인해 여러 번의 판정이 가능합니다. 또는 이야기꾼이 질문했을 때 NPC가 그냥 답해주어도 무방합니다.
	
	이 이야기는 NPC가 수상한 작가와의 거래로 접근 권한을 얻은 이야기로, ``진실의 방"이라고 불리는, 진실만을 보여주는 서사죠. 이를 통해 진실을 알고자 하는 것이 NPC의 목적입니다.
	
	이를 알게 된 이야기꾼은 무언가 위화감을 느끼게 됩니다.
	
	\hypertarget{poison:mirror}{}
	\subsubsection{거울}
	
	이야기꾼이 이 공간에 대해 알게 된 이후, 거울이나 반사되는 표면을 들여다보려고 하면, 판정을 해주세요. 이 역시 여러 번의 판정이 가능하게 할 수 있습니다.
	
	이야기꾼이 판정에 성공한다면, 거울에 비치는 자신의 모습이 이 세상의 것이 아니라는 사실을 알게 됩니다. 다음 이야기를 얻습니다:
	
	\begin{lite}[poison:not-my-world]{이 세상은 내 세상이 아니야.}
		\negative{나는 이미 죽은게 아닐까.}
		\neutral{독이 든 걸 먹더라도 나는 멀쩡할거야.}
	\end{lite}
	
	이야기꾼은 얼마 전에 출신 서사에서의 죽음을 맞이했습니다. 하지만 어떤 이유에서인지 다시 살아나 돌아다니고 있고, 그 사실을 눈치챈 NPC는 이야기꾼이 정말 본인인지 의심하고 있습니다.
	
	자신이 죽었다는 사실을 깨달은 이야기꾼은 그 사실을 처음에는 부정하겠지만, 지금까지 이 서사에서 겪은 위화감을 떠올리자 아무래도 자신은 죽은 것처럼 느껴지고, 그 사실을 받아들일 수 밖에 없습니다.
	
	\subsection{이 이외의 것들}
	이 이외의 것들을 조사할 때에도 적당한 판정 등을 할 수 있게 하는 것이 좋습니다. 단, 이 판정들은 판정의 난이도를 올리지 않습니다. 조사를 하면서 진행이 너무 느려진다면, 2d6을 굴려서 아래 표대로 상황을 설정하는 것도 괜찮습니다.
	
	\smallskip
	
	\begin{tabularx}{\linewidth}{c!{\color{black}\vrule}X}
		2d6&상황\\\hline\hline
		
		2&NPC와 마주 앉은 테이블. NPC는 입을 굳게 다물고 있다... 뭘 뜸들이는 거야?\\\hline
		
		3&NPC와 마주 앉은 테이블. 왜 불러 놓고 말이 없는 거야? 이 침묵이 어색하다...\\\hline
		
		4&NPC와 마주 앉은 테이블. 앉은 자리가 딱딱하다. 좀 더 편한 의자는 없었나?\\\hline
		
		5&NPC와 마주 앉은 테이블. 음료에 전등의 빛이 반사되어 반짝거리고 있다...\\\hline
		
		6&NPC와 마주 앉은 테이블. NPC는 아까부터 테이블을 톡 톡 건드리고 있다.\\\hline
		
		7&NPC와 마주 앉은 테이블. 어우, 목말라. ...무심코 테이블 위의 잔을 집어들 뻔 했다!\\\hline
		
		8&NPC와 마주 앉은 테이블. 이웃집으로 추정되는 곳에서 사람들이 떠드는 소리가  들리는 것 같다.\\\hline
		
		9&NPC와 마주 앉은 테이블. 당신은 집으로 돌아갈까 진지하게 고민하고 있다...\\\hline
		
		10&NPC와 마주 앉은 테이블. NPC와 눈이 마주쳤다. 그는 당신의 시선을 피한다.\\\hline
		
		11&NPC와 마주 앉은 테이블. 잔에는 입도 대지 않았는데 속이 울렁거리는 것 같다.\\\hline
		
		12&NPC와 마주 앉은 테이블. 당신은 딴생각을 하기 시작했다
	\end{tabularx}
	
	\section{잔을 집어든 이야기꾼}
	
	어떠한 경우에도, 이야기꾼이 다시 한번 잔 하나를 집어들어 마시려고 할 때, 더 이상의 조사가 불가능하다는 점을 반드시 상기시켜주세요. 그래도 마시기로 선택한다면, NPC의 입으로 이야기꾼에게 마시기 전, 왜 그 잔을 선택했는지, 왜 독이 든 잔, 또는 들지 않은 잔을 마시고 싶은지에 대해 물어봐주세요. 이 대답(특히 어떤 잔을 왜 마시고자 하는지 그 의도)에 따라 결과가 달라집니다:
	
	\subsection{독이 든 잔을 마시고 싶지 않다고 대답한 경우}
		NPC는 힘으로라도 제압해 잔을 마시지 않겠다는 이야기꾼에게 독이 든 잔을 마시게 하려고 합니다. 이에 저항한다면, 전투가 발생합니다.
	
	\subsection{독이 든 잔을 마시고 싶다고 대답한 경우}
		이야기꾼이 \storyref{poison:not-my-world}{이 세상은 내 세상이 아니야.}를 얻었는가에 따라 다음과 같은 결과가 발생합니다:
		
		\subsubsection{[이 세상은 내 세상이 아니야.]를 얻지 못한 경우}
			NPC가 이야기꾼을 대신하여 독이 든 잔을 마시려고 합니다. 이를 막고자 한다면, 전투가 발생합니다.
	
		\subsubsection{[이 세상은 내 세상이 아니야.]를 얻은 경우}
			잔을 마시려는 순간, 이야기꾼의 몸에 깃들어 있던 [악령의 의지]가 등장합니다. [악령의 의지]는 이야기꾼의 목숨을 위협하는 NPC에게 적의를 드러냅니다. NPC는 그 자리에서 까무러치며, 이야기꾼은 NPC를 지키기 위해 [악령의 의지]와 전투를 진행합니다.
			
			만약 NPC를 죽이는 것을 이야기꾼이 방임했다면, [악령의 의지]는 한 몸에 두 자아가 있는 것은 불편하니 너도 끝장내야겠다는 말과 함께, 이야기꾼 역시 공격합니다.
			
			[악령의 의지]는 이야기꾼과 동일한 이야기들을 가지고 있지만, 이야기꾼의 개연성이 7 미만이라면 개연성만큼은 7에서 시작합니다.
	
	\section{전투가 종료되고...}
	전투가 종료되었을때 개연성이 0이 되더라도 추방되지 않습니다. 이야기꾼이 \storyref{poison:not-my-world}{이 세상은 내 세상이 아니야.}를 얻었는가, NPC가 \storyref{poison:dont-die}{제발.}을 얻었는가에 따라 엔딩의 분기가 달라집니다.
	
		\subsection{양쪽 모두 얻지 못한 경우}
			모든 것은 NPC의 장난이었습니다. 이 공간에서는 NPC가 다른 사람들의 도움을 받아 깜짝 파티를 준비하고 있었는데, 이야기 속으로 들어와보니 준비가 덜 되어 있던것을 발견해 시간을 끌고 있었습니다. 다른 공간으로 연결된 문이 열리고, NPC를 도와준 이들이 나와 이야기꾼의 생일이나 다른 이야기 속에서 이루어낸 일 등을 축하해줍니다.
			
		\subsection{한쪽이라도 얻었으며, [악령의 의지]와 전투하지 않은 경우}
			NPC와의 전투에서 승리한다면, 이야기꾼은 마지막으로 원하는 잔을 선택하여 마십니다.
			
			아예 전투가 이루어지지 않거나 전투에서 이야기꾼이 패배한다면, NPC가 원하는대로 이야기가 진행됩니다.
		
		\subsection{[악령의 의지]와 전투한 경우}
			[악령의 의지]와의 전투에서 승리한 경우, [악령의 의지]는 소멸합니다. 이야기꾼은 자신의 존재감이 소멸하는 것을 느낍니다. 이 이야기 속에서는 NPC와 마지막 짧은 대화를 나눌 시간밖에 없겠네요.
			
			[악령의 의지]와의 전투에서 패배한 경우, [악령의 의지]는 사망자를 집어삼킵니다. 다음 이야기를 얻습니다:
			\begin{lite}{악령의 지배}
				\negative{악령이 씌인 사람}
				\negative{악령은 제대로 된 영혼과 한 몸에 존재할 수 없다.}
			\end{lite}
			
\end{document}