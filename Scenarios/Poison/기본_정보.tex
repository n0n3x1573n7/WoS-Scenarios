\documentclass{report}

\begin{document}
	\textbf{시나리오 이름}: 이 잔에 독을 탔다
	
	\textbf{시나리오 작가}: 소낙(\href{https://twitter.com/knock_tr}{@knock\_tr})
	
	\textbf{룰 개변}: None(\href{https://www.twitter.com/n0n3x1573n7_WS}{@n0n3x1573n7\_WS})
	
	\textbf{사용 룰}: 이야기의 방랑자들(Wanderers of the Tales)
	
	\textbf{권장 인원}: 1인
	
	\textbf{비고}: \href{https://fwalker.postype.com/post/3654634}{원본 시나리오}는 inSANe으로 작성되었습니다. 이 개변을 하면서 이야기꾼과 NPC가 모두 이야기꾼이 되었다는 가정 하에 서사를 수정했지만, 원본 시나리오의 서사를 따라가며 이야기의 방랑자들 규칙을 사용하는 방법 역시 가능할 것입니다.
	
	\subsubsection*{시놉시스}
	{\storyfont \Large 당신이 주는 잔을 내가 어떻게 거절할 수 있겠소?}
	
	{\storyfont \Large ...근데 진짜 탄 건 아니지?}
	
	당신은 할 말이 있다는 NPC의 연락을 받고 그의 집으로 찾아갔다. NPC가 내 준 음료에 입을 대려는 순간, NPC는 그 잔에 독을 탔다고 고백한다. 하하, 무슨 농담을...
	
	...진짜야?
	
	\subsubsection*{세션이 시작하기 전에}
		세션이 시작하기 전, 이야기꾼의 시트뿐 아니라, 이야기꾼과 긴밀한 관계를 가진 NPC 이야기꾼의 시트를 이야기꾼과 함께 작성합니다.
\end{document}