\documentclass{report}

\begin{document}
	\section{시스템 파트}
		시스템은 이야기꾼에게 제목이 알려지지 않은 한 서사를 소개합니다. 서사 속에서 중요한 역할을 하는 마석이 행방불명되었는데, 타락한 자들이 서사를 오염시키기 위해 그 마석을 서사 속 변방의 마법사의 거처에 숨겨두었다고 합니다.
		
		시스템은 서사를 정상화시키기 위해 마석을 찾아 달라고 이야기꾼에게 부탁하며, 이를 위해 마법사의 탑에 출입하기에 적절한 역할 이야기를 부여할 것이라고 설명합니다. 이야기꾼이 마석을 찾아낸다면 시스템은 마석을 본디 있어야 할 곳으로 되돌릴 것입니다. 이야기꾼은 [퀘스트 : 암룡의 마석 회수]를 받고 서사 속으로 진입합니다.
	
	\section{서사 파트}
		서사에 입장한 이야기꾼이 유기물로 된 지성체라면 외형은 거의 변하지 않습니다. 아니라면 이야기꾼은 유기물로 된 지성체의 형태 중 원래 이야기꾼의 모습에 가까운 것으로 변합니다. 마법사의 탑의 서재에서 시작합니다.  
		
		세션은 마법사의 탑 내에서 진행됩니다.
		
			\subsection{전투}
				이야기꾼이 [검은 비늘 조각]을 발견하고 탑의 모든 장소의 조사를 끝냈다면 [역할:마법사의 제자]가 [큰 날개의 흑룡]이야기로 바뀝니다. 이야기꾼의 신체는 거대한 용으로 변하며, 적용 범위에 용이 나타났기 때문에 [탑의 봉인]이 활성화됩니다. 마법사는 이야기꾼을 제압하기 위해 전투를 시작합니다.
				
				전투를 포기한다면 [큰 날개의 흑룡]은 효력을 잃고 다시 [마법사의 제자]로 되돌아감을 알려주세요. 또한 [탑의 봉인]또한 공격할 수 있는 대상임을 알려주세요. 마법사는 도발을 사용하여 [탑의 봉인]을 공격하는 것을 방해할 수 있습니다.
				
				마법사를 전투에서 제압하지 못했거나 제압하지 않은 경우, 마법사는 소지하고 있던 [미타아트의 눈물]을 사용해 이야기꾼의 기억을 다시 봉인합니다. [큰 날개의 흑룡]이야기는 그 효력을 잃고 다시 [마법사의 제자]로 돌아가고, 이야기꾼은 서사에서 퇴장하게 됩니다.
				
				마법사를 전투에서 제압했다면 마법사의 몸 속에서 [미타아트의 눈물]이 출현합니다. [미타아트의 눈물]을 습득한 이야기꾼은 서사에서 퇴장할 수 있습니다.
	
	\section{시스템 파트}
		\subsection{엔딩}
			퀘스트를 달성한다면 서사 밖으로 나와 시스템을 만날 수 있습니다. 퀘스트 달성 정도에 따라 다음과 같은 결과가 발생합니다.
			\subsubsection*{메인 퀘스트의 1차  목표 달성}
				미타아트의 눈물을 획득했다면 시스템에게 건네고 보상을 받을 수 있습니다. 미타아트의 눈물을 시스템에게 주지 않는다면 다음과 같은 이야기를 획득합니다.
			\subsubsection*{메인 퀘스트의 이차 목표를 달성하였으나 일차 목표를 달성하지 못함}
				시스템은 이야기꾼의 공로를 치하하며 이야기꾼의 도움으로 마석을 원래 자리에 가져다 놓을 수 있게 되었다고 말합니다. 이야기꾼에게 보상을 지급합니다.
		\subsection{보상}
			보상으로 다음과 같은 이야기 중 하나를 지급할 수 있습니다. 이야기꾼과 협의해도 좋습니다.

\end{document}
