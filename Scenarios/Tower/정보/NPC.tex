\documentclass{report}

\begin{document}
	\subsection*{마법사}
		체력 10/10
		
		사격, 기만, 의지 1\footnote{이야기들의 스탯 증가가 반영된 수치입니다.}
		
		숲 속의 외딴 탑에 기거하는 마법사.

		이 탑에 봉인된 드래곤 [큰 날개의 흑룡]이 세상에 나가지 못하도록 감시하고 있습니다. 이것은 또한 나날이 강해지고 있는 인류로부터 [큰 날개의 흑룡]을 보호하는 것이기도 합니다. 마법사는 동시에 그를 자유롭게 해 주지 못하는 것에 대한 죄책감을 가지고 있습니다. 과거에 [큰 날개의 흑룡]의 제자 또는 피보호자였습니다. 세션에서는 이러한 관계가 뒤바뀐 셈입니다.
		
		전투가 막바지에 이르면 메테오를 사용합니다.
	
		\begin{spoiler}[npc-wizard:wizard]{마법사}{[권능][직업]}
			\flavour{현실과 환상의 경계를 허무는 이능을 부리는 자. PC의 스승이자 이 탑의 주인.}
			
			\entry{마나 50을 가진다.}
			
			\entry[\hline]{\statchange{+}{사격}}
		\end{spoiler}
		
		\begin{spoiler}[npc-wizard:warder]{마탑의 간수}{[생애][권능]}
			\flavour{가장 강력하고 위험한 환상이 두 발과 날개가 묶인 채 여기에 있다. 내가 여기에 있는 한 당신은 저 밖의 누구도 해칠 수 없으며, 저 밖의 누구도 당신을 해칠 수 없다. }
			
			\entry[\hline]{\statchange{+}{기만}}
		\end{spoiler}
		
		\begin{spoiler}[npc-wizard:taunt]{도발}{[요령]}
			\entry[\hline]{대상을 정해 의지 판정을 하게 한다. 만약 자신의 도발이 더 높다면, 다음 턴에는 해당 대상은 반드시 자신을 공격해야 한다.}
		\end{spoiler}
		
		\begin{spoiler}[npc-wizard:armor]{어둠의 가호}{[방어구]}
			\flavour{어둠을 묵도할지언정, 그 그늘 아래 보호받았던 나는 두려워하지 않는다.}
			
			\entry{모든 종류의 피해를 1 경감한다.}
			
			\entry[\hline]{\statchange{+}{의지}}
		\end{spoiler}
		
		\begin{spoiler}[npc-wizard:magic-bullet]{마력탄}{[마법]}
			\entry[\hline]{한 차례 영창하여 마나1을 소모하고 3의 피해를 준다.}
		\end{spoiler}
		
		\begin{spoiler}[npc-wizard:consecutive-magic-bullet]{마력탄 연사}{[마법]}
			\entry[\hline]{마나 3을 소모하여 여러 개의 마력탄을 동시에 발사한다. 사격으로 판정하여 성공하면 나온 수치+2만큼의 피해를 준다.}
		\end{spoiler}
		
		\begin{spoiler}[npc-wizard:meteo]{메테오 스트라이크}{[마법]}
			\entry[\hline]{응축된 거대한 마력 덩어리를 떨어트린다. 두 턴 영창하여 마나 5를 소모하고 10의 피해를 준다. 한 전투에서 한 번만 사용 가능하다.}
		\end{spoiler}
	
	\subsection*{패밀리어}
		체력 10/10
		마법사는 제자를 보호하기 위해 패밀리어를 이야기꾼의 곁에 붙여놓았습니다.
		
		패밀리어는 세션이 진행되는 동안 이야기꾼을 따라다니며 애교를 부리거나, 놀아 주거나, 때로는 위협으로부터 이야기꾼을 대신해 싸우기도 합니다. GM은 패밀리어를 통해 PC에게 힌트를 건네거나 장난을 치는 등의 RP를 할 수 있습니다. 패밀리어의 모습은 어떤 동물도 좋지만 고양이, 햄스터, 도마뱀, 참새 등 작은 동물을 권합니다. 플레이어와 상의하여 선호하는 동물로 정해도 좋습니다.
		
		이야기꾼과 마법사의 전투가 발생하면 마법사에게 돌아가 \storyref{npc-wizard:armor}{어둠의 가호}로 변합니다.
		
		\begin{spoiler}[npc-familiar:familiar]{패밀리어}{[마법]}
			\entry[\hline]{동물의 모습을 한 작은 그림자. 마법사가 부리는 마법생물이다.}
		\end{spoiler}
		
		\begin{spoiler}[npc-familiar:truth]{진실의 수호자}{[임무]}
			\entry[\hline]{당신의 술자는 사랑하는 이를 지킬 것을 명령했다.}
		\end{spoiler}
		
		\begin{spoiler}[npc-familiar:darkness]{어둠}{[성질]}
			\entry[\hline]{모든 종류의 피해 1 경감}
		\end{spoiler}
	
	\subsection*{슬라임}
		체력 5/5
		
		창고의 잡동사니 틈에서 튀어나온 작은 몬스터. 몸이 산성으로 되어 있어 접촉 시 가벼운 화상을 입힙니다. 슬라임에게 상처를 입은 채로 마법사와 마주친다면 마법사가 상처를 치료해줍니다. 박쥐나 뱀 등 작고 위협도가 크지 않은 다른 동물이나 몬스터로 개변해도 좋습니다.
		
		\begin{spoiler}[npc-silme:jump]{튀어오르기}{[행위]}
			\entry[\hline]{아무 일도 일어나지 않는다.}
		\end{spoiler}
		
		\begin{spoiler}[npc-silme:acid]{산성}{[성질]}
			\entry[\hline]{접촉시 가벼운 화상을 입는다. 위협적이진 않으나 따갑다.}
		\end{spoiler}
	
\end{document}
