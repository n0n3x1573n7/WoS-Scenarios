\documentclass{report}

\begin{document}
	
	\begin{story}[quest:magicstone]{암룡의 마석 회수}{[퀘스트]}
		\entry{\textbf{일차 목표} : `암룡의 마석' 습득}
		
		\entry{\textbf{이차 목표} : `암룡의 마석'의 소재 확인}
		
		\entry{\textbf{지급 이야기} : \storyref{role:disciple}{마법사의 제자}}
		
		\entry{\textbf{성공 보상} : 서사의 기여도와 역할에 관련된 이야기 한 개}
		
		\entry[\hline]{\textbf{실패 보상} : -}
	\end{story}
	
	\begin{story}[quset:familiar]{놀아주세요!}{[서브퀘스트]}
		\flavour{패밀리어와 즐거운 시간을 보내면 달성되는 퀘스트입니다. 별다른 판정 없이 RP만으로 수행 가능하며, 어떻게 노는지에 대해 특별한 아이디어가 없다면 패밀리어가 공 등의 장난감을 물고 와서 놀아달라고 조르는 것으로 시작하는 것이 무난합니다.}
		
		\entry{\textbf{목표} : 패밀리어와 놀아주기}
		
		\entry{\textbf{성공 보상} : 패밀리어가 기뻐합니다.}
		
		\entry[\hline]{\textbf{실패 보상} : 패밀리어가 속상해합니다.}
	\end{story}
	
	\begin{story}[quest:book]{책 정리}{[서브퀘스트]}
		\entry{세 개의 서가를 정리해야 합니다. 각 서가당, 4df를 한 번 굴립니다. 2번 이상 0 이상의 판정으로 성공하면 퀘스트에 성공합니다.}
		
		\entry{\textbf{목표} : 서재의 책 정리}
		
		\entry{\textbf{성공 보상} : 
			\begin{story}[state:proudness]{뿌듯함}{[상태]}
				\flavour{만사를 긍정적으로 볼 수 있을 것 같다. 오래 갈 기분은 아니지만, 기분이 좋다는 건 소중한 것이다.}
								
				\entry[\hline]{판정 실패시 재굴림. 이 이야기는 한번 사용하면 소멸한다.}
			\end{story}
		}
		
		\entry[\hline]{\textbf{실패 보상} : 
			\begin{story}[state:stressed]{스트레스}{[상태]}
				\flavour{마치 고통 받기 위해 만들어진 인생 같아.}
				\entry[\hline]{2장면동안 지속된다.}
			\end{story}
		}
	\end{story}
	
\end{document}
