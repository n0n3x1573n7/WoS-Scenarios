\documentclass{report}

\begin{document}
	
	\begin{story}[role:disciple]{마법사의 제자}{[역할][직업]}
		\flavour{나는 제자인가, 잡일꾼인가……. 바쁘다는 핑계로 아무것도 가르쳐 주지 않은 마법사는 이번 논문의 집필이 끝나면 꼭 수업을 시작해 주겠다고 약속했다. “이번에는 진짜죠?”}
		\entry[\hline]{\statchange{+}{의지}}
	\end{story}
	
	서사 속에 진입했을 때 시스템으로부터 주어지는 이야기입니다. 이야기꾼은 본래부터 존재하던 ‘마법사의 제자’자리에 대신 들어갔거나 그에게 빙의했다고 보아도 무방합니다. 아직 마법을 제대로 배우지 못한 수련생이므로 고등한 마법과 관련된 이야기는 사용하기 어렵습니다.

	플레이어가 만약 대학원생이거나 악덕 교수에게 고통 받은 경험이 있다면 이런 설정에 반감을 가지고 마법사를 지나치게 적대시할 수 있습니다. 이 경우 마법사를 불쌍하게 연출하여 동정심을 불러일으키거나 아예 다른 관계성을 제시하는 것을 추천합니다.
	
	\begin{story}[item:fountain-pen]{마법사의 만년필}{[도구]}
		\entry{마도구 전문 브랜드 `타미노펜'에서 출시된 한정 에디션 `깜뿌기불'. 몹시 오래 된 것임에도 불구하고 관리가 잘 되어 있다. 다음과 같은 문장이 각인되어 있다.}
		
		\flavour[\hline]{빛은 길을 비춰줄 뿐, 내딛는 것은 당신의 몫이다.}
	\end{story}
	
	만년필을 자세히 조사하면 \storyref{item:memo-in-pen}{만년필 속 쪽지}를 발견할 수 있습니다.
	
	\begin{story}[item:memo-in-pen]{만년필 속 쪽지}{[서류]}
		\flavour[\hline]{`당신이 잊었어도 괜찮아요. 이제 당신의 곁을 떠나지 않을게요.'}
	\end{story}
	
	\begin{story}[item:teardrop]{미타이트의 눈물}{[도구]}
		\flavour{메인 퀘스트의 목적인 마석입니다.}
		
		\entry[\hline]{강력한 힘을 가진 마석. 인류가 기억하는 가장 오래된 용 미타아트의 이름과 함께 전해져 내려온다. 짙은 그림자와 어둠으로 이루어진 암룡(暗龍)은 자손들과 자녀들을 거느리고 인류와 전쟁을 벌였다고 한다.}
	\end{story}
	
	\begin{story}[magic:tower]{탑의 봉인}{[마법]}
		\flavour{이야기꾼이 공격하여 파괴할 수 있는 대상입니다. 이 사실을 전투가 시작했을 때 알려주세요. 다섯 번 공격당하면 파괴됩니다.}
		
		\entry[\hline]{드래곤을 억압하는 마법. 영역 안에 있는 용은 비행이 불가능하며, 모든 판정에 -1을 받고, 매 차례마다 체력을 1 잃는다.}
	\end{story}
	
	\begin{story}[role:big-wing]{큰 날개의 흑룡}{[역할][종족][권능]}
		\flavour{절멸한 용족의 마지막 후예 중 하나. 예로부터 드래곤은 가장 강력하고 위험한 환상 중 하나였고 재앙으로 불렸다. 인류는 당신을 위협으로 여겨 제압하고 이 탑에 봉인했다. 당신은 거짓된 기억을 주입당해 자신이 사람이라 믿으며 이 곳에 갇혀있었다.}
		
		\entry[\hline]{사격 판정에 성공 시 목표에게 3의 피해를 준다.}
	\end{story}
	
	특정 조건을 만족할 경우 \storyref{role:disciple}{마법사의 제자}는 이 이야기로 변합니다. 이 이야기를 얻었을 때 의지판정(+2 혹은 그 이상일 경우 성공)을 하여 실패하면 3차례 동안 지속되는 상태인 \storyref{state:angry}{격노}를 가집니다. 의지 판정에 대성공했을 경우 \storyref{memoey:even-different}{서로 다르더라도}를 얻습니다.
	
	\begin{story}[state:angry]{격노}{[상태]}
		\entry[\hline]{외부의 자극으로 이성을 잃은 상태. 공격할 수 있는 대상이 있다면 반드시 공격해야 한다.}
	\end{story}
	
	\begin{story}[memoey:even-different]{서로 다르더라도}{[기억]}
		\entry[\hline]{검은 용과 작은 인간은 친구였습니다. 둘은 서로 달랐지만 사랑했습니다. 크고 강한 용은 작은 인간을 지켜주겠다고 약속했습니다.}
	\end{story}

	
\end{document}
