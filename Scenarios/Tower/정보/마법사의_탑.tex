\documentclass{report}

\begin{document}

	마법사의 연구소이자 거주지. 왕국에서 먼 곳에 떨어진 숲 속에 위치하고 있습니다.
	
	숨바꼭질 : 탑 내 장소이동시 4df를 굴립니다. 0과 그 이하의 값이 나오면 입장한 장소에 마법사가 등장합니다. 한 장소에서 마법사가 등장했다면 바로 다음 장소 이동시에는 마법사 등장 여부를 판정하지 않습니다.
	
	구조는 다음과 같습니다. 높은 곳부터 서술합니다.

	\subsection*{탑의 꼭대기}
		탑의 옥상. 탑 주변의 경치가 한눈에 들어옵니다. 자세히 조사한다면 탑 주변에 둘러쳐진 결계를 발견합니다.
	
	\subsection*{서재}
		마법사의 서재이자 연구실. 벽 가득히 책이 꽂혀 있습니다. 책이 엉망진창으로 섞여 있다는 것을 발견할 수 있습니다. 정리하는 것은 분명 이야기꾼의 몫이겠죠. 자세히 조사하면 책들 일부에 어린아이 글씨로 낙서가 되어 있는 것을 발견합니다.
		
		세션을 시작하는 장소로, 마법사로부터 \storyref{item:fountain-pen}{마법사의 만년필}을 받습니다. 원한다면 해당 이야기를 사용하여 논문 작업을 할 수 있습니다.
		
		서재 외의 공간에서 마법사를 만난다면 `왜 여기에 있느냐. 일은 다 했느냐. 그럴 리가 없지.' 라는 등의 말을 하며 이야기꾼이 농땡이를 치고 있다는 사실을 지적합니다. 그러나 크게 탓하거나 돌아가 일하라고 지시하지는 않습니다.
	
	\subsection*{창고}
		다양한 물건들이 분류되지 않은 채로 여기저기 쌓여 있는 공간. 물건들 틈에서 슬라임을 발견할 수 있습니다.
	
	\subsection*{마법사의 방}
		마법사의 침실. 자세히 조사한다면 마법사로 보이는 인물과 누군가가 함께 찍힌 사진을 발견할 수 있습니다. 이 사진은 어린 시절의 마법사와 그 스승-검은 날개의 흑룡-이 함께 찍은 사진으로, 이야기꾼은 자세히 조사하더라도 이 이야기를 알 수 없습니다. 다만 이유를 알 수 없는 친숙함이나 묘한 기시감 등을 느낄 수 있을 뿐입니다.
		
		마법사를 동반하지 않고 방문했을 경우, 문이 마법으로 잠겨 있어 열 수 없습니다. 문을 억지로 열기 위해서는 5 이상의 충격을 주어야 합니다. 마법사는 문이 열린 것을 눈치 채고 1턴 후에 달려오므로, 빠르게 조사를 마치거나 자리를 이탈해야 합니다. 만약 행운 판정에 성공한다면 마법사는 3턴 후에 도착합니다.
		
		마법사에게 현장을 들켰을 경우, 문이 부서진 이유에 대해 마법사에게 납득시키지 못한다면 서사에서 추방당합니다.
	
	\subsection*{정원}
		탑을 중심으로 원형으로 자리한 공터. 걸터앉을 수 있는 잡동사니가 있습니다.
		
		정원을 통해 이야기꾼이 밖으로 나가려고 시도할 경우, 먼저 `탑 밖으로 나가면 마법사가 걱정할 것이다', `마석은 탑 밖이 아니라 안에 숨겨져 있다' 등으로 이야기꾼을 설득하는 쪽을 권합니다. 그럼에도 불구하고 이야기꾼이 나가고자 한다면 뭔가에 가로막힌 듯 숲으로 들어갈 수 없습니다.
	
	\subsection*{정원}
		쇠사슬과 커다란 수갑 등 뭔가를 가두어 놓았던 흔적이 있습니다. 자세히 조사할 경우 \hypertarget{black-scale}{검은 비늘 조각}을 발견할 수 있습니다.
	
\end{document}
