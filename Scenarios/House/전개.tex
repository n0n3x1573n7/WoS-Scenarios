\documentclass{report}

\begin{document}
	이 장이 시작하기 전, [이야기의 사교도들] 규칙에 의해 각 이야기꾼이 1\textasciitilde2개씩의 흐름을 작성합니다.
	
	\begin{multicols}{2}
		
		\begin{flow}{사실 거짓말이었어!}
			\entry{\textbf{흐름}: 질문}
			\entry{\textbf{대상}: 자신을 제외한 아무나}
			\entry{당신이 말한 어떤 정보는 거짓말이었습니다! 자신에 관련된 이야기 하나를 골라, 수정하거나 삭제할 수 있습니다.}
			\entry{이 흐름은 버리지 않고 옆으로 둡니다.}
		\end{flow}
		
		\begin{flow}{저택 안의 방을 하나 고릅니다. \\ 왜 이 방이 당신을 소름돋게 하나요?}
			\entry{\textbf{흐름}: 질문}
			\entry{\textbf{대상}: 아무나}
			\entry{해당 내용에 따른 이야기를 추가합니다.}
			\entry{이 흐름은 버리지 않고 옆으로 둡니다.}
		\end{flow}
		
		\begin{flow}{저택 안의 방을 하나 고릅니다. \\ 왜 당신은 이 방에 흥미를 느끼나요?}
			\entry{\textbf{흐름}: 질문}
			\entry{\textbf{대상}: 아무나}
			\entry{해당 내용에 따른 이야기를 추가합니다.}
			\entry{이 흐름은 버리지 않고 옆으로 둡니다.}
		\end{flow}
		
		\begin{flow}{당신은 이곳저곳 돌아다닙니다...}
			\entry{\textbf{흐름}: 판정 - 성공 난이도 3, 실패 난이도 3}
			\entry{\textbf{대상}: 아무나, 함께 있는 등장인물만 판정에 참여합니다.}
			\entry{[길을 잃음] 이야기가 있다면, 성공 난이도가 5로 증가합니다.}
			\entry{판정에 성공한다면, [길을 잃음] 이야기를 잃거나, 흥미로운 무언가를 하나 찾아 저택에 추가합니다.}
			\entry{[길을 잃음] 이야기를 가진 등장인물이 판정에 실패한다면, 해당 이야기를 잃고 [겁에 질림] 이야기를 얻습니다. 겁에 질린 이유가 무엇인지에 대한 내용을 추가합니다.}
			\entry{이 흐름은 버리지 않고 옆으로 둡니다.}
		\end{flow}
		
		\begin{flow}{저택 안을 혼자 돌아다니다가...}
			\entry{\textbf{흐름}: 분기}
			\entry{\textbf{대상}: [단독행동] 이야기의 소유자. 없다면 이 흐름을 옆에 둡니다.}
			\entry{[단독행동] 이야기를 다음 중 하나로 대체합니다:}
			\entry{\textbf{선택지 1}: [길을 잃음]. 이 이야기를 잃기 전까지 다른 등장인물과 조우할 수 없습니다.}
			\entry{\textbf{선택지 2}: [겁에 질림]. 겁에 질린 이유가 무엇인지에 대한 내용을 추가합니다.}
		\end{flow}
		
		\begin{flow}{당신은 이 집에 대한 기억을 떠올려봅니다.}
			\entry{\textbf{흐름}: 분기}
			\entry{\textbf{대상}: [집과의 인연] 이야기의 소유자.}
			\entry{당신은 이 집에 대한 한 가지 사실을 알고 있습니다. [집에 대한 기억] 이야기를 얻습니다.}
			\entry{이 흐름을 당신 앞에 둡니다. 언제든 이 흐름을 버리고 [서술자]가 될 수 있습니다. 흐름을 버릴 때, [집에 대한 기억] 이야기에 해당 정보에 관한 서술을 추가합니다.}
		\end{flow}
		
		\begin{flow}{관리인이 당신에게 다가와 귓속말합니다.}
			\entry{\textbf{흐름}: 분기}
			
			\entry{\textbf{대상}: [관리인과의 안면] 이야기의 소유자.}
			\entry{관리인은 당신에게 이 집에 대한 한 가지 사실을 알려줍니다. [관리인과의 밀담] 이야기를 얻습니다.}
			\entry{이 흐름을 당신 앞에 둡니다. 언제든 이 흐름을 버리고 [서술자]가 될 수 있습니다. 흐름을 버릴 때, [관리인과의 밀담] 이야기에 해당 정보에 관한 서술을 추가합니다.}
		\end{flow}
	\end{multicols}
	
	\begin{lite}{두 번째 장의 대단원}
		\entry{\textbf{흐름}: 가지}
		\entry{이제부터, 서술자는 흐름을 뽑아 사용하는 대신 다음 흐름을 사용할 수 있습니다. 이는 [흑막] 이야기의 소유자가 없을 때에만 사용할 수 있습니다.
		\begin{flow}{흑막 선언}
			\entry{\textbf{흐름}: 분기}
			\entry{\textbf{대상}: 자신}
			\entry{당신은 이 저택에 일어나는 기현상에 밀접한 관련이 있습니다. [흑막] 이야기를 얻고, 저택에 대한 공포에 관련된 모든 이야기를 잃습니다. 이제 [흑막] 이야기를 가진 동안 당신의 목적은 당신이 아닌 모든 이들을 쫓아내는 것입니다. 어떤 방법으로든 말이죠.}
			\entry{다른 누군가에게 서술자의 역할을 넘깁니다. 이 이후로도 이야기를 계속 진행합니다.}
		\end{flow}}
		\entry{이제부터, 매 장이 끝날때마다 다음 판정을 합니다:
		\begin{flow}{공포로 얼룩진 저택}
			\entry{\textbf{흐름}: 판정 - 성공 난이도 가변, 실패 난이도 3}
			\entry{\textbf{대상}: [흑막]을 제외한 저택 내부의 전원}
			\entry{판정에 실패한다면, 저택에 또 다른 기현상이 발생했습니다. 서사 포화도가 가장 낮은 이야기꾼이 다음 중 한가지 이야기를 얻습니다:}
			\entry{[길을 잃음], [겁에 질림]}
			\entry{같은 이야기를 중복으로 얻는다면, 해당 이야기에 서술을 추가합니다.}
		\end{flow}}
		\entry{어떤 이야기꾼이든, 사망하지 않았고 [흑막]이 아니며, 공포에 관련된 서술이 셋 이상 있다면, 자신이 서술자일 때 다음 흐름을 사용할 수 있습니다:
		
		\begin{flow}{여기는 내가 있을 곳이 아니야.}
			\entry{\textbf{흐름}: 분기}
			\entry{\textbf{대상}: 자신}
			\entry{겁에 질려 저택에서 떠납니다.}
			\entry{여전히 서술자로서 참여할 수는 있지만, 저택 안의 일에 관여할 수는 없습니다.}
			\entry{다른 누군가에게 서술자의 역할을 넘깁니다.}
		\end{flow}}
		
		\entry{하지만 본인이 서술자가 된다면, 다음 흐름을 사용해 저택으로 돌아갈 수 있습니다!
		
		\begin{flow}{다시, 저택으로}
			\entry{\textbf{흐름}: 분기}
			\entry{\textbf{대상}: 자신}
			\entry{당신은 다시 한번 용기를 내어 저택으로 돌아갑니다.}
			\entry{다른 누군가에게 서술자의 역할을 넘깁니다.}
		\end{flow}}
		\entry{옆에 둔 흐름은 다음 장의 흐름과 함께 잘 섞어 사용합니다.}
	\end{lite}
	
	다음 장 부터는 [이야기의 사교도들] 규칙에 의해 각 이야기꾼이 3\textasciitilde4개씩의 흐름을 작성합니다. 더 많이 작성해도 괜찮습니다!
	
	이야기를 마무리할때에는 흑막이 있다면 흑막에게 이야기의 흐름을 맡깁니다. 없다면, 다 같이 결말을 정합니다. 너무 이야기가 늘어지거나 길어진다고 생각한다면 다음 장으로 넘어가는 대신 이렇게 결말을 정해도 괜찮습니다.
\end{document}
