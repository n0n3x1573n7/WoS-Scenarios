\documentclass{report}

\begin{document}
	\begin{multicols}{2}
		\begin{flow}{당신은 왜 이 무서운 저택에 오기로 결정했나요?}
			\entry{\textbf{흐름}: 질문}
			\entry{\textbf{대상}: 아무나}
			\entry{해당 내용에 대한 이야기를 추가합니다.}
		\end{flow}
		
		\begin{flow}{당신은 저택에 대한 사실을 한가지 알고 있습니다. \\ 어떤 사실인가요?}
			\entry{\textbf{흐름}: 질문}
			\entry{\textbf{대상}: 아무나}
			\entry{해당 내용에 대한 이야기를 소재 [언덕 위의 무서운 집]에 추가합니다.}
		\end{flow}
		
		\begin{flow}{당신은 일행 중 누군가와 특별한 인연이 있습니다. \\ 어떤 인연인가요?}
			\entry{\textbf{흐름}: 질문}
			\entry{\textbf{대상}: 아무나}
			\entry{해당 내용에 대한 이야기를 받습니다. 쌍방의 관계라면, 상대 역시 받습니다.}
		\end{flow}
		
		\begin{flow}{당신은 무언가에 대해 공포를 느낍니다. \\ 무엇인가요?}
			\entry{\textbf{흐름}: 질문}
			\entry{\textbf{대상}: 아무나}
			\entry{해당 내용에 대한 이야기를 추가합니다.}
		\end{flow}
		
		\begin{flow}{당신은 아무도 모르게 무언가를 들고 왔습니다. \\ 무엇인가요?}
			\entry{\textbf{흐름}: 질문}
			\entry{\textbf{대상}: 아무나}
			\entry{해당 내용에 따른 이야기를 추가합니다.}
		\end{flow}
		
		\begin{flow}{공포영화의 등장인물로서, 당신의 장점은 무엇인가요?}
			\entry{\textbf{흐름}: 질문}
			\entry{\textbf{대상}: 아무나}
			\entry{해당 내용에 따른 이야기를 추가합니다.}
		\end{flow}
		
		\begin{flow}{공포영화의 등장인물로서, 당신의 단점은 무엇인가요?}
			\entry{\textbf{흐름}: 질문}
			\entry{\textbf{대상}: 아무나}
			\entry{해당 내용에 따른 이야기를 추가합니다.}
		\end{flow}
		
		\begin{flow}{당신은 다른 등장인물과 관련해 알고 있는 비밀이 한 가지 있습니다. \\ 무엇인가요?}
			\entry{\textbf{흐름}: 질문}
			\entry{\textbf{대상}: 자신을 제외한 아무나}
			\entry{해당 내용에 따른 이야기를 추가합니다.}
		\end{flow}
		
		\begin{flow}{당신이 이 역할을 선택한 이유는 무엇인가요?}
			\entry{\textbf{흐름}: 질문}
			\entry{\textbf{대상}: 아무나}
			\entry{해당 내용에 따른 이야기를 추가합니다.}
		\end{flow}
		
		\begin{flow}{당신은 이 역할을 원하지 않았습니다. \\ 왜 그런가요?}
			\entry{\textbf{흐름}: 질문}
			\entry{\textbf{대상}: 아무나}
			\entry{[원치 않은 역할] 이야기를 얻고, 해당 내용에 따른 서술을 추가합니다.}
		\end{flow}
		
		\begin{flow}{저택을 관리하는 이가 있다는 소문이 있습니다. \\ 당신은 저택 관리인을 만난 적이 있나요?}
			\entry{\textbf{흐름}: 분기}
			\entry{\textbf{대상}: 아무나}
			\entry{누군가 ``예"라고 대답하기 전까지 이 흐름을 받지 않은 누군가에게 이 흐름을 넘깁니다.}
			\entry{모든 이야기꾼이 ``아니요" 라고 대답한다면 이 흐름을 옆에 둡니다.}
			\entry{누군가 ``예"라고 대답한다면, [관리인과의 안면] 이야기를 받습니다. 내용은 아직 채우지 않습니다.}
		\end{flow}
		
		\begin{flow}{당신은 갑자기 화장실이 가고 싶습니다.}
			\entry{\textbf{흐름}: 분기 - 선택지}
			\entry{\textbf{대상}: [화장실 갈래] 이야기의 소유자, 없다면 아무나}
			\entry{\textbf{선택지 1}: 저택에 도착하기 전이라면, 양해를 구하고 저택에 도착하기 전에 화장실을 다녀옵니다. 이 흐름을 처리한 후, 버리는 대신 옆에 둡니다.}
			\entry{\textbf{선택지 2}: 화장실을 갑니다. [화장실 갈래] 이야기가 있다면 이 선택지를 반드시 선택해야 하지만 해당 이야기를 잃습니다. [단독행동] 이야기를 얻습니다.}
			\entry{\textbf{선택지 3}: 참습니다. [화장실 갈래] 이야기를 얻고, [메타] 1회를 얻습니다. 이 흐름을 처리한 후, 버리는 대신 옆에 둡니다.}
		\end{flow}

		\begin{flow}{당신은 이 무서운 저택에 와 본 적이 있나요?}
			\entry{\textbf{흐름}: 분기}
			\entry{\textbf{대상}: 아무나}
			\entry{누군가 ``예"라고 대답하기 전까지 이 흐름을 받지 않은 누군가에게 이 흐름을 넘깁니다.}
			\entry{모든 이야기꾼이 ``아니요" 라고 대답한다면 이 흐름을 버립니다.}
			\entry{누군가 ``예"라고 대답한다면, [집과의 인연] 이야기를 받습니다. 내용은 아직 채우지 않습니다.}
		\end{flow}
	\end{multicols}
	
	\begin{lite}{첫 번째 장의 대단원}
		\entry{\textbf{흐름}: 가지}
		\entry{[집과의 인연] 이야기를 받은 이야기꾼이 있다면, 이제 그 인연의 내용을 채워넣습니다.
			
			없다면, \emph{모든} 이야기꾼이 아래 흐름의 적용을 받습니다:
			
			\begin{flow}{저택의 압박감}
				\entry{\textbf{흐름}: 판정 - 성공 난이도 3 / 실패 난이도 3}
				\entry{\textbf{대상}: 전원}
				\entry{저택 앞에 도착하자, 어딘가 압박감이 듭니다. 저택의 소문과 모습이 어우러져 무서운 느낌이 드네요.}
				\entry{판정에 실패한다면, 전원 이야기 [여기 너무 무서워...]를 받습니다.}
			\end{flow}
			}
		\entry{모두들 기대 반 공포 반의 기분으로 저택의 문 앞에 도착했습니다. [NPC: 저택 관리인]을 공개합니다.
			\begin{lite}{NPC: 저택 관리인}
				\entry{오랫만의 손님이군요. 환영합니다.}
				\entry{이상할 정도로 창백한 얼굴}
				\entry{살아는 있는거... 맞지...?}
			\end{lite}
			
			[관리인과의 안면] 이야기를 가진 이야기꾼이 있다면, 이야기꾼의 이름을 부르며 등장인물들을 반깁니다.}
		
		\entry{
		\begin{flow}{이 대저택에는 많은 방이 있습니다. \\ 어떤 방이 있을까요?}
			\entry{\textbf{흐름}: 질문}
			\entry{\textbf{대상}: 전원}
			\entry{해당 내용에 따른 이야기를 [오래된 저택]에 추가합니다.}
		\end{flow}}
		
		\entry{옆에 둔 흐름은 다음 장의 흐름과 함께 잘 섞어 사용합니다.}
	\end{lite}
\end{document}
