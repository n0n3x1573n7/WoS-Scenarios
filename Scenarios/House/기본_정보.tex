\documentclass{report}

\begin{document}
	\textbf{시나리오 이름}: 언덕위의 유령의 집
	
	\textbf{시나리오 작가}: None(\href{https://www.twitter.com/n0n3x1573n7_WS}{@n0n3x1573n7\_WS})
	
	\textbf{사용 룰}: 이야기의 생성자들(Wrights of the Plots)
	
	\textbf{장수 제한}: 5
	
	\textbf{권장 인원}: 3\textasciitilde 4인
	
	\textbf{트리거 워닝}: 이 시나리오는 공포 시나리오로, 호러 분위기가 연출되며, 등장인물이 사망하게 될 수 있습니다.
	
	\subsubsection*{시놉시스}
	
	등장인물들은 언덕 위에 있는 유령의 집에 다 같이 놀러가기로 합니다. B급 공포영화에서나 나올법한 집이네요! ...잠깐, 이거 세계관이 어디죠? B급 공포영화라고요?
	
	\subsubsection*{진행 중 참고사항}
	초반 일부만이 작성된 시나리오입니다. 해당 내용 이후의 내용은 [이야기의 사교도들]을 사용해서 채워넣어 주세요.
	
\end{document}
