\documentclass[11pt]{report}
%Use \documentclass[12pt]{book} to printout and make it into a book
\renewcommand{\numberline}[1]{#1~}

\newif\ifDLC
\DLCfalse %\DLCtrue or \DLCfalse

\newif\iffullchangelog
\fullchangelogfalse %\fullchangelogtrue or \fullchangelogfalse

\newcommand{\version}{0.0.0}

%\usepackage{kotex}
\usepackage{CJKutf8}

\usepackage[utf8]{inputenc}
\usepackage[OT1, T2A]{fontenc}

\usepackage{imakeidx}
\indexsetup{othercode=\footnotesize}
\makeindex[columns=3, title={이야기 찾아보기}]
\let\oldindex\index
\renewcommand*{\index}[1]{\oldindex{#1}\ignorespaces}
\makeatletter
\def\indexspace{}
\makeatother

\usepackage{fontspec}
\setmainfont[ItalicFont={*},ItalicFeatures={FakeSlant=.167}]{NanumBarunGothic}
\usepackage[hidelinks,unicode,bookmarks=true]{hyperref}
\usepackage[usenames,dvipsnames]{color}
\usepackage[round]{natbib}
\hypersetup{colorlinks,
	citecolor=Red,
	linkcolor=Green,
	urlcolor=Blue}

\usepackage[nobottomtitles]{titlesec}
\titleformat*{\section}{\LARGE\bfseries}
\titleformat*{\subsection}{\Large\bfseries}
\titleformat*{\subsubsection}{\Large\bfseries}

\usepackage{fancyvrb}
\usepackage{graphicx}
\usepackage{subfig}
\usepackage{amsmath}
\usepackage{amsthm}
\usepackage{amssymb}
%\usepackage{relsize}
\usepackage{centernot}
\usepackage[top=2cm, left=2cm, right=2cm, bottom=2cm]{geometry}
\usepackage{titling}
%\usepackage{lipsum}
\usepackage{standalone}
\usepackage{enumitem}
\setlist{nolistsep}%{topsep=0pt,itemsep=0pt,parsep=0pt,before=\vspace{0mm},after=\vspace{0mm}}

\usepackage{etoolbox}
\AfterEndEnvironment{enumerate}{\vskip-\lastskip}
\AfterEndEnvironment{itemize}{\vskip-\lastskip}

\usepackage{multirow}
\usepackage{ulem}
\usepackage{bm}
\usepackage{tikz}
\usepackage{tabularx}
\usepackage{mdframed}
\usepackage{etoolbox}

\usepackage[
type={CC},
modifier={by-nc-sa},
version={4.0},
]{doclicense}

\usepackage[yyyymmdd]{datetime}

\usepackage{makecell}

\usepackage{minitoc}
\noptcrule
\doparttoc

\usepackage{chngcntr}
\counterwithin*{chapter}{part}

\usepackage{setspace}
\renewcommand{\baselinestretch}{1.2}

\setlength{\droptitle}{-3em}

\renewcommand\mtcgapbeforeheads{0pt}

\usepackage{arydshln}

\usepackage{xifthen}
\usepackage{xparse}

\makeatletter
\newcommand\footnoteref[1]{\protected@xdef\@thefnmark{\ref{#1}}\@footnotemark}
\makeatother

\setlength\dashlinedash{0.5pt}
\setlength\dashlinegap{2.0pt}
\setlength\arrayrulewidth{0.3pt}
\newcommand{\basesepline}{\hdashline}

\newcommand{\widthratio}{0.975}

\newenvironment{tightcenter}{%
	\setlength\topsep{0pt}%
	\setlength\parskip{0pt}%
	\par\centering}{\par\noindent\ignorespacesafterend}

%environment for normal stories, appears on index
\newenvironment{story}[3][]
{ \ignorespaces \smallbreak \par
	\noindent
	\begin{minipage}{\linewidth}
		\begin{tightcenter}
			{\large \ifthenelse{\isempty{#1}}{\textbf{[#2]}}{\hypertarget{#1}{\textbf{[#2]}}}}\\[1ex]
			\tabular{|p{\widthratio\linewidth}|}
			\hline
			\textbf{속성}: #3
			\index{#2}
			\\\hline
			\ignorespaces\unskip
		}
		{
			\endtabular
		\end{tightcenter}
	\end{minipage}
	\par
	\smallbreak
}

%environment for spoiling stories(i.e. scenarios), does not appear on index
\newenvironment{spoiler}[3][]
{ \ignorespaces \smallbreak \par
	\noindent
	\begin{minipage}{\linewidth}
		\begin{tightcenter}
			{\large \ifthenelse{\isempty{#1}}{\textbf{[#2]}}{\hypertarget{#1}{\textbf{[#2]}}}}\\[1ex]
			\tabular{|p{\widthratio\linewidth}|}
			\hline
			\textbf{속성}: #3
			\\\hline
			\ignorespaces\unskip
		}
		{
			\endtabular
		\end{tightcenter}
	\end{minipage}
	\par
	\smallbreak
}

\newcommand{\storyref}[2]{\hyperlink{#1}{\textnormal{[#2]}}}

\newcounter{qnactr}

\newenvironment{faq}[1]
{
	\refstepcounter{qnactr}
	\par\smallbreak
	\textbf{\large Q\theqnactr. #1}
	\newline\rmfamily
	A\theqnactr.
}
{\par\smallbreak}

\newcommand{\statchange}[2]
{
	\textbf{스탯 #1}: #2
}

\newcommand{\entry}[2][\basesepline]
{
	\ignorespaces#2\ignorespaces\unskip\\#1
}

\newcommand{\pre}[2][\basesepline]
{
	\entry[#1]{\textbf{필요 조건}: \textit{#2}}
}

\newcommand{\cost}[2][\hline]
{
	\entry[#1]{\textbf{개연성 코스트}: #2}
}

\newcommand{\limitedtrauma}[3][\basesepline]
{
	\entry[#1]{\textbf{제약}\ifthenelse{\isempty{#2}}{}{\textbf{(#2)}}: #3}
}

\newcommand{\triggertrauma}[4][\basesepline]
{
	\entry[#1]{\textbf{트리거}\ifthenelse{\isempty{#2}}{}{\textbf{(#2)}}: #3\\\textbf{효과}: #4}
}

\newfontfamily\storyfont{Nanum Pen Script}
\newcommand{\flavour}[2][\basesepline]
{
	\entry[#1]{{\storyfont\large#2}}
}

\newcommand{\world}[1]{{\storyfont \large #1 \par}}

\setlength\parindent{0pt}

\renewenvironment{center}
{\smallskip\parskip=0pt\par\nopagebreak\centering}
{\par\noindent\ignorespacesafterend\smallskip}

\usepackage{fancyhdr}
\ifDLC
\pagestyle{fancy}
\fancyhf{}
\cfoot{\thepage}
\rhead{\textcolor{red}{DLC ENABLED}}
\lhead{\textcolor{red}{DLC ENABLED}}
\rfoot{\textcolor{red}{DLC ENABLED}}
\lfoot{\textcolor{red}{DLC ENABLED}}
\fi

\begin{document}
	\chapter{시놉시스와 시나리오 기본 정보}
		\documentclass{report}

\begin{document}
	\textbf{시나리오 이름}: ???(The Unliving)
	
	\textbf{권장 인원}: 1인 타이만, 2\textasciitilde4인 대립
	
	\textbf{트리거 워닝}: 밀폐 공간, 어둠
	
	\textbf{시놉시스}
	
	중세 시대의 작은 마을. 어느 주말, 예배를 보려고 모여있던 사람들이 어째서인지 좀비로 변하고 있었다. 마을에 있는 사람 중 좀비가 되지 않은 사람은 단 네 명.
	
	예배당에서 노래를 부르고 있던 성가단원.
	
	예배당에서 예배를 집전하고 있던 사제.
	
	은퇴해서 마을에서 쉬고 있다가 예배를 빼먹은 성기사.
	
	그리고 이 소식을 듣고 급하게 이 마을로 파견된 이단심판관.
	
	네 명의 생존자는 이 일을 조사해 해결하고자 합니다. 이 마을에 내린 저주를 이들은 해결할 수 있을까요?
\end{document}
	
	\chapter{등장인물들의 이야기(이야기꾼 공개용)}
		\documentclass{report}

\begin{document}
	\section*{성가단원}
	
	\begin{spoiler}{성가대}{[역할]}
		\entry[\hline]{노래를 부르거나 악기를 연주하여 음악을 통해 마법을 사용할 수 있다. 다음 노래를 사용할 수 있다:
		\begin{spoiler}{성가}{[노래]}
			\entry[\hline]{한 턴에 이 노래를 부르기로 선택한다면 이동을 제외한 다른 행동을 할 수 없다. 턴이 종료될 때, 같은 구역에 있는 모든 생명체의 체력을 1 회복하고, 언데드에게 [신성] 피해를 1 준다.}
		\end{spoiler}
		}
	\end{spoiler}
	
	\section*{사제}
	
	\begin{spoiler}{기도}{[역할]}
		\entry[\hline]{사거리에 상관 없이 대상을 정한다. 대상의 다음 턴이 시작될 때, 대상의 체력을 2 회복시키거나, [신성] 피해를 2 준다.}
	\end{spoiler}
	
	\section*{성기사}
	
	\begin{spoiler}{신성한 일격}{[역할]}
		\entry[\hline]{사거리에 상관 없이 대상을 정한다. 대상에게 [신성] 피해를 2 주고, [기절] 상태에 빠트린다. [기절] 상태의 상대는 다음 턴 모든 행동이 불가능하다.}
	\end{spoiler}
	
	\section*{이단심판관}
	
	\begin{spoiler}{심판}{[역할]}
		\entry[\hline]{씬이 시작할 때, [심판]의 대상을 하나 정한다. 해당 대상에게 주는 모든 피해가 1 증가한다.}
	\end{spoiler}
\end{document}
	
	\chapter{서사의 이야기(이야기꾼 공개용)}
		\documentclass{report}

\begin{document}
	
\end{document}
	
	\chapter*{스포일러 방지 및 메모용 빈 페이지 입니다. 본 시나리오를 플레이하실 분께서는 파일을 닫아주시길 부탁드립니다.}
	
	\parttoc
	
	\chapter{시나리오의 흐름}
		\documentclass{report}

\begin{document}
	
\end{document}
	
	\chapter{등장인물들의 진실된 이야기}
		\documentclass{report}

\begin{document}
	
\end{document}
	
	\chapter{장소별 이야기들}
		\documentclass{report}

\begin{document}
	
\end{document}
	
	\chapter{대체 세계선}
		\documentclass{report}

\begin{document}
	\hypertarget{alternative:no-criminal}{}
	\section{사제 역할의 소유자가 흑막 역할을 거부한 경우}
	
	\hypertarget{alternative:war-ready}{}
	\section{호전적인 이야기꾼들이 등장인물로서 들어온 경우}
	
\end{document}
	
	\chapter{좀비}
		\documentclass{report}

\begin{document}
	
\end{document}
		
	\chapter{패치 노트}
		\documentclass{report}

\begin{document}
	
\end{document}
\end{document}