\documentclass[11pt]{report}
%Use \documentclass[12pt]{book} to printout and make it into a book
\renewcommand{\numberline}[1]{#1~}

\newif\ifDLC
\DLCfalse %\DLCtrue or \DLCfalse

\newif\iffullchangelog
\fullchangelogfalse %\fullchangelogtrue or \fullchangelogfalse

\newcommand{\version}{0.0.0}

%\usepackage{kotex}
\usepackage{CJKutf8}

\usepackage[utf8]{inputenc}
\usepackage[OT1, T2A]{fontenc}

\usepackage{imakeidx}
\indexsetup{othercode=\footnotesize}
\makeindex[columns=3, title={이야기 찾아보기}]
\let\oldindex\index
\renewcommand*{\index}[1]{\oldindex{#1}\ignorespaces}
\makeatletter
\def\indexspace{}
\makeatother

\usepackage{fontspec}
\setmainfont[ItalicFont={*},ItalicFeatures={FakeSlant=.167}]{NanumBarunGothic}
\usepackage[hidelinks,unicode,bookmarks=true]{hyperref}
\usepackage[usenames,dvipsnames]{color}
\usepackage[round]{natbib}
\hypersetup{colorlinks,
	citecolor=Red,
	linkcolor=Green,
	urlcolor=Blue}

\usepackage[nobottomtitles]{titlesec}
\titleformat*{\section}{\LARGE\bfseries}
\titleformat*{\subsection}{\Large\bfseries}
\titleformat*{\subsubsection}{\Large\bfseries}

\usepackage{fancyvrb}
\usepackage{graphicx}
\usepackage{subfig}
\usepackage{amsmath}
\usepackage{amsthm}
\usepackage{amssymb}
%\usepackage{relsize}
\usepackage{centernot}
\usepackage[top=2cm, left=2cm, right=2cm, bottom=2cm]{geometry}
\usepackage{titling}
%\usepackage{lipsum}
\usepackage{standalone}
\usepackage{enumitem}
\setlist{nolistsep}%{topsep=0pt,itemsep=0pt,parsep=0pt,before=\vspace{0mm},after=\vspace{0mm}}

\usepackage{etoolbox}
\AfterEndEnvironment{enumerate}{\vskip-\lastskip}
\AfterEndEnvironment{itemize}{\vskip-\lastskip}

\usepackage{multirow}
\usepackage{ulem}
\usepackage{bm}
\usepackage{tikz}
\usepackage{tabularx}
\usepackage{mdframed}
\usepackage{etoolbox}

\usepackage[
type={CC},
modifier={by-nc-sa},
version={4.0},
]{doclicense}

\usepackage[yyyymmdd]{datetime}

\usepackage{makecell}

\usepackage{minitoc}
\noptcrule
\doparttoc

\usepackage{chngcntr}
\counterwithin*{chapter}{part}

\usepackage{setspace}
\renewcommand{\baselinestretch}{1.2}

\setlength{\droptitle}{-3em}

\renewcommand\mtcgapbeforeheads{0pt}

\usepackage{arydshln}

\usepackage{xifthen}
\usepackage{xparse}

\makeatletter
\newcommand\footnoteref[1]{\protected@xdef\@thefnmark{\ref{#1}}\@footnotemark}
\makeatother

\setlength\dashlinedash{0.5pt}
\setlength\dashlinegap{2.0pt}
\setlength\arrayrulewidth{0.3pt}
\newcommand{\basesepline}{\hdashline}

\newcommand{\widthratio}{0.975}

\newenvironment{tightcenter}{%
	\setlength\topsep{0pt}%
	\setlength\parskip{0pt}%
	\par\centering}{\par\noindent\ignorespacesafterend}

%environment for normal stories, appears on index
\newenvironment{story}[3][]
{ \ignorespaces \smallbreak \par
	\noindent
	\begin{minipage}{\linewidth}
		\begin{tightcenter}
			{\large \ifthenelse{\isempty{#1}}{\textbf{[#2]}}{\hypertarget{#1}{\textbf{[#2]}}}}\\[1ex]
			\tabular{|p{\widthratio\linewidth}|}
			\hline
			\textbf{속성}: #3
			\index{#2}
			\\\hline
			\ignorespaces\unskip
		}
		{
			\endtabular
		\end{tightcenter}
	\end{minipage}
	\par
	\smallbreak
}

%environment for spoiling stories(i.e. scenarios), does not appear on index
\newenvironment{spoiler}[3][]
{ \ignorespaces \smallbreak \par
	\noindent
	\begin{minipage}{\linewidth}
		\begin{tightcenter}
			{\large \ifthenelse{\isempty{#1}}{\textbf{[#2]}}{\hypertarget{#1}{\textbf{[#2]}}}}\\[1ex]
			\tabular{|p{\widthratio\linewidth}|}
			\hline
			\textbf{속성}: #3
			\\\hline
			\ignorespaces\unskip
		}
		{
			\endtabular
		\end{tightcenter}
	\end{minipage}
	\par
	\smallbreak
}

\newcommand{\storyref}[2]{\hyperlink{#1}{\textnormal{[#2]}}}

\newcounter{qnactr}

\newenvironment{faq}[1]
{
	\refstepcounter{qnactr}
	\par\smallbreak
	\textbf{\large Q\theqnactr. #1}
	\newline\rmfamily
	A\theqnactr.
}
{\par\smallbreak}

\newcommand{\statchange}[2]
{
	\textbf{스탯 #1}: #2
}

\newcommand{\entry}[2][\basesepline]
{
	\ignorespaces#2\ignorespaces\unskip\\#1
}

\newcommand{\pre}[2][\basesepline]
{
	\entry[#1]{\textbf{필요 조건}: \textit{#2}}
}

\newcommand{\cost}[2][\hline]
{
	\entry[#1]{\textbf{개연성 코스트}: #2}
}

\newcommand{\limitedtrauma}[3][\basesepline]
{
	\entry[#1]{\textbf{제약}\ifthenelse{\isempty{#2}}{}{\textbf{(#2)}}: #3}
}

\newcommand{\triggertrauma}[4][\basesepline]
{
	\entry[#1]{\textbf{트리거}\ifthenelse{\isempty{#2}}{}{\textbf{(#2)}}: #3\\\textbf{효과}: #4}
}

\newfontfamily\storyfont{Nanum Pen Script}
\newcommand{\flavour}[2][\basesepline]
{
	\entry[#1]{{\storyfont\large#2}}
}

\newcommand{\world}[1]{{\storyfont \large #1 \par}}

\setlength\parindent{0pt}

\renewenvironment{center}
{\smallskip\parskip=0pt\par\nopagebreak\centering}
{\par\noindent\ignorespacesafterend\smallskip}

\usepackage{fancyhdr}
\ifDLC
\pagestyle{fancy}
\fancyhf{}
\cfoot{\thepage}
\rhead{\textcolor{red}{DLC ENABLED}}
\lhead{\textcolor{red}{DLC ENABLED}}
\rfoot{\textcolor{red}{DLC ENABLED}}
\lfoot{\textcolor{red}{DLC ENABLED}}
\fi

\begin{document}
	\chapter{시놉시스와 시나리오 기본 정보}
		\documentclass{report}

\begin{document}
	\textbf{시나리오 이름}: 나갈 수 없는 탑
	
	\textbf{시나리오 작가}: 소낙(\href{https://twitter.com/knock_tr}{@knock\_tr})
	
	\textbf{사용 룰}: 이야기꾼의 세계(World of the Storytellers)
	
	\textbf{권장 인원}: 1인
	
	\textbf{트리거 워닝}: 이 시나리오는 플레이어에게 감정적 고통을 유발시킬 수 있는 내용을 포함하고 있습니다. 이에 대해 플레이어에게 충분히 숙지시키지 않고 플레이하는 행위를 금합니다.
	
	\subsubsection*{시놉시스}
	
	이야기꾼은 어떤 마법사의 제자 역할이 되어 탑에 들어가게 됩니다. 마법사의 일을 돕는 척 하며 마법사의 탑에 숨겨진 마석을 찾아야 합니다.
\end{document}

	
	\chapter{등장인물들의 이야기(이야기꾼 공개용)}
		\documentclass{report}

\begin{document}
	\section*{마법사의 제자}
	
	\begin{story}{마법사의 제자}{[역할]}
		\flavour{나는 제자인가, 잡일꾼인가……. 바쁘다는 핑계로 아무것도 가르쳐 주지 않은 마법사는 이번 논문의 집필이 끝나면 꼭 수업을 시작해 주겠다고 약속했다. "이번에는 진짜죠?"}
		\entry{\statchange{+}{의지}}
	\end{story}
\end{document}

	
	\chapter{서사의 이야기(이야기꾼 공개용)}
		\documentclass{report}

\begin{document}
	이 시나리오에는 조사해야할 것 같은 대상은 많아보이지만 실제 진상에 접근하는 데에 사용할 수 있는 정보는 적습니다. 그렇기 때문에, \storyref{system:tutorial}{튜토리얼} 이야기의 사용을 권장합니다.
	
	\begin{story}{어둠의 시대}{[중세]}
		\flavour{마녀 사냥이 일어나는 어둠의 시대.}
		
		\entry{현대적인 기술력을 사용하거나, 신성력 외의 마법을 사용하기 위해서는 반드시 [추방] 판정을 거쳐야 한다.}
		
		\entry[\hline]{[추방] 판정에 실패하면, 개연성에 피해를 받는 대신 해당 이야기를 이 서사의 결말을 맞기 전까지 봉인하는 것으로 대체할 수 있다.}
	\end{story}
	
	\begin{story}{몰입}{[이야기의 의지]}
		\entry{등장인물들이 알고 있는 것들 중 \textbf{공개 조건}이 존재하는 지식 또는 이야기들은 등장인물의 자존심이나 비밀과 매우 크게 연관이 있기 때문에 서사 상으로 스스로 밝힐 이유가 없는 정보들이기 때문에, 스스로 밝힐 수 없다. 이를 스스로 밝힌다면, [침범] 판정의 난이도가 영구적으로 1 증가하며, 즉시 [침범] 판정을 해, 실패한다면 서사 속에서 얻은 것이 아닌 무작위 이야기가 하나 봉인된다.}
	\end{story}
\end{document}
	
	\chapter*{스포일러 방지 및 메모용 빈 페이지 입니다. 본 시나리오를 플레이하실 분께서는 파일을 닫아주시길 부탁드립니다.}
	
	\parttoc
	
	\chapter{시나리오의 흐름}
		\documentclass{report}

\begin{document}
	\section{태초의 이야기(선택)}
		태초의 이야기에서 시스템은 네 이야기꾼을 불러, 서사의 기본 정보와 공개 이야기에 대해 이야기합니다. 여기에서 네 가지 역할, 즉 \hyperlink{cursed-bard}{성가단원}, \hyperlink{cowardly-priest}{사제}, \hyperlink{corrupt-paladin}{타락한}, \hyperlink{hurt-rogue}{이단심판관}을 부여합니다. 이 때, 네 가지 역할을 네 명이 시스템과 따로 만나서 나눠줘도 되고, 네명이 같이 시스템과 모여서 각자 고르도록 해도 상관 없습니다.
		
		개인적으로는 시스템이 네 개의 역할을 지정하는 것을 추천드립니다. \hyperlink{cowardly-priest}{사제}를 받은 이에게 흑막이 본인이라는 사실을 밝혀야 하기 때문이기도 하고, 동의를 구해야 하기 때문입니다. 동의하지 않은 경우에는 "역할은 역할일 뿐"이라는 점을 명시하고, \hyperlink{alternative:no-criminal}{흑막 거부시의 대체 세계선}을 따라서 진행하시면 됩니다.
	
	\section{마을의 구조}
		마을은 총 마을 외곽, 마을 내부, 교회의 세 겹으로 구성되어 있습니다.
	
	\section{다시, 태초의 이야기(선택)}
	
\end{document}
	
	\chapter{등장인물들의 진실된 이야기}
		\documentclass{report}

\begin{document}
	개별 이야기꾼들이 이미 알고 있는 정보를 전달하고, 얻을 수 있는 이야기들과 획득을 위한 필요 조건, 그리고 칭호의 효과와 칭호의 \textbf{공개 조건}과 \textbf{제거 조건}을 명백하게 밝혀야 합니다. 이 이야기들은 페이지 단위로 잘려 있어, 정보를 제공할 때 페이지 단위로 제공할 수 있게 했습니다.
	
	\pagebreak \hypertarget{cursed-bard}{}
	\section{저주받은 성가단원}
		\documentclass{report}

\begin{document}
	\subsection*{알고 있는 정보}
		당신은 사람들이 처음으로 언데드로 변하는 것을 본 꼬마아이입니다. 당신의 보호자 역시 좀비로 변해버렸고요.
		
		최근 들어 성가단의 단장이자 당신에게 오르간을 연주하는 법을 가르쳐준 당신의 보호자는 당신이 잘 때 밤늦게 어딘가로 향하는 일이 잦아졌습니다. 공책을 들고가는거로 봐서는 뭔가 적을것이 있는 것 같은데, 그게 무엇일까요?
	
	\subsection*{가지고 시작하는 이야기}
		\begin{spoiler}[choir:bard]{성가대}{[역할]}
			\entry{노래를 부르거나 악기를 연주하여 음악을 통해 마법을 사용할 수 있다. 아래 노래들을 사용할 수 있다.}
			
			\entry{
				\begin{spoiler}{성가}{[노래]}
					\entry[\hline]{한 턴에 이 노래를 부르기로 선택한다면 이동을 제외한 다른 행동을 할 수 없다. 턴이 종료될 때, 같은 구역에 있는 모든 생명체의 체력을 1 회복하고, 언데드에게 [신성] 피해를 1 준다.}
				\end{spoiler}
			}
			
			\entry[\hline]{\statchange{+}{지식:음악[2]}}
		\end{spoiler}
	
	\subsection*{획득 가능한 이야기}
		\begin{spoiler}{저주받은}{[공포][칭호]}
			\pre{언데드에게 효과가 적용되도록 \storyref{choir:bard}{성가대}의 노래를 부른다.}
			
			\limitedtrauma{공포}{\storyref{choir:bard}{성가대}의 노래들의 효과가 언데드를 상대로는 적용되지 않는다.}
			
			\entry[\hline]{\textbf{제거 조건}: 특정 아이템을 소유하고 있는 동안, 또는 어떤 사실을 알게 되면 이 칭호를 무시한다. 이 시점은 시스템이 알려준다.}
		\end{spoiler}
	
	\bigskip
	
	\storyref{choir:bard}{성가대}로 사용할 수 있는 더 많은 노래는 성가단의 연습장소에 있는 악보들으로 알 수 있습니다. 최대 한 곡을 추가로 기억할 수 있고, 그 이상을 기억하기 위해서는 악보를 직접 소유하고 있어야만 합니다.
\end{document}%
	
	\pagebreak \hypertarget{corrupt-paladin}{}
	\section{타락한 성기사}
		\documentclass{report}

\begin{document}
	\subsection*{알고 있는 정보}
		이 마을에 오래전에 살고 있던 사람 중에는 전설적인 작곡가이자 사제가 있었습니다. 마을의 오르간 연주자의 부탁을 받아 그 사람의 유작을 찾아나선 당신은 그의 무덤을 파헤쳤고, 그의 관에 새겨져있던 노래를 하나 찾게 되었습니다. 좀비 사태가 발생했을 때, 당신은 무덤의 뒤처리를 하고 있었고요. 당신은 이로 인해 당신이 타락하게 되었다는 사실을 알고 있기 때문에, 다른 등장인물들이 무덤을 발견하기 전에는 이 사실을 최대한 숨기고자 합니다.
		
		당신의 집에는 이 무덤을 파헤쳐서 젖은 흙이 묻은 삽이 현관문 뒤에 있고, 당신이 사용했던 방패와 십자가에는 각각 [흑마법사의 피]와 [순수한 피]가 작은 병에 담긴 채로 각각의 안의 비어있는 공간에 숨겨져 있습니다. [순수한 피]는 수년 전, 당신의 보호 하에 있었지만 지키지 못했던 첫 번째 이의 것으로, [순수한 피] 끼리는 잘 섞인다는 성질을 가지고 있기에 이를 이용해 숨겨두었던 자신의 아기를 찾고, 아기를 보호해달라는 부탁을 받았습니다.
	
	\subsection*{가지고 시작하는 이야기}
		\begin{story}[paladin:fallen]{타락한}{[공포][칭호]}
			\limitedtrauma{공포}{\storyref{paladin:smite}{신성한 일격}을 사용할 수 없다.}
			
			\entry{\textbf{제거 조건}: 자신의 집 안에 있는 [순수한 피]를 마시거나, 특정한 음악을 듣는다.}
		\end{story}
		
		\begin{story}[paladin:smite]{신성한 일격}{[역할]}
			\entry{세 턴에 한 번 사용할 수 있다. 사거리에 상관 없이 대상을 정한다. 대상에게 [신성] 피해를 2 주고, [기절] 상태에 빠트린다. [기절] 상태의 상대는 다음 턴 모든 행동이 불가능하다.}
		\end{story}
	
	\subsection*{획득 가능한 이야기}
		\begin{story}[search:paladin-shield]{낡은 방패}{[튜토리얼:조사대상]}
			\entry{손잡이 부분에 숨겨진 유리병에 다음이 있습니다:
				\begin{story}[paladin-shield:dark-blood]{흑마법사의 피}{[타락]}
					\flavour{흑마법사의 피입니다. 과거, 성기사에 의해 처단되었습니다.}
					
					\entry{자신이 있거나 인접한 구역에 이 피를 흩뿌릴 수 있다. 다음 턴에, 좀비는 이동하지 않는다고 하더라도 반드시 해당 칸을 향해서 이동하며, 공격하지 않는다.}
				\end{story}
			}
		\end{story}
		
		\begin{story}[search:paladin-cross]{십자가}{[튜토리얼:조사대상]}
			\entry{내부에 숨겨진 유리병에 다음이 있습니다:
				\begin{story}[paladin-cross:pure-blood]{순수한 피}{[신성]}
					\flavour{순수한 인간의 피입니다. 과거, 성기사는 이 사람을 지키는 데에 실패했습니다.}
					
					\entry{자신이 있거나 인접한 구역에 이 피를 흩뿌릴 수 있다. 해당 구역의 모든 좀비는 영구히 행동을 멈춘다.}
				\end{story}
			}
		\end{story}
		
		\begin{story}[paladin:protect]{권능의 보호막}{[신성]}
			\pre{\storyref{paladin:fallen}{타락한} 칭호 제거, \storyref{search:paladin-shield}{낡은 방패} 장착}
			
			\entry{세 턴에 한 번 사용할 수 있다. 다음 자신의 턴까지, 타락한 자들과 그 영향을 막아주는 방벽을 자신이 있는 구역에 칠 수 있다. 해당 칸에 있는 타락한 존재는 무작위 인접한 구역으로 밀려난다.}
		\end{story}

	\subsection*{직업 퀘스트}
	\begin{story}[quest:paladin]{성기사}{[직업]}
		\entry{\statchange{+}{근력, 의지}}
		
		\dual{\begin{story}{순수한 지식}{[퀘스트]}
				
				\entry{\textbf{성공 조건}
					
					\storyref{paladin-cross:pure-blood}{순수한 피}를 얻는다.}
				
				\entry{\textbf{보상}
					
					\statchange{+}{지식:신성[2]}}
				
		\end{story}}
		{\begin{story}{타락한 지식}{[퀘스트]}
				
				\entry{\textbf{성공 조건}
					
					\storyref{paladin-shield:dark-blood}{흑마법사의 피}를 얻는다.}
				
				\entry{\textbf{보상}
					
					\statchange{+}{지식:흑마법[2]}}
				
		\end{story}}
	\end{story}
\end{document}%
	
	\pagebreak \hypertarget{cowardly-priest}{}
	\section{겁에 질린 사제}
		\documentclass{report}

\begin{document}
	\subsection*{알고 있는 정보}
		당신이 이 좀비 사태의 원흉입니다. 당신은 사실 악마를 숭배하는 이교도이며, 이 악마를 숭배하고 소환하기 위해 악마의 가르침을 담은 책을 성경 표지만 덧씌워두었습니다. 당신은 이런 노력을 통해 악마와 계약해 [타락의 노래]를 얻었고, 이를 이용해 사람들을 좀비로 바꾸었습니다.
		
		당신의 방 안의 침대 밑에는 당신의 피를 이용하거나, 종 안에 숨겨진 열쇠를 사용해서만 안에 들어있는 해골과 악마에게서 받은 [타락의 노래]를 발견할 수 있다는 사실을 알고 있습니다. 상자가 부서지면 수면가스가 나오도록 되어 있어 어느 정도의 보호조치를 해 두었습니다.
		
		이 모든 이야기는 예배당에 진입하기 이전까지 자의적으로 공개할 수 없습니다.
	
	\subsection*{가지고 시작하는 이야기}
		\begin{spoiler}{겁에 질린}{[공포][칭호]}
			\limitedtrauma{공포}{\storyref{cleric:prayer}{기도}로 언데드에게 피해를 줄 수 없다.}
			
			\entry[\hline]{\textbf{제거 조건}: 예배당에 진입한다.}
		\end{spoiler}
		
		\begin{spoiler}[cleric:prayer]{기도}{[역할]}
			\entry[\hline]{사거리에 상관 없이 대상을 정한다. 대상의 다음 턴이 시작될 때, 생명체인 대상의 체력을 2 회복시키거나, 언데드인 대상에게 [신성] 피해를 2 준다.}
		\end{spoiler}
	
	\subsection*{획득 가능한 이야기}
		악마 숭배 사실을 들키면, \storyref{cleric:prayer}{기도}를 포함한 언데드와 생명체에 서로 다른 효과를 주는 능력 모두의 언데드에 대한 효과와 생명체에 대한 효과가 뒤바뀌며, 언데드에게 [신성] 피해를 주는 기술의 경우 생명체에게 [타락] 피해를 줍니다\footnote{[기도]의 경우, 이제 언데드의 체력을 2 회복하거나, 생명체인 대상에게 [타락] 피해를 2 줍니다.}.
		
		또한, 예배당에 진입하면 자동으로 왼쪽의 오르간에 앉으며, 다음 이야기를 얻습니다:
		\begin{spoiler}{타락의 연주자}{[역할]}
			\entry{좀비는 당신의 명령을 따르며, 당신은 특정 효과가 생명체와 언데드에게 서로 다른 효과를 발휘한다면, 어느 쪽을 따를지를 선택할 수 있다.}
			
			\entry{매 자신의 턴에, 예배당의 의자에 앉은 좀비 하나를 일으켜 세워, 좀비 대열에 합류시킬 수 있다.}
			
			\entry[\hline]{방어적인 행동을 제외한 모든 행동을 할 수 없다.}
		\end{spoiler}
\end{document}%
	
	\pagebreak \hypertarget{hurt-rogue}{}
	\section{부상당한 이단심판관}
		\documentclass{report}

\begin{document}
	\subsection*{알고 있는 정보}
		당신은 도둑입니다. 여러 마을을 돌아다니며 이단심판관인척 하며 그들의 재산을 훔쳤죠.
		
		이번에 당신이 노리는 것은 어떤 작곡가의 유작입니다. [정화의 노래]로 알려진 이 노래는 어느샌가 기억에서 사라졌지만, 이 마을 출신이었고 여기에 묻히기까지 했으니 흔적은 남아 있겠죠. 이 노래에 대해 조사한 결과, "순수한 피"를 가진 이가 연주해야만 효과가 있다고 하는데, 마침 이 마을에 "순수한 피"를 가진 이와 접촉한 성기사가 있다는 사실을 이전에 있던 마을의 흑마술사로부터 알아냈습니다. 이 마을에서 성기사의 도움을 얻고자 마을의 수장이나 다름없는 사제에게 도움을 청하기 위해, 이 두 명에 대한 뒷조사를 간략하게나마 하고 왔습니다. 물론, 불법이니만큼 들키기 전에는 말할 생각은 없지만요.
		
		당신의 가방 안에는 이 뒷조사 자료와 [정화의 노래]에 대한 자료, 그리고 당신이 사용하는 락픽과 단도, 만능툴들이 무기 주머니에 들어있습니다. 다른 사람들이 당신을 알아보지 못하기를, 그리고 이 가방을 열지 않기를 바라는수밖에는 없겠군요.
	
	\subsection*{가지고 시작하는 이야기}
		\begin{story}[rogue:hurt]{부상당한}{[공포][칭호]}
			\limitedtrauma{공포}{언데드를 대상으로 한 무기와 \storyref{rogue:judgment}{심판}의 사용이 불가능하다.}
			
			\entry{\textbf{제거 조건}: 자신의 진정한 정체를 다른 사람들이 추궁한다. 이 제거 조건은 공개할 수 없다.}
		\end{story}
		
		\begin{story}[rogue:judgment]{심판}{[역할]}
			\entry{\storyref{rogue:hurt}{부상당한}이 제거되면, 이 이야기를 잃는다. 이 조건은 공개할 수 없다.}
			
			\entry{씬이 시작할 때, [심판]의 대상을 하나 정한다. 해당 대상에게 주는 모든 피해가 1 증가한다.}
		\end{story}
	
	\subsection*{획득 가능한 이야기}
	\begin{story}[rogue:dagger]{단도 투척}{[역할]}
		\pre{칭호 \storyref{rogue:hurt}{부상당한} 제거}
		
		\entry{매 턴 한 번, 단도를 던질 수 있다. 단도는 떨어진 구역당 피해 1을 준다.}
	\end{story}
	
	\begin{story}[rogue:lockpick]{자물쇠 따기}{[역할]}
		\pre{칭호 \storyref{rogue:hurt}{부상당한} 제거}
		
		\entry{잠겨있는 물체를 30분을 소모하여 열 수 있습니다.}
	\end{story}
	
	\subsection*{직업 퀘스트}
		\begin{story}{이단심판관}{[직업]}
			\entry{\statchange{+}{기민[2]}}
			
			\entry{\begin{story}[rogue:protect-secret]{비밀의 수호자}{[퀘스트]}
						
						\entry{\textbf{성공 조건}
							\storyref{rogue:hurt}{부상당한}이 마을 내부에서 교회 구역으로 들어가기 전까지 제거당하지 않는다.
						}
						
						\entry{\textbf{보상}
							\storyref{rogue:judgment}{심판}을 \storyref{rogue:hurt}{부상당한}을 잃더라도 사용할 수 있으나, 이름이 [쇠약의 독]으로 바뀐다.
						}
						
			\end{story}}
			
			\entry{\begin{story}[rogue:ultimate-secret]{영원한 비밀}{[퀘스트]}
					
					\entry{\textbf{성공 조건}
						\storyref{rogue:hurt}{부상당한}을 교회 내부로 들어갈 때 까지 제거당하지 않는다.
					}
					
					\entry{\textbf{보상}
						\storyref{rogue:judgment}{심판}의 이름이 [이단의 심판]으로 바뀌며, 피해 증가량이 피해량의 50\%(최소 1)로 증가한다.
					}
					
			\end{story}}
		\end{story}
\end{document}%
	
\end{document}
	
	\chapter{장소별 이야기들}
		\input{./Unliving/장소.tex}
	
	\chapter{대체 세계선}
		\documentclass{report}

\begin{document}
	\hypertarget{alternative:no-criminal}{}
	\section{사제 역할의 소유자가 흑막 역할을 거부한 경우}
	
	\hypertarget{alternative:war-ready}{}
	\section{호전적인 이야기꾼들이 등장인물로서 들어온 경우}
	
\end{document}
	
	\chapter{좀비}
		\input{./Unliving/좀비.tex}
		
	\chapter{패치 노트}
		\documentclass{report}

\begin{document}
	\section*{2020. 01. 21. 시나리오 아이디어 생성 및 작성 시작}
	
	\section*{2020. 02. 22. 1차 베타테스트}
	베타테스트에 참여해주신 소낙님, 아르곤님, 엘링님, 엠케님께 감사를 드립니다.
\end{document}
\end{document}