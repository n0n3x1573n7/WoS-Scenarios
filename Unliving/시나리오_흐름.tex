\documentclass{report}

\begin{document}
	\section{태초의 이야기(선택)}
		태초의 이야기에서 시스템은 네 이야기꾼을 불러, 서사의 기본 정보와 공개 이야기에 대해 이야기합니다. 여기에서 네 가지 역할, 즉 \hyperlink{cursed-bard}{성가단원}, \hyperlink{cowardly-priest}{사제}, \hyperlink{corrupt-paladin}{타락한}, \hyperlink{hurt-rogue}{이단심판관}을 부여합니다. 이 때, 네 가지 역할을 네 명이 시스템과 따로 만나서 나눠줘도 되고, 네명이 같이 시스템과 모여서 각자 고르도록 해도 상관 없습니다.
		
		개인적으로는 시스템이 네 개의 역할을 지정하는 것을 추천드립니다. \hyperlink{cowardly-priest}{사제}를 받은 이에게 흑막이 본인이라는 사실을 밝혀야 하기 때문이기도 하고, 동의를 구해야 하기 때문입니다. 동의하지 않은 경우에는 "역할은 역할일 뿐"이라는 점을 명시하고, \hyperlink{alternative:no-criminal}{흑막 거부시의 대체 세계선}을 따라서 진행하시면 됩니다.
	
	\section{마을의 구조}
		마을은 총 마을 외곽, 마을 내부, 교회의 세 겹으로 구성되어 있습니다.
	
	\section{다시, 태초의 이야기(선택)}
	
\end{document}